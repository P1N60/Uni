\documentclass[a4paper,12pt]{article}
\usepackage{standalone}
\usepackage{amsmath}



% Basic
\usepackage[T1]{fontenc}
\usepackage[utf8]{inputenc}
\usepackage{titlesec}
\titleformat{\section}
  {\normalfont\fontsize{14}{15}\bfseries}{\thesection}{1em}{}
\titleformat{\subsection}
  {\normalfont\fontsize{12}{15}\bfseries}{\thesubsection}{1em}{}

% Changes sections from 1.1 to 1.a
\renewcommand{\thesubsection}{\thesection.\alph{subsection}}

% ------------------------------------------------------------ %

% Packages
\RequirePackage{tcolorbox}
\usepackage{amsmath, amsthm, amssymb}
\usepackage{blindtext}
\usepackage{enumitem}
\usepackage{extramarks}
\usepackage{fancyhdr}
\usepackage[margin=1in]{geometry}
\usepackage{graphicx}
\usepackage{hyperref}
\usepackage{indentfirst}
\usepackage{listings}
\usepackage{mathrsfs}
\usepackage{mdframed}
\usepackage{multicol, multirow}
\usepackage{needspace, setspace}
\usepackage{paracol}
\usepackage{pgf, pgfplots}
\usepackage{tikz}
\usetikzlibrary{patterns}
\usepackage{silence}
\usepackage{xcolor}
\usepackage{bookmark}
\setlength{\parindent}{0pt}
\usetikzlibrary{patterns}
% \usepackage{subcaption}
\usetikzlibrary{decorations.pathreplacing}
\usepackage{caption}
\usepackage{subcaption}
\usepackage{xcolor}
\definecolor{maroon}{RGB}{128, 0, 0}
% ------------------------------------------------------------ %

% Spacings
\newcommand{\n}{\vspace{3mm}} % Context spacing
\newcommand{\s}{\vspace{1mm}} % Equation spacing
\newcommand{\m}{\vspace{-3mm}} % Reverse context spacing
\newcommand{\propdisp}{\pagebreak} % Proper display page break
\newcommand{\br}{\n\\\n}

% Wordings
\newcommand{\ans}[1][zb]{{\color{#1}\textit{Answer. }\hspace{3mm}}} % Answer
\newcommand{\arl}[1][zr]{{\color{#1}$\brr{\Leftarrow}$\hspace{3mm}}} % Left arrow
\newcommand{\arr}[1][zr]{{\color{#1}$\brr{\Rightarrow}$\hspace{3mm}}} % Right arrow
\newcommand{\cse}[2][zr]{{\color{#1}\textit{Case #2: }\hspace{1mm}}} % Case
\newcommand{\clm}[2][zr]{{\color{#1}$\vdash_{#2}$\hspace{1mm}}} % Claim
\newcommand{\prf}[1][zr]{{\color{#1}\textit{Proof. }\hspace{3mm}}} % Proof
\newcommand{\prt}[2][a]{\hspace{-2mm}{\color{#2}\textit{Part (#1) }}\hspace{1mm}} % Part
\newcommand{\prtc}[2][a]{\hspace{2mm}\prt[#1]{#2}} % Continued part

\newcommand{\rdft}[1][\sct]{{\color{zg}\textit{Definition #1}}} % Refer definition
\newcommand{\rexm}[1][\sct]{{\color{zb}\textit{Example #1}}} % Refer example
\newcommand{\rfig}[1][\sct]{{\color{zy}\textit{Figure #1}}} % Refer figure
\newcommand{\rpst}[1][\sct]{{\color{zr}\textit{Proposition #1}}} % Refer proposition
\newcommand{\rthm}[1][\sct]{{\color{zr}\textit{Theorem #1}}} % Refer theorem

\newcommand{\sct}{\thesection.\thescount} % Counter
\newcommand{\sctr}[2][0]{\the\numexpr\value{section}-#1\relax.\the\numexpr\value{scount}-#2\relax} % Relative counter

% Equations
\newcommand{\C}{\mathbb{C}} % Complex
\newcommand{\F}{\mathbb{F}} % Field
\newcommand{\I}{\mathbb{I}} % Irrational
\newcommand{\N}{\mathbb{N}} % Natural
\newcommand{\Q}{\mathbb{Q}} % Rational
\newcommand{\R}{\mathbb{R}} % Real
\newcommand{\Z}{\mathbb{Z}} % Integer

\newcommand{\GL}{\mathrm{GL}} % General linear group
\newcommand{\SL}{\mathrm{SL}} % Special linear group

\newcommand{\abs}[1]{\left| #1\right|} % Absolute
\newcommand{\bra}[1]{\left\langle #1\right\rangle} % Angled brackets
\newcommand{\brc}[1]{\left\{ #1\right\}} % Curly brackets
\newcommand{\brr}[1]{\left( #1\right)} % Round brackets
\newcommand{\brs}[1]{\left[ #1\right]} % Square brackets
\newcommand{\cond}[1]{\left. #1\right|} % Condition with right bar
\newcommand{\diff}{\,\mathrm{d}} % Differential
\newcommand{\erm}[1]{\;\;\;\;\text{#1}} % Equation remarks
\newcommand{\nrm}[1]{\left\| #1\right\|} % Norm
\newcommand{\srm}{\,\mid\,} % Set remarks

% Math operators
\let\Im\relax
\let\Re\relax

\DeclareMathOperator{\Im}{Im} % Imaginary function
\DeclareMathOperator{\Re}{Re} % Real function

% ------------------------------------------------------------ %

% Colours
\definecolor{gr}{RGB}{120, 120, 120} % Grey
\definecolor{zb}{RGB}{0, 38, 77} % Blue
\definecolor{zg}{RGB}{0, 77, 51} % Green
\definecolor{zp}{RGB}{51, 0, 77} % Purple
\definecolor{zr}{RGB}{77, 0, 38} % Red
\definecolor{zy}{RGB}{77, 64, 0} % Yellow

% Graphing
\usetikzlibrary{arrows}
\usetikzlibrary{calc}
\usetikzlibrary{patterns}
\pgfplotsset{compat=1.15}

% ------------------------------------------------------------ %

% Remark
\newcommand{\remark}[1]{
  \noindent\textbf{Remarks}
  
  \begin{nlist}
    \item Context of this document is based on university course \textit{\gettitle} from \textit{Department of Mathematics, The Chinese University of Hong Kong}. The original source can be found at \url{https://www.math.cuhk.edu.hk/course}. The author does not own the source.
    \item This document is assumed unavailable for unauthorized parties that have not attended the university course. It is prohibited to share, including distributing or copying this document to unauthorized parties in any means for any non-academic purpose.
    \item Context of this doucment may not be completely accurate. The author assumes no responsibility or liability for any errors or omissions in the context of this document.
    \item This document is under license CC-BY-SA 4.0. It is allowed to make any editions on this document, as long as terms of the license is not violated.
    #1
  \end{nlist}
}

% Prerequisites
\newenvironment{prereq}{
  \noindent \textbf{Prerequisites}\n

  This course requires prerequisites of
}{
  \n
}

% Reference list
\newenvironment{reflist}{
  \begin{alist}
    \item Course material from various professors associated to \textit{\gettitle}
}{
  \end{alist}
}

% ------------------------------------------------------------ %

% Environments
\newcounter{scount}[section] % Counter

\newenvironment{crl}{ % Corollary
  \parindent 0pt
  \begin{siderule}[linecolor=zr]{\color{zr}\textit{Corollary. }}
}{
  \end{siderule}
}

\newenvironment{cmt}{ % Comment
  \parindent 0pt
  \begin{siderule}[linecolor=zp]{\color{zp}\textit{Comment. }}
}{
  \end{siderule}
}

\newenvironment{dft}{ % Definition
  \parindent 0pt
  \refstepcounter{scount}
  \begin{siderule}[linecolor=zg]{\color{zg}\textit{Definition \sct. }}
}{
  \end{siderule}
}
\newenvironment{lem}{ % Lemma
  \parindent 0pt
  \refstepcounter{scount}
  \begin{siderule}[linecolor=zg]{\color{zg}\textit{Lemma \sct. }}
}{
  \end{siderule}
}

\newenvironment{exm}{ % Example
  \parindent 0pt
  \refstepcounter{scount}
  \begin{siderule}[linecolor=zb]{\color{zb}\textit{Example \sct. }}
}{
  \end{siderule}
}

\newenvironment{fig}{ % Figure
  \parindent 0pt
  \refstepcounter{scount}
  \begin{siderule}[linecolor=zy]{\color{zy}\textit{Figure \sct. }}\n
  
}{
  \end{siderule}
}

\newenvironment{prv}{ % Proof
  \parindent 0pt
  \begin{siderule}[linecolor=zr]\prf
}{
  \end{siderule}
}

\newenvironment{pst}{ % Proposition
  \parindent 0pt
  \refstepcounter{scount}
  \begin{siderule}[linecolor=zr]{\color{zr}\textit{Proposition \sct. }}
}{
  \end{siderule}
}

\newenvironment{tcn}{ % Technique
  \parindent 0pt
  \begin{siderule}[linecolor=zp]{\color{zp}\textit{Technique. }}
}{
  \end{siderule}
}

\newenvironment{thm}{ % Theorem
  \parindent 0pt
  \refstepcounter{scount}
  \begin{siderule}[linecolor=zr]{\color{zr}\textit{Sætning \sct. }}
}{
  \end{siderule}
}

\newenvironment{rmatrix}{ % Matrix in round brackets
  \left(\begin{matrix}
}{
  \end{matrix}\right)
}

\newenvironment{alist}{ % Alphabetical list
  \begin{enumerate}[label=(\alph*)]
}{
  \end{enumerate}
}

\newenvironment{Alist}{ % Capitalized alphabetical list
  \begin{enumerate}[label=(\Alph*)]
}{
  \end{enumerate}
}

\newenvironment{nlist}{ % Number list
  \begin{enumerate}[label=(\arabic*)]
}{
  \end{enumerate}
}

\newenvironment{plist}{ % Point list
  \begin{itemize}
}{
  \end{itemize}
}

\newenvironment{rlist}{ % Roman list
  \begin{enumerate}[label=(\roman*)]
}{
  \end{enumerate}
}

\newmdenv[ % Siderule line
  topline=false,
  bottomline=false,
  rightline=false,
  rightmargin=0
]{siderule}


% ------------------------------------------------------------ %
% Warning filters
\WarningFilter{mdframed}{You got a bad break}
\hfuzz=8pt


% Load required packages
\usepackage{tcolorbox}
% ------------------------------------------------------------ %

% Code

\usepackage{amsmath}
\usepackage{graphicx}
\definecolor{bluekeywords}{rgb}{0.13,0.13,1}
\definecolor{greencomments}{rgb}{0,0.5,0}
\definecolor{redstrings}{rgb}{0.9,0,0}
\definecolor{bgcolor}{rgb}{0.95,0.94,0.94}
\usepackage{listings}
\usepackage{upquote}
\usepackage{xcolor}

\lstdefinelanguage{Python}
{
  keywords={typeof, null, catch, switch, in, int, str, float, self},
  keywordstyle=\color{ForestGreen}\bfseries,
  ndkeywords={boolean, throw, import},
  ndkeywords={return, class, if ,elif, endif, while, do, else, True, False , catch, def},
  ndkeywordstyle=\color{BrickRed}\bfseries,
  identifierstyle=\color{black},
  sensitive=false,
  comment=[l]{\#},
  morecomment=[s]{/*}{*/},
  commentstyle=\color{purple}\ttfamily,
  stringstyle=\color{red}\ttfamily,
}

\lstset
{ %Formatting for code in appendix
    language=Python,
    numbers=left,
    stepnumber=1,
    showstringspaces=false,
    tabsize=1,
    breaklines=true,
    breakatwhitespace=false,
    backgroundcolor=\color{bgcolor},  % Background color
    basicstyle=\ttfamily\footnotesize, % Code font size and style
    frame=single,                    % Adds a frame around the code
    rulecolor=\color{bgcolor},       % Frame color
    breaklines=true,                 % Breaks long lines
}


% Changes sections from 1.1 to 1.a
\renewcommand{\thesubsection}{\thesection.\alph{subsection}}
\graphicspath{ {../../pictures/IDMA/IDMA3_a}} 
\title{Københavns Universitet\\
LinAlgDat - Project B}
\author{Victor Vangkilde Jørgensen - kft410\\ 
kft410@alumni.ku.dk\\
Hold 13 Mach}

\begin{document}
\author{Victor Vangkilde Jørgensen}
\makeatletter
\let\getauthor\@author
\let\gettitle\@title
\makeatother
\maketitle
\thispagestyle{empty}
\n\n
 

\pagebreak
\pagestyle{empty}
\tableofcontents
\pagebreak
\pagestyle{fancy}
\fancyhf{}
\setlength{\headheight}{15.2pt}
\renewcommand{\footrulewidth}{0.4pt}
% \fancyhead[R]{\nouppercase \lastrightmark}
\fancyfoot[L]{\gettitle}
\fancyfoot[R]{\thepage}
 
\maketitle 

\section[Opgave]{Opgave}
\subsection{}

Vi kan aflæse $M_a$ til:\\
\[
\left[\begin{array}{ccc}
    a & -1 & -1 \\
    0 & (a-1) & -1 \\
    0 & 2 & (a+2)
\end{array}\right]
\]


\subsection{}
$
\left[\begin{array}{ccc|c}
    a & -1 & -1 & 0 \\
    0 & a-1 & -1 & 0 \\
    0 & 2 & a+2 & 0
\end{array}\right]
\begin{array}{ccc}
    \\
    \cdot \frac{1}{a-1} \\
    \\
\end{array}
\leadsto
\left[\begin{array}{ccc|c}
    a & -1 & -1 & 0 \\
    0 & 1 & -\frac{1}{a-1} & 0 \\
    0 & 2 & a+2 & 0
\end{array}\right]
\begin{array}{ccc}
    \\
    \\
    -2r_2\\
\end{array}
\leadsto
$\\
$
\left[\begin{array}{ccc|c}
    a & -1 & -1 & 0 \\
    0 & 1 & -\frac{1}{a-1} & 0 \\
    0 & 0 & \frac{a^2+a}{a-1} & 0
\end{array}\right]
\begin{array}{ccc}
    \\
    \\
    \cdot \frac{a-1}{a^2+a}\\
\end{array}
\leadsto
\left[\begin{array}{ccc|c}
    a & -1 & -1 & 0 \\
    0 & 1 & -\frac{1}{a-1} & 0 \\
    0 & 0 & 1 & 0
\end{array}\right]
\begin{array}{ccc}
    +r_3\\
    +r_3 \cdot \frac{1}{a-1}\\
    \\
\end{array}
\leadsto
$\\
$
\left[\begin{array}{ccc|c}
    a & -1 & 0 & 0 \\
    0 & 1 & 0 & 0 \\
    0 & 0 & 1 & 0
\end{array}\right]
\begin{array}{ccc}
    +r_2\\
    \\
    \\
\end{array}
\leadsto
\left[\begin{array}{ccc|c}
    a & 0 & 0 & 0 \\
    0 & 1 & 0 & 0 \\
    0 & 0 & 1 & 0
\end{array}\right]
\begin{array}{ccc}
    \cdot \frac{1}{a}\\
    \\
    \\
\end{array}
\leadsto
$\\
$
\left[\begin{array}{ccc|c}
    1 & 0 & 0 & 0 \\
    0 & 1 & 0 & 0 \\
    0 & 0 & 1 & 0
\end{array}\right]
$\\

$T_a$ er altså injektiv.\\
$T_a$ er surjektiv, da vi har 3 vektoerer. $T_a$ er dermed bijektiv, da den både er injektiv og surjektiv.\\

Vi bestemmer nu $T^{-1}_a$:

$
\left[\begin{array}{ccc|ccc}
    a & -1 & -1 & 1 & 0 & 0 \\
    0 & a-1 & -1 & 0 & 1 & 0 \\
    0 & 2 & a+2 & 0 & 0 & 1
\end{array}\right]
\begin{array}{ccc}
    \\
    \cdot \frac{1}{a-1} \\
    \\
\end{array}
\leadsto
\left[\begin{array}{ccc|ccc}
    a & -1 & -1 & 1 & 0 & 0 \\
    0 & 1 & -\frac{1}{a-1} & 0 & \frac{1}{a-1} & 0 \\
    0 & 2 & a+2 & 0 & 0 & 1
\end{array}\right]
\begin{array}{ccc}
    \\
    \\
    -2r_2\\
\end{array}
\leadsto
$\\
$
\left[\begin{array}{ccc|ccc}
    a & -1 & -1 & 1 & 0 & 0 \\
    0 & 1 & -\frac{1}{a-1} & 0 & \frac{1}{a-1} & 0 \\
    0 & 0 & \frac{a^2+a}{a-1} & 0 & -\frac{2}{a-1} & 1
\end{array}\right]
\begin{array}{ccc}
    \\
    \\
    \cdot \frac{a-1}{a^2+a}\\
\end{array}
\leadsto
\left[\begin{array}{ccc|ccc}
    a & -1 & -1 & 1 & 0 & 0 \\
    0 & 1 & -\frac{1}{a-1} & 0 & \frac{1}{a-1} & 0 \\
    0 & 0 & 1 & 0 & -\frac{2}{a^2+a} & \frac{a-1}{a^2+a}
\end{array}\right]
\begin{array}{ccc}
    +r_3\\
    +r_3 \cdot \frac{1}{a-1}\\
    \\
\end{array}
\leadsto
$\\
$
\left[\begin{array}{ccc|ccc}
    a & -1 & 0 & 1 & -\frac{2}{a^2+a} & \frac{a-1}{a^2+a} \\
    0 & 1 & 0 & 0 & \frac{a+2}{a^2+a} & \frac{1}{a^2+a} \\
    0 & 0 & 1 & 0 & -\frac{2}{a^2+a} & \frac{a-1}{a^2+a}
\end{array}\right]
\begin{array}{ccc}
    +r_2\\
    \\
    \\
\end{array}
\leadsto
\left[\begin{array}{ccc|ccc}
    a & 0 & 0 & 1 & \frac{1}{a+1} & \frac{1}{a+1} \\
    0 & 1 & 0 & 0 & \frac{a+2}{a^2+a} & \frac{1}{a^2+a} \\
    0 & 0 & 1 & 0 & -\frac{2}{a^2+a} & \frac{a-1}{a^2+a}
\end{array}\right]
\begin{array}{ccc}
    \cdot \frac{1}{a}\\
    \\
    \\
\end{array}
\leadsto
$\\
$
\left[\begin{array}{ccc|ccc}
    1 & 0 & 0 & \frac{1}{a} & \frac{1}{a^2+a} & \frac{1}{a^2+a} \\
    0 & 1 & 0 & 0 & \frac{a+2}{a^2+a} & \frac{1}{a^2+a} \\
    0 & 0 & 1 & 0 & -\frac{2}{a^2+a} & \frac{a-1}{a^2+a}
\end{array}\right]
$\\

\subsection{}
Vi opstill igen $T_a$, hvor $a = -1$:\\

$
\left[\begin{array}{ccc|c}
    -1 & -1 & -1 & 0 \\
    0 & -1-1 & -1 & 0 \\
    0 & 2 & -1+2 & 0
\end{array}\right]
\begin{array}{ccc}
    \\
    reducer\\
    \\
\end{array}
\leadsto
\left[\begin{array}{ccc|c}
    -1 & -1 & -1 & 0 \\
    0 & -2 & -1 & 0 \\
    0 & 2 & 1 & 0
\end{array}\right]
\begin{array}{ccc}
    \\
    \\
    -r_2\\
\end{array}
\leadsto
$\\
$
\left[\begin{array}{ccc|c}
    -1 & -1 & -1 & 0 \\
    0 & -2 & -1 & 0 \\
    0 & 0 & 2 & 0
\end{array}\right]
\begin{array}{ccc}
    \\
    \\
    \cdot \frac{1}{2}\\
\end{array}
\leadsto
\left[\begin{array}{ccc|c}
    -1 & -1 & -1 & 0 \\
    0 & -2 & -1 & 0 \\
    0 & 0 & 1 & 0
\end{array}\right]
\begin{array}{ccc}
    +r_3\\
    +r_3\\
    \\
\end{array}
\leadsto
$\\
$
\left[\begin{array}{ccc|c}
    -1 & -1 & 0 & 0 \\
    0 & -2 & 0 & 0 \\
    0 & 0 & 1 & 0
\end{array}\right]
\begin{array}{ccc}
    \cdot (-1)\\
    \cdot (-\frac{1}{2})\\
    \\
\end{array}
\leadsto
\left[\begin{array}{ccc|c}
    1 & 1 & 0 & 0 \\
    0 & 1 & 0 & 0 \\
    0 & 0 & 1 & 0
\end{array}\right]
\begin{array}{ccc}
    -r_2\\
    \\
    \\
\end{array}
\leadsto
$\\
$
\left[\begin{array}{ccc|c}
    1 & 0 & 0 & 0 \\
    0 & 1 & 0 & 0 \\
    0 & 0 & 1 & 0
\end{array}\right]
$\\

\subsection{}


\subsection{}

\section[Opgave]{Opgave}
\subsection{}
Vi opstiller et ligningssystem i form af en totalmatrix, hvor vi sætter $u_1, u_2, u_3$ lig hhv. $v_1, v_2, v_3$, og finder løsningerne til disse, ved brug af Gauss-Jordan elimination.\\

$u_1 + u_2 + u_3 = v_1 \Leftrightarrow$\\

$
\left[\begin{array}{ccc|c}
    2 & 0 & 1 & 7\\
    1 & 1 & -1 & -2\\
    -1 & -1 & 2 & 3\\
    1 & 1 & 2 & 1
\end{array}\right]
\begin{array}{ccc}
    \\
    \cdot 2\\
    \cdot (-2)\\
    \cdot 2\\
\end{array}
\leadsto
\left[\begin{array}{ccc|c}
    2 & 0 & 1 & 7\\
    2 & 2 & -2 & -4\\
    2 & 2 & -4 & -6\\
    2 & 2 & 4 & 2
\end{array}\right]
\begin{array}{ccc}
    \\
    -r_1\\
    -r_1\\
    -r_1\\
\end{array}
\leadsto
$\\
$
\left[\begin{array}{ccc|c}
    2 & 0 & 1 & 7\\
    0 & 2 & -3 & -11\\
    0 & 2 & -5 & -13\\
    0 & 2 & 3 & -5
\end{array}\right]
\begin{array}{ccc}
    \\
    \\
    -r_2\\
    -r_2\\
\end{array}
\leadsto
\left[\begin{array}{ccc|c}
    2 & 0 & 1 & 7\\
    0 & 2 & -3 & -11\\
    0 & 0 & -2 & -2\\
    0 & 0 & 6 & 6
\end{array}\right]
\begin{array}{ccc}
    \\
    \\
    \\
    +3r_3\\
\end{array}
\leadsto
$\\
$
\left[\begin{array}{ccc|c}
    2 & 0 & 1 & 7\\
    0 & 2 & -3 & -11\\
    0 & 0 & -2 & -2\\
    0 & 0 & 0 & 0
\end{array}\right]
\begin{array}{ccc}
    \\
    \\
    \cdot (-\frac{1}{2})\\
    \\
\end{array}
\leadsto
\left[\begin{array}{ccc|c}
    2 & 0 & 1 & 7\\
    0 & 2 & -3 & -11\\
    0 & 0 & 1 & 1\\
    0 & 0 & 0 & 0
\end{array}\right]
\begin{array}{ccc}
    -1r_3\\
    +3r_3\\
    \\
    \\
\end{array}
\leadsto
$\\
$
\left[\begin{array}{ccc|c}
    2 & 0 & 0 & 6\\
    0 & 2 & 0 & -8\\
    0 & 0 & 1 & 1\\
    0 & 0 & 0 & 0
\end{array}\right]
\begin{array}{ccc}
    \cdot \frac{1}{2}\\
    \cdot \frac{1}{2}\\
    \\
    \\
\end{array}
\leadsto
\left[\begin{array}{ccc|c}
    1 & 0 & 0 & 3\\
    0 & 1 & 0 & -4\\
    0 & 0 & 1 & 1\\
    0 & 0 & 0 & 0
\end{array}\right]
$\\

Vores første kolonne i $P_{B\leftarrow C}$ er dermed: 
$\left[\begin{array}{ccc}
    3\\
    -4\\
    1\\
    0\\
\end{array}\right]$\\

$u_1 + u_2 + u_3 = v_2 \Leftrightarrow$\\

$
\left[\begin{array}{ccc|c}
    2 & 0 & 1 & -1\\
    1 & 1 & -1 & 0\\
    -1 & -1 & 2 & -1\\
    1 & 1 & 2 & -3
\end{array}\right]
\begin{array}{ccc}
    \\
    \cdot 2\\
    \cdot (-2)\\
    \cdot 2\\
\end{array}
\leadsto
\left[\begin{array}{ccc|c}
    2 & 0 & 1 & -1\\
    2 & 2 & -2 & 0\\
    2 & 2 & -4 & 2\\
    2 & 2 & 4 & -6
\end{array}\right]
\begin{array}{ccc}
    \\
    -r_1\\
    -r_1\\
    -r_1\\
\end{array}
\leadsto
$\\
$
\left[\begin{array}{ccc|c}
    2 & 0 & 1 & -1\\
    0 & 2 & -3 & 1\\
    0 & 2 & -5 & 3\\
    0 & 2 & 3 & -5
\end{array}\right]
\begin{array}{ccc}
    \\
    \\
    -r_2\\
    -r_2\\
\end{array}
\leadsto
\left[\begin{array}{ccc|c}
    2 & 0 & 1 & -1\\
    0 & 2 & -3 & 1\\
    0 & 0 & -2 & 2\\
    0 & 0 & 6 & -6
\end{array}\right]
\begin{array}{ccc}
    \\
    \\
    \\
    +3r_3\\
\end{array}
\leadsto
$\\
$
\left[\begin{array}{ccc|c}
    2 & 0 & 1 & -1\\
    0 & 2 & -3 & 1\\
    0 & 0 & -2 & 2\\
    0 & 0 & 0 & 0
\end{array}\right]
\begin{array}{ccc}
    \\
    \\
    \cdot (-\frac{1}{2})\\
    \\
\end{array}
\leadsto
\left[\begin{array}{ccc|c}
    2 & 0 & 1 & -1\\
    0 & 2 & -3 & 1\\
    0 & 0 & 1 & -1\\
    0 & 0 & 0 & 0
\end{array}\right]
\begin{array}{ccc}
    -1r_3\\
    +3r_3\\
    \\
    \\
\end{array}
\leadsto
$\\
$
\left[\begin{array}{ccc|c}
    2 & 0 & 0 & 0\\
    0 & 2 & 0 & -2\\
    0 & 0 & 1 & -1\\
    0 & 0 & 0 & 0
\end{array}\right]
\begin{array}{ccc}
    \cdot \frac{1}{2}\\
    \cdot \frac{1}{2}\\
    \\
    \\
\end{array}
\leadsto
\left[\begin{array}{ccc|c}
    1 & 0 & 0 & 0\\
    0 & 1 & 0 & -1\\
    0 & 0 & 1 & -1\\
    0 & 0 & 0 & 0
\end{array}\right]
$\\

Vores anden kolonne i $P_{B\leftarrow C}$ er dermed: 
$\left[\begin{array}{ccc}
    0\\
    -1\\
    -1\\
    0\\
\end{array}\right]$\\

$u_1 + u_2 + u_3 = v_3 \Leftrightarrow$\\

$
\left[\begin{array}{ccc|c}
    2 & 0 & 1 & 3\\
    1 & 1 & -1 & -1\\
    -1 & -1 & 2 & 2\\
    1 & 1 & 2 & 2
\end{array}\right]
\begin{array}{ccc}
    \\
    \cdot 2\\
    \cdot (-2)\\
    \cdot 2\\
\end{array}
\leadsto
\left[\begin{array}{ccc|c}
    2 & 0 & 1 & 3\\
    2 & 2 & -2 & -2\\
    2 & 2 & -4 & -4\\
    2 & 2 & 4 & 4
\end{array}\right]
\begin{array}{ccc}
    \\
    -r_1\\
    -r_1\\
    -r_1\\
\end{array}
\leadsto
$\\
$
\left[\begin{array}{ccc|c}
    2 & 0 & 1 & 3\\
    0 & 2 & -3 & -5\\
    0 & 2 & -5 & -7\\
    0 & 2 & 3 & 1
\end{array}\right]
\begin{array}{ccc}
    \\
    \\
    -r_2\\
    -r_2\\
\end{array}
\leadsto
\left[\begin{array}{ccc|c}
    2 & 0 & 1 & 3\\
    0 & 2 & -3 & -5\\
    0 & 0 & -2 & -2\\
    0 & 0 & 6 & 6
\end{array}\right]
\begin{array}{ccc}
    \\
    \\
    \\
    +3r_3\\
\end{array}
\leadsto
$\\
$
\left[\begin{array}{ccc|c}
    2 & 0 & 1 & 3\\
    0 & 2 & -3 & -5\\
    0 & 0 & -2 & -2\\
    0 & 0 & 0 & 0
\end{array}\right]
\begin{array}{ccc}
    \\
    \\
    \cdot (-\frac{1}{2})\\
    \\
\end{array}
\leadsto
\left[\begin{array}{ccc|c}
    2 & 0 & 1 & 3\\
    0 & 2 & -3 & -5\\
    0 & 0 & 1 & 1\\
    0 & 0 & 0 & 0
\end{array}\right]
\begin{array}{ccc}
    -1r_3\\
    +3r_3\\
    \\
    \\
\end{array}
\leadsto
$\\
$
\left[\begin{array}{ccc|c}
    2 & 0 & 0 & 2\\
    0 & 2 & 0 & -2\\
    0 & 0 & 1 & 1\\
    0 & 0 & 0 & 0
\end{array}\right]
\begin{array}{ccc}
    \cdot \frac{1}{2}\\
    \cdot \frac{1}{2}\\
    \\
    \\
\end{array}
\leadsto
\left[\begin{array}{ccc|c}
    1 & 0 & 0 & 1\\
    0 & 1 & 0 & -1\\
    0 & 0 & 1 & 1\\
    0 & 0 & 0 & 0
\end{array}\right]
$\\

Vores sidste kolonne i $P_{B\leftarrow C}$ er dermed: 
$\left[\begin{array}{ccc}
    1\\
    -1\\
    1\\
    0\\
\end{array}\right]$\\

Sammensætter vi nu vores tre kolonner til en matrix, får vi:\\
\[
P_{B\leftarrow C} = 
\left[\begin{array}{ccc}
    3 & 0 & 1\\
    -4 & -1 & -1\\
    1 & -1 & 1\\
\end{array}\right]
\]

\subsection{}
\[
x = 
\left[\begin{array}{ccc}
    7\\
    -2\\
    3\\
    1
\end{array}\right]
+
\left[\begin{array}{ccc}
    -1\\
    0\\
    -1\\
    -1
\end{array}\right]
+
\left[\begin{array}{ccc}
    3\\
    -1\\
    2\\
    2
\end{array}\right]
=
\left[\begin{array}{ccc}
    9\\
    -3\\
    4\\
    0
\end{array}\right]
\]\\

Da konstanterne foran $v$ i hvert led er 1, og $v_1, v_2, v_3 \in \mathcal{C}$, er koordinaterne for $x$ med henhold til $\mathcal{C}$:
\[
[x]_\mathcal{C} = 
\left[\begin{array}{ccc}
    1\\
    1\\
    1
\end{array}\right]
\]

Vi benytter vores basisskriftmatrice til at transformere vores koordinater til basen $\mathcal{B}$ fra $\mathcal{C}$:
\[
[x]_\mathcal{B} = 
\left[\begin{array}{ccc}
    3 & 0 & 1\\
    -4 & -1 & -1\\
    1 & -1 & 1\\
\end{array}\right]
\left[\begin{array}{ccc}
    1\\
    1\\
    1
\end{array}\right]
=
\left[\begin{array}{ccc}
    4\\
    -6\\
    1
\end{array}\right]
\]

\subsection{}

Vi ganger kolonne 2 i vores basisskriftmatrice på $u_1$ og $u_2$:
\[
-1 \cdot u_1 + (-1)\cdot u_2 =
-1 \cdot
\left[\begin{array}{ccc}
    0\\
    1\\
    -1\\
    1
\end{array}\right]
+
(-1) \cdot
\left[\begin{array}{ccc}
    1\\
    -1\\
    2\\
    2
\end{array}\right]
=
\left[\begin{array}{ccc}
    -1\\
    0\\
    -1\\
    -3
\end{array}\right]
\]\\

Vi får $v_2$, så $v_2$ må dermed række spannet af $u2, u3$.\\

Mangler at lave resten af opgaven

\subsection{}


\subsection{}

\end{document}