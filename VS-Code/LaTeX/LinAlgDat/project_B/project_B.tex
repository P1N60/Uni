\documentclass[a4paper,12pt]{article}
\usepackage{standalone}
\usepackage{amsmath}
\input{../../sty/setup.sty}

% Changes sections from 1.1 to 1.a
\renewcommand{\thesubsection}{\thesection.\alph{subsection}}
\graphicspath{ {../../pictures/IDMA/IDMA3_a}} 
\title{Københavns Universitet\\
LinAlgDat - Project B}
\author{Victor Vangkilde Jørgensen - kft410\\ 
kft410@alumni.ku.dk\\
Hold 13 Mach}

\begin{document}
\author{Victor Vangkilde Jørgensen - kft410\\ 
kft410@alumni.ku.dk}
\makeatletter
\let\getauthor\@author
\let\gettitle\@title
\makeatother
\maketitle
\thispagestyle{empty}
\n\n
 

\pagebreak
\pagestyle{empty}
\tableofcontents
\pagebreak
\pagestyle{fancy}
\fancyhf{}
\setlength{\headheight}{15.2pt}
\renewcommand{\footrulewidth}{0.4pt}
\fancyhead[R]{\nouppercase \lastrightmark}
\fancyfoot[L]{\gettitle}
\fancyfoot[R]{\thepage}
 
\maketitle 

\section[Opgave]{Opgave}
\subsection{}

Vi kan aflæse $M_a$ til:\\
\[
\left[\begin{array}{ccc}
    a & -1 & -1 \\
    0 & (a-1) & -1 \\
    0 & 2 & (a+2)
\end{array}\right]
\]


\subsection{}
$
\left[\begin{array}{ccc|c}
    a & -1 & -1 & 0 \\
    0 & a-1 & -1 & 0 \\
    0 & 2 & a+2 & 0
\end{array}\right]
\begin{array}{ccc}
    \\
    \cdot \frac{1}{a-1} \\
    \\
\end{array}
\leadsto
\left[\begin{array}{ccc|c}
    a & -1 & -1 & 0 \\
    0 & 1 & -\frac{1}{a-1} & 0 \\
    0 & 2 & a+2 & 0
\end{array}\right]
\begin{array}{ccc}
    \\
    \\
    -2r_2\\
\end{array}
\leadsto
$\\
$
\left[\begin{array}{ccc|c}
    a & -1 & -1 & 0 \\
    0 & 1 & -\frac{1}{a-1} & 0 \\
    0 & 0 & \frac{a^2+a}{a-1} & 0
\end{array}\right]
\begin{array}{ccc}
    \\
    \\
    \cdot \frac{a-1}{a^2+a}\\
\end{array}
\leadsto
\left[\begin{array}{ccc|c}
    a & -1 & -1 & 0 \\
    0 & 1 & -\frac{1}{a-1} & 0 \\
    0 & 0 & 1 & 0
\end{array}\right]
\begin{array}{ccc}
    +r_3\\
    +r_3 \cdot \frac{1}{a-1}\\
    \\
\end{array}
\leadsto
$\\
$
\left[\begin{array}{ccc|c}
    a & -1 & 0 & 0 \\
    0 & 1 & 0 & 0 \\
    0 & 0 & 1 & 0
\end{array}\right]
\begin{array}{ccc}
    +r_2\\
    \\
    \\
\end{array}
\leadsto
\left[\begin{array}{ccc|c}
    a & 0 & 0 & 0 \\
    0 & 1 & 0 & 0 \\
    0 & 0 & 1 & 0
\end{array}\right]
\begin{array}{ccc}
    \cdot \frac{1}{a}\\
    \\
    \\
\end{array}
\leadsto
$\\
$
\left[\begin{array}{ccc|c}
    1 & 0 & 0 & 0 \\
    0 & 1 & 0 & 0 \\
    0 & 0 & 1 & 0
\end{array}\right]
$\\

$T_a$ er altså injektiv.\\
$T_a$ er surjektiv, da vi har 3 vektoerer. $T_a$ er dermed bijektiv, da den både er injektiv og surjektiv.\\

Vi bestemmer nu $T^{-1}_a$:

$
\left[\begin{array}{ccc|ccc}
    a & -1 & -1 & 1 & 0 & 0 \\
    0 & a-1 & -1 & 0 & 1 & 0 \\
    0 & 2 & a+2 & 0 & 0 & 1
\end{array}\right]
\begin{array}{ccc}
    \\
    \cdot \frac{1}{a-1} \\
    \\
\end{array}
\leadsto
\left[\begin{array}{ccc|ccc}
    a & -1 & -1 & 1 & 0 & 0 \\
    0 & 1 & -\frac{1}{a-1} & 0 & \frac{1}{a-1} & 0 \\
    0 & 2 & a+2 & 0 & 0 & 1
\end{array}\right]
\begin{array}{ccc}
    \\
    \\
    -2r_2\\
\end{array}
\leadsto
$\\
$
\left[\begin{array}{ccc|ccc}
    a & -1 & -1 & 1 & 0 & 0 \\
    0 & 1 & -\frac{1}{a-1} & 0 & \frac{1}{a-1} & 0 \\
    0 & 0 & \frac{a^2+a}{a-1} & 0 & -\frac{2}{a-1} & 1
\end{array}\right]
\begin{array}{ccc}
    \\
    \\
    \cdot \frac{a-1}{a^2+a}\\
\end{array}
\leadsto
\left[\begin{array}{ccc|ccc}
    a & -1 & -1 & 1 & 0 & 0 \\
    0 & 1 & -\frac{1}{a-1} & 0 & \frac{1}{a-1} & 0 \\
    0 & 0 & 1 & 0 & -\frac{2}{a^2+a} & \frac{a-1}{a^2+a}
\end{array}\right]
\begin{array}{ccc}
    +r_3\\
    +r_3 \cdot \frac{1}{a-1}\\
    \\
\end{array}
\leadsto
$\\
$
\left[\begin{array}{ccc|ccc}
    a & -1 & 0 & 1 & -\frac{2}{a^2+a} & \frac{a-1}{a^2+a} \\
    0 & 1 & 0 & 0 & \frac{a+2}{a^2+a} & \frac{1}{a^2+a} \\
    0 & 0 & 1 & 0 & -\frac{2}{a^2+a} & \frac{a-1}{a^2+a}
\end{array}\right]
\begin{array}{ccc}
    +r_2\\
    \\
    \\
\end{array}
\leadsto
\left[\begin{array}{ccc|ccc}
    a & 0 & 0 & 1 & \frac{1}{a+1} & \frac{1}{a+1} \\
    0 & 1 & 0 & 0 & \frac{a+2}{a^2+a} & \frac{1}{a^2+a} \\
    0 & 0 & 1 & 0 & -\frac{2}{a^2+a} & \frac{a-1}{a^2+a}
\end{array}\right]
\begin{array}{ccc}
    \cdot \frac{1}{a}\\
    \\
    \\
\end{array}
\leadsto
$\\
$
\left[\begin{array}{ccc|ccc}
    1 & 0 & 0 & \frac{1}{a} & \frac{1}{a^2+a} & \frac{1}{a^2+a} \\
    0 & 1 & 0 & 0 & \frac{a+2}{a^2+a} & \frac{1}{a^2+a} \\
    0 & 0 & 1 & 0 & -\frac{2}{a^2+a} & \frac{a-1}{a^2+a}
\end{array}\right]
$\\

\subsection{}
Vi opstill igen $T_a$, hvor $a = -1$:\\

$
\left[\begin{array}{ccc|c}
    -1 & -1 & -1 & 0 \\
    0 & -1-1 & -1 & 0 \\
    0 & 2 & -1+2 & 0
\end{array}\right]
\begin{array}{ccc}
    \\
    reducer\\
    \\
\end{array}
\leadsto
\left[\begin{array}{ccc|c}
    -1 & -1 & -1 & 0 \\
    0 & -2 & -1 & 0 \\
    0 & 2 & 1 & 0
\end{array}\right]
\begin{array}{ccc}
    \\
    \\
    -r_2\\
\end{array}
\leadsto
$\\
$
\left[\begin{array}{ccc|c}
    -1 & -1 & -1 & 0 \\
    0 & -2 & -1 & 0 \\
    0 & 0 & 2 & 0
\end{array}\right]
\begin{array}{ccc}
    \\
    \\
    \cdot \frac{1}{2}\\
\end{array}
\leadsto
\left[\begin{array}{ccc|c}
    -1 & -1 & -1 & 0 \\
    0 & -2 & -1 & 0 \\
    0 & 0 & 1 & 0
\end{array}\right]
\begin{array}{ccc}
    +r_3\\
    +r_3\\
    \\
\end{array}
\leadsto
$\\
$
\left[\begin{array}{ccc|c}
    -1 & -1 & 0 & 0 \\
    0 & -2 & 0 & 0 \\
    0 & 0 & 1 & 0
\end{array}\right]
\begin{array}{ccc}
    \cdot (-1)\\
    \cdot (-\frac{1}{2})\\
    \\
\end{array}
\leadsto
\left[\begin{array}{ccc|c}
    1 & 1 & 0 & 0 \\
    0 & 1 & 0 & 0 \\
    0 & 0 & 1 & 0
\end{array}\right]
\begin{array}{ccc}
    -r_2\\
    \\
    \\
\end{array}
\leadsto
$\\
$
\left[\begin{array}{ccc|c}
    1 & 0 & 0 & 0 \\
    0 & 1 & 0 & 0 \\
    0 & 0 & 1 & 0
\end{array}\right]
$\\

\subsection{}


\subsection{}

\section[Opgave]{Opgave}
\subsection{}
Vi opstiller et ligningssystem i form af en totalmatrix, hvor vi sætter $u_1, u_2, u_3$ lig hhv. $v_1, v_2, v_3$, og finder løsningerne til disse, ved brug af Gauss-Jordan elimination.\\

$u_1 + u_2 + u_3 = v_1 \Leftrightarrow$\\

$
\left[\begin{array}{ccc|c}
    2 & 0 & 1 & 7\\
    1 & 1 & -1 & -2\\
    -1 & -1 & 2 & 3\\
    1 & 1 & 2 & 1
\end{array}\right]
\begin{array}{ccc}
    \\
    \cdot 2\\
    \cdot (-2)\\
    \cdot 2\\
\end{array}
\leadsto
\left[\begin{array}{ccc|c}
    2 & 0 & 1 & 7\\
    2 & 2 & -2 & -4\\
    2 & 2 & -4 & -6\\
    2 & 2 & 4 & 2
\end{array}\right]
\begin{array}{ccc}
    \\
    -r_1\\
    -r_1\\
    -r_1\\
\end{array}
\leadsto
$\\
$
\left[\begin{array}{ccc|c}
    2 & 0 & 1 & 7\\
    0 & 2 & -3 & -11\\
    0 & 2 & -5 & -13\\
    0 & 2 & 3 & -5
\end{array}\right]
\begin{array}{ccc}
    \\
    \\
    -r_2\\
    -r_2\\
\end{array}
\leadsto
\left[\begin{array}{ccc|c}
    2 & 0 & 1 & 7\\
    0 & 2 & -3 & -11\\
    0 & 0 & -2 & -2\\
    0 & 0 & 6 & 6
\end{array}\right]
\begin{array}{ccc}
    \\
    \\
    \\
    +3r_3\\
\end{array}
\leadsto
$\\
$
\left[\begin{array}{ccc|c}
    2 & 0 & 1 & 7\\
    0 & 2 & -3 & -11\\
    0 & 0 & -2 & -2\\
    0 & 0 & 0 & 0
\end{array}\right]
\begin{array}{ccc}
    \\
    \\
    \cdot (-\frac{1}{2})\\
    \\
\end{array}
\leadsto
\left[\begin{array}{ccc|c}
    2 & 0 & 1 & 7\\
    0 & 2 & -3 & -11\\
    0 & 0 & 1 & 1\\
    0 & 0 & 0 & 0
\end{array}\right]
\begin{array}{ccc}
    -1r_3\\
    +3r_3\\
    \\
    \\
\end{array}
\leadsto
$\\
$
\left[\begin{array}{ccc|c}
    2 & 0 & 0 & 6\\
    0 & 2 & 0 & -8\\
    0 & 0 & 1 & 1\\
    0 & 0 & 0 & 0
\end{array}\right]
\begin{array}{ccc}
    \cdot \frac{1}{2}\\
    \cdot \frac{1}{2}\\
    \\
    \\
\end{array}
\leadsto
\left[\begin{array}{ccc|c}
    1 & 0 & 0 & 3\\
    0 & 1 & 0 & -4\\
    0 & 0 & 1 & 1\\
    0 & 0 & 0 & 0
\end{array}\right]
$\\

Vores første kolonne i $P_{B\leftarrow C}$ er dermed: 
$\left[\begin{array}{ccc}
    3\\
    -4\\
    1\\
    0\\
\end{array}\right]$\\

$u_1 + u_2 + u_3 = v_2 \Leftrightarrow$\\

$
\left[\begin{array}{ccc|c}
    2 & 0 & 1 & -1\\
    1 & 1 & -1 & 0\\
    -1 & -1 & 2 & -1\\
    1 & 1 & 2 & -3
\end{array}\right]
\begin{array}{ccc}
    \\
    \cdot 2\\
    \cdot (-2)\\
    \cdot 2\\
\end{array}
\leadsto
\left[\begin{array}{ccc|c}
    2 & 0 & 1 & -1\\
    2 & 2 & -2 & 0\\
    2 & 2 & -4 & 2\\
    2 & 2 & 4 & -6
\end{array}\right]
\begin{array}{ccc}
    \\
    -r_1\\
    -r_1\\
    -r_1\\
\end{array}
\leadsto
$\\
$
\left[\begin{array}{ccc|c}
    2 & 0 & 1 & -1\\
    0 & 2 & -3 & 1\\
    0 & 2 & -5 & 3\\
    0 & 2 & 3 & -5
\end{array}\right]
\begin{array}{ccc}
    \\
    \\
    -r_2\\
    -r_2\\
\end{array}
\leadsto
\left[\begin{array}{ccc|c}
    2 & 0 & 1 & -1\\
    0 & 2 & -3 & 1\\
    0 & 0 & -2 & 2\\
    0 & 0 & 6 & -6
\end{array}\right]
\begin{array}{ccc}
    \\
    \\
    \\
    +3r_3\\
\end{array}
\leadsto
$\\
$
\left[\begin{array}{ccc|c}
    2 & 0 & 1 & -1\\
    0 & 2 & -3 & 1\\
    0 & 0 & -2 & 2\\
    0 & 0 & 0 & 0
\end{array}\right]
\begin{array}{ccc}
    \\
    \\
    \cdot (-\frac{1}{2})\\
    \\
\end{array}
\leadsto
\left[\begin{array}{ccc|c}
    2 & 0 & 1 & -1\\
    0 & 2 & -3 & 1\\
    0 & 0 & 1 & -1\\
    0 & 0 & 0 & 0
\end{array}\right]
\begin{array}{ccc}
    -1r_3\\
    +3r_3\\
    \\
    \\
\end{array}
\leadsto
$\\
$
\left[\begin{array}{ccc|c}
    2 & 0 & 0 & 0\\
    0 & 2 & 0 & -2\\
    0 & 0 & 1 & -1\\
    0 & 0 & 0 & 0
\end{array}\right]
\begin{array}{ccc}
    \cdot \frac{1}{2}\\
    \cdot \frac{1}{2}\\
    \\
    \\
\end{array}
\leadsto
\left[\begin{array}{ccc|c}
    1 & 0 & 0 & 0\\
    0 & 1 & 0 & -1\\
    0 & 0 & 1 & -1\\
    0 & 0 & 0 & 0
\end{array}\right]
$\\

Vores anden kolonne i $P_{B\leftarrow C}$ er dermed: 
$\left[\begin{array}{ccc}
    0\\
    -1\\
    -1\\
    0\\
\end{array}\right]$\\

$u_1 + u_2 + u_3 = v_3 \Leftrightarrow$\\

$
\left[\begin{array}{ccc|c}
    2 & 0 & 1 & 3\\
    1 & 1 & -1 & -1\\
    -1 & -1 & 2 & 2\\
    1 & 1 & 2 & 2
\end{array}\right]
\begin{array}{ccc}
    \\
    \cdot 2\\
    \cdot (-2)\\
    \cdot 2\\
\end{array}
\leadsto
\left[\begin{array}{ccc|c}
    2 & 0 & 1 & 3\\
    2 & 2 & -2 & -2\\
    2 & 2 & -4 & -4\\
    2 & 2 & 4 & 4
\end{array}\right]
\begin{array}{ccc}
    \\
    -r_1\\
    -r_1\\
    -r_1\\
\end{array}
\leadsto
$\\
$
\left[\begin{array}{ccc|c}
    2 & 0 & 1 & 3\\
    0 & 2 & -3 & -5\\
    0 & 2 & -5 & -7\\
    0 & 2 & 3 & 1
\end{array}\right]
\begin{array}{ccc}
    \\
    \\
    -r_2\\
    -r_2\\
\end{array}
\leadsto
\left[\begin{array}{ccc|c}
    2 & 0 & 1 & 3\\
    0 & 2 & -3 & -5\\
    0 & 0 & -2 & -2\\
    0 & 0 & 6 & 6
\end{array}\right]
\begin{array}{ccc}
    \\
    \\
    \\
    +3r_3\\
\end{array}
\leadsto
$\\
$
\left[\begin{array}{ccc|c}
    2 & 0 & 1 & 3\\
    0 & 2 & -3 & -5\\
    0 & 0 & -2 & -2\\
    0 & 0 & 0 & 0
\end{array}\right]
\begin{array}{ccc}
    \\
    \\
    \cdot (-\frac{1}{2})\\
    \\
\end{array}
\leadsto
\left[\begin{array}{ccc|c}
    2 & 0 & 1 & 3\\
    0 & 2 & -3 & -5\\
    0 & 0 & 1 & 1\\
    0 & 0 & 0 & 0
\end{array}\right]
\begin{array}{ccc}
    -1r_3\\
    +3r_3\\
    \\
    \\
\end{array}
\leadsto
$\\
$
\left[\begin{array}{ccc|c}
    2 & 0 & 0 & 2\\
    0 & 2 & 0 & -2\\
    0 & 0 & 1 & 1\\
    0 & 0 & 0 & 0
\end{array}\right]
\begin{array}{ccc}
    \cdot \frac{1}{2}\\
    \cdot \frac{1}{2}\\
    \\
    \\
\end{array}
\leadsto
\left[\begin{array}{ccc|c}
    1 & 0 & 0 & 1\\
    0 & 1 & 0 & -1\\
    0 & 0 & 1 & 1\\
    0 & 0 & 0 & 0
\end{array}\right]
$\\

Vores sidste kolonne i $P_{B\leftarrow C}$ er dermed: 
$\left[\begin{array}{ccc}
    1\\
    -1\\
    1\\
    0\\
\end{array}\right]$\\

Sammensætter vi nu vores tre kolonner til en matrix, får vi:\\
\[
P_{B\leftarrow C} = 
\left[\begin{array}{ccc}
    3 & 0 & 1\\
    -4 & -1 & -1\\
    1 & -1 & 1\\
\end{array}\right]
\]

\subsection{}
\[
x = 
\left[\begin{array}{ccc}
    7\\
    -2\\
    3\\
    1
\end{array}\right]
+
\left[\begin{array}{ccc}
    -1\\
    0\\
    -1\\
    -1
\end{array}\right]
+
\left[\begin{array}{ccc}
    3\\
    -1\\
    2\\
    2
\end{array}\right]
=
\left[\begin{array}{ccc}
    9\\
    -3\\
    4\\
    0
\end{array}\right]
\]\\

Da konstanterne foran $v$ i hvert led er 1, og $v_1, v_2, v_3 \in \mathcal{C}$, er koordinaterne for $x$ med henhold til $\mathcal{C}$:
\[
[x]_\mathcal{C} = 
\left[\begin{array}{ccc}
    1\\
    1\\
    1
\end{array}\right]
\]

Vi benytter vores basisskriftmatrice til at transformere vores koordinater til basen $\mathcal{B}$ fra $\mathcal{C}$:
\[
[x]_\mathcal{B} = 
\left[\begin{array}{ccc}
    3 & 0 & 1\\
    -4 & -1 & -1\\
    1 & -1 & 1\\
\end{array}\right]
\left[\begin{array}{ccc}
    1\\
    1\\
    1
\end{array}\right]
=
\left[\begin{array}{ccc}
    4\\
    -6\\
    1
\end{array}\right]
\]

\subsection{}

Vi ganger kolonne 2 i vores basisskriftmatrice på $u_1$ og $u_2$:
\[
-1 \cdot u_1 + (-1)\cdot u_2 =
-1 \cdot
\left[\begin{array}{ccc}
    0\\
    1\\
    -1\\
    1
\end{array}\right]
+
(-1) \cdot
\left[\begin{array}{ccc}
    1\\
    -1\\
    2\\
    2
\end{array}\right]
=
\left[\begin{array}{ccc}
    -1\\
    0\\
    -1\\
    -3
\end{array}\right]
\]\\

Vi får $v_2$, så $v_2$ må dermed række spannet af $u2, u3$.\\

Mangler at lave resten af opgaven

\subsection{}


\subsection{}

\end{document}