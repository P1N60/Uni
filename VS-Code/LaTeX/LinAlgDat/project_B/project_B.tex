\documentclass[a4paper,12pt]{article}
\usepackage{standalone}
\usepackage{amsmath}



% Basic
\usepackage[T1]{fontenc}
\usepackage[utf8]{inputenc}
\usepackage{titlesec}
\titleformat{\section}
  {\normalfont\fontsize{14}{15}\bfseries}{\thesection}{1em}{}
\titleformat{\subsection}
  {\normalfont\fontsize{12}{15}\bfseries}{\thesubsection}{1em}{}

% Changes sections from 1.1 to 1.a
\renewcommand{\thesubsection}{\thesection.\alph{subsection}}

% ------------------------------------------------------------ %

% Packages
\RequirePackage{tcolorbox}
\usepackage{amsmath, amsthm, amssymb}
\usepackage{blindtext}
\usepackage{enumitem}
\usepackage{extramarks}
\usepackage{fancyhdr}
\usepackage[margin=1in]{geometry}
\usepackage{graphicx}
\usepackage{hyperref}
\usepackage{indentfirst}
\usepackage{listings}
\usepackage{mathrsfs}
\usepackage{mdframed}
\usepackage{multicol, multirow}
\usepackage{needspace, setspace}
\usepackage{paracol}
\usepackage{pgf, pgfplots}
\usepackage{tikz}
\usetikzlibrary{patterns}
\usepackage{silence}
\usepackage{xcolor}
\usepackage{bookmark}
\setlength{\parindent}{0pt}
\usetikzlibrary{patterns}
% \usepackage{subcaption}
\usetikzlibrary{decorations.pathreplacing}
\usepackage{caption}
\usepackage{subcaption}
\usepackage{xcolor}
\definecolor{maroon}{RGB}{128, 0, 0}
% ------------------------------------------------------------ %

% Spacings
\newcommand{\n}{\vspace{3mm}} % Context spacing
\newcommand{\s}{\vspace{1mm}} % Equation spacing
\newcommand{\m}{\vspace{-3mm}} % Reverse context spacing
\newcommand{\propdisp}{\pagebreak} % Proper display page break
\newcommand{\br}{\n\\\n}

% Wordings
\newcommand{\ans}[1][zb]{{\color{#1}\textit{Answer. }\hspace{3mm}}} % Answer
\newcommand{\arl}[1][zr]{{\color{#1}$\brr{\Leftarrow}$\hspace{3mm}}} % Left arrow
\newcommand{\arr}[1][zr]{{\color{#1}$\brr{\Rightarrow}$\hspace{3mm}}} % Right arrow
\newcommand{\cse}[2][zr]{{\color{#1}\textit{Case #2: }\hspace{1mm}}} % Case
\newcommand{\clm}[2][zr]{{\color{#1}$\vdash_{#2}$\hspace{1mm}}} % Claim
\newcommand{\prf}[1][zr]{{\color{#1}\textit{Proof. }\hspace{3mm}}} % Proof
\newcommand{\prt}[2][a]{\hspace{-2mm}{\color{#2}\textit{Part (#1) }}\hspace{1mm}} % Part
\newcommand{\prtc}[2][a]{\hspace{2mm}\prt[#1]{#2}} % Continued part

\newcommand{\rdft}[1][\sct]{{\color{zg}\textit{Definition #1}}} % Refer definition
\newcommand{\rexm}[1][\sct]{{\color{zb}\textit{Example #1}}} % Refer example
\newcommand{\rfig}[1][\sct]{{\color{zy}\textit{Figure #1}}} % Refer figure
\newcommand{\rpst}[1][\sct]{{\color{zr}\textit{Proposition #1}}} % Refer proposition
\newcommand{\rthm}[1][\sct]{{\color{zr}\textit{Theorem #1}}} % Refer theorem

\newcommand{\sct}{\thesection.\thescount} % Counter
\newcommand{\sctr}[2][0]{\the\numexpr\value{section}-#1\relax.\the\numexpr\value{scount}-#2\relax} % Relative counter

% Equations
\newcommand{\C}{\mathbb{C}} % Complex
\newcommand{\F}{\mathbb{F}} % Field
\newcommand{\I}{\mathbb{I}} % Irrational
\newcommand{\N}{\mathbb{N}} % Natural
\newcommand{\Q}{\mathbb{Q}} % Rational
\newcommand{\R}{\mathbb{R}} % Real
\newcommand{\Z}{\mathbb{Z}} % Integer

\newcommand{\GL}{\mathrm{GL}} % General linear group
\newcommand{\SL}{\mathrm{SL}} % Special linear group

\newcommand{\abs}[1]{\left| #1\right|} % Absolute
\newcommand{\bra}[1]{\left\langle #1\right\rangle} % Angled brackets
\newcommand{\brc}[1]{\left\{ #1\right\}} % Curly brackets
\newcommand{\brr}[1]{\left( #1\right)} % Round brackets
\newcommand{\brs}[1]{\left[ #1\right]} % Square brackets
\newcommand{\cond}[1]{\left. #1\right|} % Condition with right bar
\newcommand{\diff}{\,\mathrm{d}} % Differential
\newcommand{\erm}[1]{\;\;\;\;\text{#1}} % Equation remarks
\newcommand{\nrm}[1]{\left\| #1\right\|} % Norm
\newcommand{\srm}{\,\mid\,} % Set remarks

% Math operators
\let\Im\relax
\let\Re\relax

\DeclareMathOperator{\Im}{Im} % Imaginary function
\DeclareMathOperator{\Re}{Re} % Real function

% ------------------------------------------------------------ %

% Colours
\definecolor{gr}{RGB}{120, 120, 120} % Grey
\definecolor{zb}{RGB}{0, 38, 77} % Blue
\definecolor{zg}{RGB}{0, 77, 51} % Green
\definecolor{zp}{RGB}{51, 0, 77} % Purple
\definecolor{zr}{RGB}{77, 0, 38} % Red
\definecolor{zy}{RGB}{77, 64, 0} % Yellow

% Graphing
\usetikzlibrary{arrows}
\usetikzlibrary{calc}
\usetikzlibrary{patterns}
\pgfplotsset{compat=1.15}

% ------------------------------------------------------------ %

% Remark
\newcommand{\remark}[1]{
  \noindent\textbf{Remarks}
  
  \begin{nlist}
    \item Context of this document is based on university course \textit{\gettitle} from \textit{Department of Mathematics, The Chinese University of Hong Kong}. The original source can be found at \url{https://www.math.cuhk.edu.hk/course}. The author does not own the source.
    \item This document is assumed unavailable for unauthorized parties that have not attended the university course. It is prohibited to share, including distributing or copying this document to unauthorized parties in any means for any non-academic purpose.
    \item Context of this doucment may not be completely accurate. The author assumes no responsibility or liability for any errors or omissions in the context of this document.
    \item This document is under license CC-BY-SA 4.0. It is allowed to make any editions on this document, as long as terms of the license is not violated.
    #1
  \end{nlist}
}

% Prerequisites
\newenvironment{prereq}{
  \noindent \textbf{Prerequisites}\n

  This course requires prerequisites of
}{
  \n
}

% Reference list
\newenvironment{reflist}{
  \begin{alist}
    \item Course material from various professors associated to \textit{\gettitle}
}{
  \end{alist}
}

% ------------------------------------------------------------ %

% Environments
\newcounter{scount}[section] % Counter

\newenvironment{crl}{ % Corollary
  \parindent 0pt
  \begin{siderule}[linecolor=zr]{\color{zr}\textit{Corollary. }}
}{
  \end{siderule}
}

\newenvironment{cmt}{ % Comment
  \parindent 0pt
  \begin{siderule}[linecolor=zp]{\color{zp}\textit{Comment. }}
}{
  \end{siderule}
}

\newenvironment{dft}{ % Definition
  \parindent 0pt
  \refstepcounter{scount}
  \begin{siderule}[linecolor=zg]{\color{zg}\textit{Definition \sct. }}
}{
  \end{siderule}
}
\newenvironment{lem}{ % Lemma
  \parindent 0pt
  \refstepcounter{scount}
  \begin{siderule}[linecolor=zg]{\color{zg}\textit{Lemma \sct. }}
}{
  \end{siderule}
}

\newenvironment{exm}{ % Example
  \parindent 0pt
  \refstepcounter{scount}
  \begin{siderule}[linecolor=zb]{\color{zb}\textit{Example \sct. }}
}{
  \end{siderule}
}

\newenvironment{fig}{ % Figure
  \parindent 0pt
  \refstepcounter{scount}
  \begin{siderule}[linecolor=zy]{\color{zy}\textit{Figure \sct. }}\n
  
}{
  \end{siderule}
}

\newenvironment{prv}{ % Proof
  \parindent 0pt
  \begin{siderule}[linecolor=zr]\prf
}{
  \end{siderule}
}

\newenvironment{pst}{ % Proposition
  \parindent 0pt
  \refstepcounter{scount}
  \begin{siderule}[linecolor=zr]{\color{zr}\textit{Proposition \sct. }}
}{
  \end{siderule}
}

\newenvironment{tcn}{ % Technique
  \parindent 0pt
  \begin{siderule}[linecolor=zp]{\color{zp}\textit{Technique. }}
}{
  \end{siderule}
}

\newenvironment{thm}{ % Theorem
  \parindent 0pt
  \refstepcounter{scount}
  \begin{siderule}[linecolor=zr]{\color{zr}\textit{Sætning \sct. }}
}{
  \end{siderule}
}

\newenvironment{rmatrix}{ % Matrix in round brackets
  \left(\begin{matrix}
}{
  \end{matrix}\right)
}

\newenvironment{alist}{ % Alphabetical list
  \begin{enumerate}[label=(\alph*)]
}{
  \end{enumerate}
}

\newenvironment{Alist}{ % Capitalized alphabetical list
  \begin{enumerate}[label=(\Alph*)]
}{
  \end{enumerate}
}

\newenvironment{nlist}{ % Number list
  \begin{enumerate}[label=(\arabic*)]
}{
  \end{enumerate}
}

\newenvironment{plist}{ % Point list
  \begin{itemize}
}{
  \end{itemize}
}

\newenvironment{rlist}{ % Roman list
  \begin{enumerate}[label=(\roman*)]
}{
  \end{enumerate}
}

\newmdenv[ % Siderule line
  topline=false,
  bottomline=false,
  rightline=false,
  rightmargin=0
]{siderule}


% ------------------------------------------------------------ %
% Warning filters
\WarningFilter{mdframed}{You got a bad break}
\hfuzz=8pt


% Load required packages
\usepackage{tcolorbox}
% ------------------------------------------------------------ %

% Code

\usepackage{amsmath}
\usepackage{graphicx}
\definecolor{bluekeywords}{rgb}{0.13,0.13,1}
\definecolor{greencomments}{rgb}{0,0.5,0}
\definecolor{redstrings}{rgb}{0.9,0,0}
\definecolor{bgcolor}{rgb}{0.95,0.94,0.94}
\usepackage{listings}
\usepackage{upquote}
\usepackage{xcolor}

\lstdefinelanguage{Python}
{
  keywords={typeof, null, catch, switch, in, int, str, float, self},
  keywordstyle=\color{ForestGreen}\bfseries,
  ndkeywords={boolean, throw, import},
  ndkeywords={return, class, if ,elif, endif, while, do, else, True, False , catch, def},
  ndkeywordstyle=\color{BrickRed}\bfseries,
  identifierstyle=\color{black},
  sensitive=false,
  comment=[l]{\#},
  morecomment=[s]{/*}{*/},
  commentstyle=\color{purple}\ttfamily,
  stringstyle=\color{red}\ttfamily,
}

\lstset
{ %Formatting for code in appendix
    language=Python,
    numbers=left,
    stepnumber=1,
    showstringspaces=false,
    tabsize=1,
    breaklines=true,
    breakatwhitespace=false,
    backgroundcolor=\color{bgcolor},  % Background color
    basicstyle=\ttfamily\footnotesize, % Code font size and style
    frame=single,                    % Adds a frame around the code
    rulecolor=\color{bgcolor},       % Frame color
    breaklines=true,                 % Breaks long lines
}


% Changes sections from 1.1 to 1.a
\renewcommand{\thesubsection}{\thesection.\alph{subsection}}
\graphicspath{ {../../pictures/IDMA/IDMA3_a}} 
\title{Københavns Universitet\\
LinAlgDat - Project B}
\author{Victor Vangkilde Jørgensen - kft410\\ 
kft410@alumni.ku.dk\\
Hold 13 Mach}

\begin{document}
\author{Victor Vangkilde Jørgensen}
\makeatletter
\let\getauthor\@author
\let\gettitle\@title
\makeatother
\maketitle
\thispagestyle{empty}
\n\n
 

\pagebreak
\pagestyle{empty}
\tableofcontents
\pagebreak
\pagestyle{fancy}
\fancyhf{}
\setlength{\headheight}{15.2pt}
\renewcommand{\footrulewidth}{0.4pt}
% \fancyhead[R]{\nouppercase \lastrightmark}
\fancyfoot[L]{\gettitle}
\fancyfoot[R]{\thepage}
 
\maketitle 

\section[Opgave]{Opgave}
\subsection{}

Vi forkaster $x_1, x_2, x_3$, og bruger deres konstanter til at aflæse $M_a$ til:\\
\[
\left[\begin{array}{ccc}
    a & -1 & -1 \\
    0 & a-1 & -1 \\
    0 & 2 & a+2
\end{array}\right]
\]
$x_1, x_2, x_3$ droppes, da disse kun indgår, når vi ganger $M_a$ med $\left[\begin{array}{ccc|c}
    x_1 \\
    x_2 \\
    x_3
\end{array}\right]$.



\subsection{}
Vi ser først på, om $T_a$ er injektiv. En transformation er injektiv, hvis kernen af transformationen kun er \{0\}. Det vil sige, at kun nulvektoren transformeret giver nulvektoren.\\

$
\left[\begin{array}{ccc|c}
    a & -1 & -1 & 0 \\
    0 & a-1 & -1 & 0 \\
    0 & 2 & a+2 & 0
\end{array}\right]
\begin{array}{ccc}
    \\
    \cdot \frac{1}{a-1} \\
    \\
\end{array}
\leadsto
\left[\begin{array}{ccc|c}
    a & -1 & -1 & 0 \\
    0 & 1 & -\frac{1}{a-1} & 0 \\
    0 & 2 & a+2 & 0
\end{array}\right]
\begin{array}{ccc}
    \\
    \\
    -2r_2\\
\end{array}
\leadsto
$\\
$
\left[\begin{array}{ccc|c}
    a & -1 & -1 & 0 \\
    0 & 1 & -\frac{1}{a-1} & 0 \\
    0 & 0 & \frac{a^2+a}{a-1} & 0
\end{array}\right]
\begin{array}{ccc}
    \\
    \\
    \cdot \frac{a-1}{a^2+a}\\
\end{array}
\leadsto
\left[\begin{array}{ccc|c}
    a & -1 & -1 & 0 \\
    0 & 1 & -\frac{1}{a-1} & 0 \\
    0 & 0 & 1 & 0
\end{array}\right]
\begin{array}{ccc}
    +r_3\\
    +r_3 \cdot \frac{1}{a-1}\\
    \\
\end{array}
\leadsto
$\\
$
\left[\begin{array}{ccc|c}
    a & -1 & 0 & 0 \\
    0 & 1 & 0 & 0 \\
    0 & 0 & 1 & 0
\end{array}\right]
\begin{array}{ccc}
    +r_2\\
    \\
    \\
\end{array}
\leadsto
\left[\begin{array}{ccc|c}
    a & 0 & 0 & 0 \\
    0 & 1 & 0 & 0 \\
    0 & 0 & 1 & 0
\end{array}\right]
\begin{array}{ccc}
    \cdot \frac{1}{a}\\
    \\
    \\
\end{array}
\leadsto
$\\
$
\left[\begin{array}{ccc|c}
    1 & 0 & 0 & 0 \\
    0 & 1 & 0 & 0 \\
    0 & 0 & 1 & 0
\end{array}\right]
$\\

Da ker $T_a = \{0\}$, er $T_a$ injektiv.\\

$T_a$ er surjektiv, hvis dimensionen af det udspændte rum er det samme som dimensionen af transformationsmatricen.

\[
dim(ran (T_a)) = rank (T_a) = 3
\]\\
da vi har 3 pivotelementer.\\

Da søjlerne i $T_a$ udspænder hele $\R^3$, er $T_a$ surjektiv. $T_a$ er dermed bijektiv, da den både er injektiv og surjektiv.\\

Vi bestemmer nu $T^{-1}_a$, ved at sætte enhedsmatricen på til højre, og reducere med Gauss-Jordan:\\

$
\left[\begin{array}{ccc|ccc}
    a & -1 & -1 & 1 & 0 & 0 \\
    0 & a-1 & -1 & 0 & 1 & 0 \\
    0 & 2 & a+2 & 0 & 0 & 1
\end{array}\right]
\begin{array}{ccc}
    \\
    \cdot \frac{1}{a-1} \\
    \\
\end{array}
\leadsto
\left[\begin{array}{ccc|ccc}
    a & -1 & -1 & 1 & 0 & 0 \\
    0 & 1 & -\frac{1}{a-1} & 0 & \frac{1}{a-1} & 0 \\
    0 & 2 & a+2 & 0 & 0 & 1
\end{array}\right]
\begin{array}{ccc}
    \\
    \\
    -2r_2\\
\end{array}
\leadsto
$\\
$
\left[\begin{array}{ccc|ccc}
    a & -1 & -1 & 1 & 0 & 0 \\
    0 & 1 & -\frac{1}{a-1} & 0 & \frac{1}{a-1} & 0 \\
    0 & 0 & \frac{a^2+a}{a-1} & 0 & -\frac{2}{a-1} & 1
\end{array}\right]
\begin{array}{ccc}
    \\
    \\
    \cdot \frac{a-1}{a^2+a}\\
\end{array}
\leadsto
\left[\begin{array}{ccc|ccc}
    a & -1 & -1 & 1 & 0 & 0 \\
    0 & 1 & -\frac{1}{a-1} & 0 & \frac{1}{a-1} & 0 \\
    0 & 0 & 1 & 0 & -\frac{2}{a^2+a} & \frac{a-1}{a^2+a}
\end{array}\right]
\begin{array}{ccc}
    +r_3\\
    +r_3 \cdot \frac{1}{a-1}\\
    \\
\end{array}
\leadsto
$\\
$
\left[\begin{array}{ccc|ccc}
    a & -1 & 0 & 1 & -\frac{2}{a^2+a} & \frac{a-1}{a^2+a} \\
    0 & 1 & 0 & 0 & \frac{a+2}{a^2+a} & \frac{1}{a^2+a} \\
    0 & 0 & 1 & 0 & -\frac{2}{a^2+a} & \frac{a-1}{a^2+a}
\end{array}\right]
\begin{array}{ccc}
    +r_2\\
    \\
    \\
\end{array}
\leadsto
\left[\begin{array}{ccc|ccc}
    a & 0 & 0 & 1 & \frac{1}{a+1} & \frac{1}{a+1} \\
    0 & 1 & 0 & 0 & \frac{a+2}{a^2+a} & \frac{1}{a^2+a} \\
    0 & 0 & 1 & 0 & -\frac{2}{a^2+a} & \frac{a-1}{a^2+a}
\end{array}\right]
\begin{array}{ccc}
    \cdot \frac{1}{a}\\
    \\
    \\
\end{array}
\leadsto
$\\
$
\left[\begin{array}{ccc|ccc}
    1 & 0 & 0 & \frac{1}{a} & \frac{1}{a^2+a} & \frac{1}{a^2+a} \\
    0 & 1 & 0 & 0 & \frac{a+2}{a^2+a} & \frac{1}{a^2+a} \\
    0 & 0 & 1 & 0 & -\frac{2}{a^2+a} & \frac{a-1}{a^2+a}
\end{array}\right]
$\\

\[
T^{-1}_a=
\left[\begin{array}{ccc}
    \frac{1}{a} & \frac{1}{a^2+a} & \frac{1}{a^2+a} \\
    0 & \frac{a+2}{a^2+a} & \frac{1}{a^2+a} \\
    0 & -\frac{2}{a^2+a} & \frac{a-1}{a^2+a}
\end{array}\right]
\]

\subsection{}
Vi opstill igen $T_a$, hvor $a = -1$:\\

$
\left[\begin{array}{ccc|c}
    -1 & -1 & -1 & 0 \\
    0 & -1-1 & -1 & 0 \\
    0 & 2 & -1+2 & 0
\end{array}\right]
\begin{array}{ccc}
    \\
    reducer\\
    \\
\end{array}
\leadsto
\left[\begin{array}{ccc|c}
    -1 & -1 & -1 & 0 \\
    0 & -2 & -1 & 0 \\
    0 & 2 & 1 & 0
\end{array}\right]
\begin{array}{ccc}
    \\
    \\
    +r_2\\
\end{array}
\leadsto
$\\
$
\left[\begin{array}{ccc|c}
    -1 & -1 & -1 & 0 \\
    0 & -2 & -1 & 0 \\
    0 & 0 & 0 & 0
\end{array}\right]
\begin{array}{ccc}
    \cdot (-1)\\
    \cdot (-1)\\
    \\
\end{array}
\leadsto
\left[\begin{array}{ccc|c}
    1 & 1 & 1 & 0 \\
    0 & 2 & 1 & 0 \\
    0 & 0 & 0 & 0
\end{array}\right]
\begin{array}{ccc}
    \\
    \cdot \frac{1}{2}\\
    \\
\end{array}
\leadsto
$\\
$
\left[\begin{array}{ccc|c}
    1 & 1 & 1 & 0 \\
    0 & 1 & \frac{1}{2} & 0 \\
    0 & 0 & 0 & 0
\end{array}\right]
\begin{array}{ccc}
    -r_2\\
    \\
    \\
\end{array}
\leadsto
\left[\begin{array}{ccc|c}
    1 & 0 & \frac{1}{2} & 0 \\
    0 & 1 & \frac{1}{2} & 0 \\
    0 & 0 & 0 & 0
\end{array}\right]
$\\

Vi aflæser løsningerne til:\\
\[
\left[\begin{array}{ccc}
    x_1\\
    x_2\\
    x_3\\
\end{array}\right]
=
t
\left[\begin{array}{ccc}
    -\frac{1}{2}\\
    -\frac{1}{2}\\
    1\\
\end{array}\right]
\]

Kigger vi nu på baserne for spannet af $T_{-1}$, ser vi, at vi kun skal benytte $x_1$ og $x_2$, da $x_3$ er en fri variable. 

\[
span(T_{-1}) = \left\{ 
\left[\begin{array}{ccc|c}
    -1 \\
    0   \\
    0  
\end{array}\right]\
,
\left[\begin{array}{ccc|c}
    -1 \\
    -2   \\
    2  
\end{array}\right]
\right\}
\Rightarrow
dim(ran(T_{-1})) = 2
\]
hvilket igen giver mening, da vi har 2 pivotelementer.\\

Da vi valgte at løse efter nulrummet for $T_{-1}$, har vi nu de løsninger, som udspænder netop dette:\\

\[
ker(T_{-1}) = 
\left[\begin{array}{ccc}
    -\frac{1}{2}\\
    -\frac{1}{2}\\
    1\\
\end{array}\right]
\Rightarrow
dim(ker(T_{-1})) = 1
\]

Så dimensionen af $ran(T_{-1}) = 2$, og dimensionen af $ker(T_{-1}) = 1$\\

Vi gør det samme for $a= 0$:\\

$
\left[\begin{array}{ccc|c}
    0 & -1 & -1 & 0\\
    0 & (0-1) & -1 & 0\\
    0 & 2 & (0+2) & 0
\end{array}\right]
\begin{array}{ccc}
    \\
    reducer\\
    \\
\end{array}
\leadsto
\left[\begin{array}{ccc|c}
    0 & -1 & -1 & 0 \\
    0 & -1 & -1 & 0 \\
    0 & 2 & 2 & 0
\end{array}\right]
\begin{array}{ccc}
    \\
    -r_1\\
    +2r_2\\
\end{array}
\leadsto
$\\
$
\left[\begin{array}{ccc|c}
    0 & -1 & -1 & 0 \\
    0 & 0 & 0 & 0 \\
    0 & 0 & 0 & 0
\end{array}\right]
\begin{array}{ccc}
    \cdot (-1)\\
    \\
    \\
\end{array}
\leadsto
\left[\begin{array}{ccc|c}
    0 & 1 & 1 & 0 \\
    0 & 0 & 0 & 0 \\
    0 & 0 & 0 & 0
\end{array}\right]
$\\

Her kan løsningerne aflæses til:\\

\[
\left[\begin{array}{ccc}
    x_1\\
    x_2\\
    x_3\\
\end{array}\right]
=
t
\left[\begin{array}{ccc}
    1\\
    0\\
    0\\
\end{array}\right]
+
s
\left[\begin{array}{ccc}
    0\\
    -1\\
    1\\
\end{array}\right]
\]

Vi har 1 pivotelement, som vi finder i 2. søjle.\\

\[
\left[\begin{array}{ccc|c}
    -1 \\
    -1 \\
    2 
\end{array}\right]
= ran(T_{0}) \Rightarrow dim(ran(T_{0})) = 1
\]

og for kernen:\\

\[
\left\{
\left[\begin{array}{ccc|c}
    0 \\
    0 \\
    0
\end{array}\right]
,
\left[\begin{array}{ccc|c}
    1 \\
    1 \\
    -2
\end{array}\right]
\right\}
= ker(T_{0}) \Rightarrow dim(ker(T_{0})) = 2
\]

\subsection{}
Ganger vi $M_{-1}$ med sig selv, får vi:\\
\[
M_{-1}^2 = M_{-1}\cdot M_{-1}=
\left[\begin{array}{ccc}
    -1 & -1 & -1 \\
    0 & -2 & -1 \\
    0 & 2 & 1
\end{array}\right]
\cdot
\left[\begin{array}{ccc}
    -1 & -1 & -1 \\
    0 & -2 & -1 \\
    0 & 2 & 1
\end{array}\right]
=
\left[\begin{array}{ccc}
    1 & 1 & 1 \\
    0 & 2 & 1 \\
    0 & -2 & -1
\end{array}\right]
\]

og gør vi det endnu en gang, får vi:\\

\[
M_{-1}^3 = M_{-1}^2\cdot M_{-1}=
\left[\begin{array}{ccc}
    1 & 1 & 1 \\
    0 & 2 & 1 \\
    0 & -2 & -1
\end{array}\right]
\cdot
\left[\begin{array}{ccc}
    -1 & -1 & -1 \\
    0 & -2 & -1 \\
    0 & 2 & 1
\end{array}\right]
=
\left[\begin{array}{ccc}
    -1 & -1 & -1 \\
    0 & -2 & -1 \\
    0 & 2 & 1
\end{array}\right]
\]
hvilket bringer os tilbage hvor vi startede.\\

Hvis vi gentog denne process, ville vi blot skifte fortegn hver gang vi ganger $M_{-1}$ på.\\
Vi kan dermed sige:\\

\[
M_{-1}^n = \left\{
\begin{array}{cc}
    M_{-1} & hvis \ n \ er \ lige
    \\
    M_{-1}^2 & hvis \ n \ er \ ulige
\end{array}
\ \forall n\in \N \\
\right.
\]

Ganger vi $M_0$ med sig selv, får vi:\\
\[
M_{0}^2 = M_{0}\cdot M_{0}=
\left[\begin{array}{ccc}
    0 & -1 & -1 \\
    0 & -1 & -1 \\
    0 & 2 & 2
\end{array}\right]
\cdot
\left[\begin{array}{ccc}
    0 & -1 & -1 \\
    0 & -1 & -1 \\
    0 & 2 & 2
\end{array}\right]
=
\left[\begin{array}{ccc}
    0 & -1 & -1 \\
    0 & -1 & -1 \\
    0 & 2 & 2
\end{array}\right]
\]

Vi ser, at hvis vi ganger denne matrice med sig selv, får vi blot den samme matrice igen.\\
Da matricen er uændret, kan vi beskrive dette som:
\[
M_{0}^n = M_0 \ \forall n \in \N
\]

\subsection{}

 \(T_a(\mathcal{L})\subseteq \mathcal{L}\):
Anvend \(T_a\) på \(\mathcal{L}\):
  \[
    T_a\bigl(\mathcal{L})
    = \left[\begin{array}{ccc}
        a\,t & - t & -(-2t)\\[6pt]
        0& (a-1)\,t & -(-2t)\\[6pt]
        0 & 2t + (a+2)(-2t)
      \end{array}\right]
    = \left[\begin{array}{ccc}
        (a+1)t\\[3pt]
        (a+1)t\\[3pt]
        -2(a+1)t
      \end{array}\right]
    = (a+1)\,t\,(1,1,-2).
  \]
Da \((a+1)\,t\,(1,1,-2)\in L\) for alle \(t\), følger
  \[
    T_a(L)\subseteq L.
  \]

\bigskip

\noindent\textbf{2. Bevis for \(T_a(P)\subseteq P\).}
\begin{itemize}
  \item[\emph{a)}] Lad \(\mathbf x=(x_1,x_2,x_3)\in P\). Så er
  \[
    x_2 + x_3 = 0.
  \]
  \item[\emph{b)}] Skriv \(\mathbf y = T_a(\mathbf x) = (y_1,y_2,y_3)\). Ifølge definitionen:
  \[
    \begin{aligned}
      y_2 &= (a-1)\,x_2 - x_3,\\
      y_3 &= 2\,x_2 + (a+2)\,x_3.
    \end{aligned}
  \]
  \item[\emph{c)}] Beregn
  \[
    y_2 + y_3
    = \bigl((a-1)x_2 - x_3\bigr) + \bigl(2x_2 + (a+2)x_3\bigr)
    = (a+1)\,x_2 + (a+1)\,x_3
    = (a+1)\,(x_2 + x_3).
  \]
  \item[\emph{d)}] Da \(x_2 + x_3=0\), får vi \(y_2 + y_3 = 0\). Dermed
  \(\mathbf y\in P\) og
  \[
    T_a(P)\subseteq P.
  \]
\end{itemize}




\section[Opgave]{Opgave}
\subsection{}
Vi opstiller et ligningssystem i form af en totalmatrix, hvor vi sætter $u_1, u_2, u_3$ lig hhv. $v_1, v_2, v_3$, og finder løsningerne til disse, ved brug af Gauss-Jordan elimination.\\

$u_1 + u_2 + u_3 = v_1 \Leftrightarrow$\\

$
\left[\begin{array}{ccc|c}
    2 & 0 & 1 & 7\\
    1 & 1 & -1 & -2\\
    -1 & -1 & 2 & 3\\
    1 & 1 & 2 & 1
\end{array}\right]
\begin{array}{ccc}
    \\
    \cdot 2\\
    \cdot (-2)\\
    \cdot 2\\
\end{array}
\leadsto
\left[\begin{array}{ccc|c}
    2 & 0 & 1 & 7\\
    2 & 2 & -2 & -4\\
    2 & 2 & -4 & -6\\
    2 & 2 & 4 & 2
\end{array}\right]
\begin{array}{ccc}
    \\
    -r_1\\
    -r_1\\
    -r_1\\
\end{array}
\leadsto
$\\
$
\left[\begin{array}{ccc|c}
    2 & 0 & 1 & 7\\
    0 & 2 & -3 & -11\\
    0 & 2 & -5 & -13\\
    0 & 2 & 3 & -5
\end{array}\right]
\begin{array}{ccc}
    \\
    \\
    -r_2\\
    -r_2\\
\end{array}
\leadsto
\left[\begin{array}{ccc|c}
    2 & 0 & 1 & 7\\
    0 & 2 & -3 & -11\\
    0 & 0 & -2 & -2\\
    0 & 0 & 6 & 6
\end{array}\right]
\begin{array}{ccc}
    \\
    \\
    \\
    +3r_3\\
\end{array}
\leadsto
$\\
$
\left[\begin{array}{ccc|c}
    2 & 0 & 1 & 7\\
    0 & 2 & -3 & -11\\
    0 & 0 & -2 & -2\\
    0 & 0 & 0 & 0
\end{array}\right]
\begin{array}{ccc}
    \\
    \\
    \cdot (-\frac{1}{2})\\
    \\
\end{array}
\leadsto
\left[\begin{array}{ccc|c}
    2 & 0 & 1 & 7\\
    0 & 2 & -3 & -11\\
    0 & 0 & 1 & 1\\
    0 & 0 & 0 & 0
\end{array}\right]
\begin{array}{ccc}
    -1r_3\\
    +3r_3\\
    \\
    \\
\end{array}
\leadsto
$\\
$
\left[\begin{array}{ccc|c}
    2 & 0 & 0 & 6\\
    0 & 2 & 0 & -8\\
    0 & 0 & 1 & 1\\
    0 & 0 & 0 & 0
\end{array}\right]
\begin{array}{ccc}
    \cdot \frac{1}{2}\\
    \cdot \frac{1}{2}\\
    \\
    \\
\end{array}
\leadsto
\left[\begin{array}{ccc|c}
    1 & 0 & 0 & 3\\
    0 & 1 & 0 & -4\\
    0 & 0 & 1 & 1\\
    0 & 0 & 0 & 0
\end{array}\right]
$\\

Vores første kolonne i $P_{B\leftarrow C}$ er dermed: 
$\left[\begin{array}{ccc}
    3\\
    -4\\
    1\\
    0\\
\end{array}\right]$\\

$u_1 + u_2 + u_3 = v_2 \Leftrightarrow$\\

$
\left[\begin{array}{ccc|c}
    2 & 0 & 1 & -1\\
    1 & 1 & -1 & 0\\
    -1 & -1 & 2 & -1\\
    1 & 1 & 2 & -3
\end{array}\right]
\begin{array}{ccc}
    \\
    \cdot 2\\
    \cdot (-2)\\
    \cdot 2\\
\end{array}
\leadsto
\left[\begin{array}{ccc|c}
    2 & 0 & 1 & -1\\
    2 & 2 & -2 & 0\\
    2 & 2 & -4 & 2\\
    2 & 2 & 4 & -6
\end{array}\right]
\begin{array}{ccc}
    \\
    -r_1\\
    -r_1\\
    -r_1\\
\end{array}
\leadsto
$\\
$
\left[\begin{array}{ccc|c}
    2 & 0 & 1 & -1\\
    0 & 2 & -3 & 1\\
    0 & 2 & -5 & 3\\
    0 & 2 & 3 & -5
\end{array}\right]
\begin{array}{ccc}
    \\
    \\
    -r_2\\
    -r_2\\
\end{array}
\leadsto
\left[\begin{array}{ccc|c}
    2 & 0 & 1 & -1\\
    0 & 2 & -3 & 1\\
    0 & 0 & -2 & 2\\
    0 & 0 & 6 & -6
\end{array}\right]
\begin{array}{ccc}
    \\
    \\
    \\
    +3r_3\\
\end{array}
\leadsto
$\\
$
\left[\begin{array}{ccc|c}
    2 & 0 & 1 & -1\\
    0 & 2 & -3 & 1\\
    0 & 0 & -2 & 2\\
    0 & 0 & 0 & 0
\end{array}\right]
\begin{array}{ccc}
    \\
    \\
    \cdot (-\frac{1}{2})\\
    \\
\end{array}
\leadsto
\left[\begin{array}{ccc|c}
    2 & 0 & 1 & -1\\
    0 & 2 & -3 & 1\\
    0 & 0 & 1 & -1\\
    0 & 0 & 0 & 0
\end{array}\right]
\begin{array}{ccc}
    -1r_3\\
    +3r_3\\
    \\
    \\
\end{array}
\leadsto
$\\
$
\left[\begin{array}{ccc|c}
    2 & 0 & 0 & 0\\
    0 & 2 & 0 & -2\\
    0 & 0 & 1 & -1\\
    0 & 0 & 0 & 0
\end{array}\right]
\begin{array}{ccc}
    \cdot \frac{1}{2}\\
    \cdot \frac{1}{2}\\
    \\
    \\
\end{array}
\leadsto
\left[\begin{array}{ccc|c}
    1 & 0 & 0 & 0\\
    0 & 1 & 0 & -1\\
    0 & 0 & 1 & -1\\
    0 & 0 & 0 & 0
\end{array}\right]
$\\

Vores anden kolonne i $P_{B\leftarrow C}$ er dermed: 
$\left[\begin{array}{ccc}
    0\\
    -1\\
    -1\\
    0\\
\end{array}\right]$\\

$u_1 + u_2 + u_3 = v_3 \Leftrightarrow$\\

$
\left[\begin{array}{ccc|c}
    2 & 0 & 1 & 3\\
    1 & 1 & -1 & -1\\
    -1 & -1 & 2 & 2\\
    1 & 1 & 2 & 2
\end{array}\right]
\begin{array}{ccc}
    \\
    \cdot 2\\
    \cdot (-2)\\
    \cdot 2\\
\end{array}
\leadsto
\left[\begin{array}{ccc|c}
    2 & 0 & 1 & 3\\
    2 & 2 & -2 & -2\\
    2 & 2 & -4 & -4\\
    2 & 2 & 4 & 4
\end{array}\right]
\begin{array}{ccc}
    \\
    -r_1\\
    -r_1\\
    -r_1\\
\end{array}
\leadsto
$\\
$
\left[\begin{array}{ccc|c}
    2 & 0 & 1 & 3\\
    0 & 2 & -3 & -5\\
    0 & 2 & -5 & -7\\
    0 & 2 & 3 & 1
\end{array}\right]
\begin{array}{ccc}
    \\
    \\
    -r_2\\
    -r_2\\
\end{array}
\leadsto
\left[\begin{array}{ccc|c}
    2 & 0 & 1 & 3\\
    0 & 2 & -3 & -5\\
    0 & 0 & -2 & -2\\
    0 & 0 & 6 & 6
\end{array}\right]
\begin{array}{ccc}
    \\
    \\
    \\
    +3r_3\\
\end{array}
\leadsto
$\\
$
\left[\begin{array}{ccc|c}
    2 & 0 & 1 & 3\\
    0 & 2 & -3 & -5\\
    0 & 0 & -2 & -2\\
    0 & 0 & 0 & 0
\end{array}\right]
\begin{array}{ccc}
    \\
    \\
    \cdot (-\frac{1}{2})\\
    \\
\end{array}
\leadsto
\left[\begin{array}{ccc|c}
    2 & 0 & 1 & 3\\
    0 & 2 & -3 & -5\\
    0 & 0 & 1 & 1\\
    0 & 0 & 0 & 0
\end{array}\right]
\begin{array}{ccc}
    -1r_3\\
    +3r_3\\
    \\
    \\
\end{array}
\leadsto
$\\
$
\left[\begin{array}{ccc|c}
    2 & 0 & 0 & 2\\
    0 & 2 & 0 & -2\\
    0 & 0 & 1 & 1\\
    0 & 0 & 0 & 0
\end{array}\right]
\begin{array}{ccc}
    \cdot \frac{1}{2}\\
    \cdot \frac{1}{2}\\
    \\
    \\
\end{array}
\leadsto
\left[\begin{array}{ccc|c}
    1 & 0 & 0 & 1\\
    0 & 1 & 0 & -1\\
    0 & 0 & 1 & 1\\
    0 & 0 & 0 & 0
\end{array}\right]
$\\

Vores sidste kolonne i $P_{B\leftarrow C}$ er dermed: 
$\left[\begin{array}{ccc}
    1\\
    -1\\
    1\\
    0\\
\end{array}\right]$\\

Sammensætter vi nu vores tre kolonner til en matrix, får vi:\\
\[
P_{B\leftarrow C} = 
\left[\begin{array}{ccc}
    3 & 0 & 1\\
    -4 & -1 & -1\\
    1 & -1 & 1\\
\end{array}\right]
\]

\subsection{}
Som givet af opgaven, bestemmes $x$ som $v_1 + v_2 + v_3$:\\

\[
x = 
\left[\begin{array}{ccc}
    7\\
    -2\\
    3\\
    1
\end{array}\right]
+
\left[\begin{array}{ccc}
    -1\\
    0\\
    -1\\
    -1
\end{array}\right]
+
\left[\begin{array}{ccc}
    3\\
    -1\\
    2\\
    2
\end{array}\right]
=
\left[\begin{array}{ccc}
    9\\
    -3\\
    4\\
    0
\end{array}\right]
\]\\

Da konstanterne foran $v$ i hvert led er 1, og $v_1, v_2, v_3 \in \mathcal{C}$, er koordinaterne for $x$ med henhold til $\mathcal{C}$:
\[
[x]_\mathcal{C} = 
\left[\begin{array}{ccc}
    1\\
    1\\
    1
\end{array}\right]
\]

Vi benytter vores basisskriftmatrice til at transformere vores koordinater til basen $\mathcal{B}$ fra $\mathcal{C}$:
\[
[x]_\mathcal{B} = 
\left[\begin{array}{ccc}
    3 & 0 & 1\\
    -4 & -1 & -1\\
    1 & -1 & 1\\
\end{array}\right]
\left[\begin{array}{ccc}
    1\\
    1\\
    1
\end{array}\right]
=
\left[\begin{array}{ccc}
    4\\
    -6\\
    1
\end{array}\right]
\]

\subsection{}

Vi ganger kolonne 2 i vores basisskriftmatrice på $u_1$ og $u_2$:
\[
0\cdot u_1 + (-1) \cdot u_2 + (-1)\cdot u_3 =
-1 \cdot
\left[\begin{array}{ccc}
    0\\
    1\\
    -1\\
    1
\end{array}\right]
+
(-1) \cdot
\left[\begin{array}{ccc}
    1\\
    -1\\
    2\\
    2
\end{array}\right]
=
\left[\begin{array}{ccc}
    -1\\
    0\\
    -1\\
    -3
\end{array}\right]
\]\\

Vi får $v_2$, så $v_2$ må dermed række spannet af $u_2$ og $u_3$, da den kan skrives som en linaerkombination af disse.\\

På samme måde skla vi vise, at $v_1$ hverken tilhører spannet af $\{u_1, u_2\}$, $\{u_2, u_3\}$ eller $\{u_1, u_3\}.$\\
Hvis vi tænker over hvad dette betyder, så skal vi kigge på basisskriftmatricen:\\

\[
[v_1]_{\mathcal{B}} =
\left[\begin{array}{ccc}
    3 \\
    -4\\
    1 
\end{array}\right]
=
3 \cdot u_1 + (-4) \cdot u_2 + 1 \cdot u_3
\]\\

hvilket betyder, at $v_1$ kun kan skrives som en linearkombination af $u_1, u_2$ og $u_3$, hvis de alle tre indgår. Ingen kombination af 2 vektorer af $u_1, u_2, u_3$ kan dermed række spannet af $v_1$.

\subsection{}


\subsection{}


\section{}
\subsection{}

Vi får givet, at koordinatforskydningen svarer til:\\

\[
\left[\begin{array}{ccc}
    s_1-c_1\\
    s_2-c_2\\
    s_1-c_1\\
    s_2-c_2
\end{array}\right]
\]

Tilføjer vi forskydningen til vores nuværende koordinater, kan vi beskrive spillerens nye position som:\\

\[
\left[\begin{array}{ccc}
    c^F_1\\
    c^F_2\\
    s^F_1\\
    s^F_2
\end{array}\right]
=
\left[\begin{array}{ccc}
    c_1\\
    c_2\\
    s_1\\
    s_2
\end{array}\right]
+
\left[\begin{array}{ccc}
    s_1-c_1\\
    s_2-c_2\\
    s_1-c_1\\
    s_2-c_2
\end{array}\right]
=
\left[\begin{array}{ccc}
    s_1\\
    s_2\\
    2s_1-c_1\\
    2s_2-c_2
\end{array}\right]
\Rightarrow
\left[\begin{array}{cccc}
    0 & 0 & 1 & 0\\
    0 & 0 & 0 & 1\\
    -1 & 0 & 2 & 0\\
    0 & -1 & 0 & 2
\end{array}\right]
\]\\


\subsection{}
Rotation mod venstre er bestemt som:\\

\[
\left[\begin{array}{ccc}
    s^L_1\\
    s^L_2
\end{array}\right]
=
\]
\[
\left[\begin{array}{ccc}
    c_1\\
    c_2
\end{array}\right]
+
\left[\begin{array}{cc}
    cos(\theta) & -sin(\theta)\\
    sin(\theta) & cos(\theta)
\end{array}\right]
\left[\begin{array}{cc}
    s_1 -c_1\\
    s_2 -c_2
\end{array}\right]
=
\]
\[
\left[\begin{array}{ccc}
    c_1\\
    c_2
\end{array}\right]
+
\left[\begin{array}{c}
    (s_1-c_1)cos(\theta) - (s_2-c_2) sin(\theta) \\
    (s_1-c_1) sin(\theta) + (s_2-c_2) cos(\theta)
\end{array}\right]
=
\]
\[
\left[\begin{array}{c}
    c_1 +(s_1-c_1)cos(\theta) - (s_2-c_2) sin(\theta) \\
    c_2 + (s_1-c_1) sin(\theta) + (s_2-c_2) cos(\theta)
\end{array}\right]
=
\]
\[
\left[\begin{array}{cccc}
    c_1 - c_1\cdot cos(\theta) + c_2 \cdot sin(\theta) + s_1 \cdot cos(\theta) - s_2\cdot sin(\theta)\\
    -c_1\cdot sin(\theta) + c_2 - c_2 \cdot cos(\theta) + s_1 \cdot sin(\theta) + s_2\cdot cos(\theta)\\
\end{array}\right]
\Rightarrow
\]
\[
\left[\begin{array}{cccc}
    1 - cos(\theta) & sin(\theta) & cos(\theta) &  -sin(\theta)\\
    -sin(\theta) & 1 - cos(\theta) & sin(\theta) & cos(\theta)\\
\end{array}\right]
\]\\

Og som der fremkommer i opgaven, er:\\
\[
\left[\begin{array}{ccc}
    c^L_1\\
    c^L_2
\end{array}\right]
=
\left[\begin{array}{ccc}
    c_1\\
    c_2
\end{array}\right]
\Rightarrow
\left[\begin{array}{cccc}
    1 & 0 & 0 & 0\\
    0 & 1 & 0 & 0
\end{array}\right]
\]\\

Den endelige matrix for rotation mod venstre er dermed bestemt ved følgende variable:\\
\[
L_\theta
=
\left[\begin{array}{cccc}
    1 & 0 & 0 & 0\\
    0 & 1 & 0 & 0\\
    1 - cos(\theta) & sin(\theta) & cos(\theta) &  -sin(\theta)\\
    -sin(\theta) & 1 - cos(\theta) & sin(\theta) & cos(\theta)\\
\end{array}\right]
\]

Vi mindes, at $cos(\theta) = cos(-\theta)$ og $sin(-\theta) = -sin(\theta)$.\\
Rotation mod højre er dermed bestemt som:\\

\[
R_\theta = L_{-\theta}=
\left[\begin{array}{cccc}
    1 & 0 & 0 & 0\\
    0 & 1 & 0 & 0\\
    1 - cos(-\theta) & sin(-\theta) & cos(-\theta) &  -sin(-\theta)\\
    -sin(-\theta) & 1 - cos(-\theta) & sin(-\theta) & cos(-\theta)\\
\end{array}\right]
=
\]
\[
\left[\begin{array}{cccc}
    1 & 0 & 0 & 0 \\
    0 & 1 & 0 & 0 \\
    1 - cos(\theta) & -sin(\theta) & cos(\theta) &  sin(\theta)\\
    sin(\theta) & 1 - cos(\theta) & -sin(\theta) & cos(\theta)
\end{array}\right]
\]

\subsection{}
Ved brug af matrixoperationerne fra python i $project \ A$, får vi følgende matricer efter vi ganger hhv. 'fremad', 'rotation til venstre' og 'rotation til højre' matricerne på til venstre:\\

$
\left[\begin{array}{cccc}
    0 & 0 & 1 & 0\\
    0 & 0 & 0 & 1\\
    -1 & 0 & 2 & 0\\
    0 & -1 & 0 & 2
\end{array}\right]
\left[\begin{array}{cccc}
    0.00000\\
    0.00000\\
    0.00000\\
    1.00000
\end{array}\right]
\leadsto
\left[\begin{array}{cccc}
    0.00000\\
    1.00000\\
    0.00000\\
    2.00000
\end{array}\right]
$\\
$
\left[\begin{array}{cccc}
    1 & 0 & 0 & 0 \\
    0 & 1 & 0 & 0 \\
    1 - cos(\theta) & -sin(\theta) & cos(\theta) &  sin(\theta)\\
    sin(\theta) & 1 - cos(\theta) & -sin(\theta) & cos(\theta)
\end{array}\right]
\left[\begin{array}{cccc}
    0.00000\\
    1.00000\\
    0.00000\\
    2.00000
\end{array}\right]
\leadsto
\left[\begin{array}{cccc}
    0.00000\\
    1.00000\\
    0.91295\\
    1.40808
\end{array}\right]
$\\
$
\left[\begin{array}{cccc}
    1 & 0 & 0 & 0 \\
    0 & 1 & 0 & 0 \\
    1 - cos(\theta) & -sin(\theta) & cos(\theta) &  sin(\theta)\\
    sin(\theta) & 1 - cos(\theta) & -sin(\theta) & cos(\theta)
\end{array}\right]
\left[\begin{array}{cccc}
    0.00000\\
    1.00000\\
    0.91295\\
    1.40808
\end{array}\right]
\leadsto
\left[\begin{array}{cccc}
    0.00000\\
    1.00000\\
    0.74511\\
    0.33306
\end{array}\right]
$\\
$
\left[\begin{array}{cccc}
    0 & 0 & 1 & 0\\
    0 & 0 & 0 & 1\\
    -1 & 0 & 2 & 0\\
    0 & -1 & 0 & 2
\end{array}\right]
\left[\begin{array}{cccc}
    0.00000\\
    1.00000\\
    0.74511\\
    0.33306
\end{array}\right]
\leadsto
\left[\begin{array}{cccc}
    0.74511\\
    0.33306\\
    1.49023\\
    -0.33388
\end{array}\right]
$\\
$
\left[\begin{array}{cccc}
    0 & 0 & 1 & 0\\
    0 & 0 & 0 & 1\\
    -1 & 0 & 2 & 0\\
    0 & -1 & 0 & 2
\end{array}\right]
\left[\begin{array}{cccc}
    0.74511\\
    0.33306\\
    1.49023\\
    -0.33388
\end{array}\right]
\leadsto
\left[\begin{array}{cccc}
    1.49023\\
    -0.33388\\
    2.23534\\
    -1.00081
\end{array}\right]
$\\
$
\left[\begin{array}{cccc}
    1 & 0 & 0 & 0\\
    0 & 1 & 0 & 0\\
    1 - cos(\theta) & sin(\theta) & cos(\theta) &  -sin(\theta)\\
    -sin(\theta) & 1 - cos(\theta) & sin(\theta) & cos(\theta)\\
\end{array}\right]
\left[\begin{array}{cccc}
    1.49023\\
    -0.33388\\
    2.23534\\
    -1.00081
\end{array}\right]
\leadsto
\left[\begin{array}{cccc}
    1.49023\\
    -0.33388\\
    2.40317\\
    0.07421
\end{array}\right]
$\\
$
\left[\begin{array}{cccc}
    1 & 0 & 0 & 0\\
    0 & 1 & 0 & 0\\
    1 - cos(\theta) & sin(\theta) & cos(\theta) &  -sin(\theta)\\
    -sin(\theta) & 1 - cos(\theta) & sin(\theta) & cos(\theta)\\
\end{array}\right]
\left[\begin{array}{cccc}
    1.49023\\
    -0.33388\\
    2.40317\\
    0.07421
\end{array}\right]
\leadsto
\left[\begin{array}{cccc}
    1.49023\\
    -0.33388\\
    1.49023\\
    0.66612
\end{array}\right]
$\\
$
\left[\begin{array}{cccc}
    1 & 0 & 0 & 0\\
    0 & 1 & 0 & 0\\
    1 - cos(\theta) & sin(\theta) & cos(\theta) &  -sin(\theta)\\
    -sin(\theta) & 1 - cos(\theta) & sin(\theta) & cos(\theta)\\
\end{array}\right]
\left[\begin{array}{cccc}
    1.49023\\
    -0.33388\\
    1.49023\\
    0.66612
\end{array}\right]
\leadsto
\left[\begin{array}{cccc}
    1.49023\\
    -0.33388\\
    0.57728\\
    0.07421
\end{array}\right]
$\\
$
\left[\begin{array}{cccc}
    0 & 0 & 1 & 0\\
    0 & 0 & 0 & 1\\
    -1 & 0 & 2 & 0\\
    0 & -1 & 0 & 2
\end{array}\right]
\left[\begin{array}{cccc}
    1.49023\\
    -0.33388\\
    0.57728\\
    0.07421
\end{array}\right]
\leadsto
\left[\begin{array}{cccc}
    0.57728\\
    0.07421\\
    -0.33566\\
    0.48229
\end{array}\right]
$\\

Efter alle 9 multiplikationer fra venste ender vi med postionen af spilleren og sidsen svarende til matricen:\\
\[
\left[\begin{array}{cccc}
    0.57728\\
    0.07421\\
    -0.33566\\
    0.48229
\end{array}\right]
\]

\subsection{}
At gange vores 'rotation mod højre' matrice på sig selv svarer til at gange det antal gange med vinkeln $\theta$, da:
\[
R_{\theta 1} \cdot R_{\theta 2} = R_{\theta 1 + \theta 2}
\]
og
\[
(R_\theta)^{n} = \prod_{i=1}^{n} R_{\theta i} = R_{\theta 1 + \theta 2 + \ \dots \ + \theta n }
\]
Vi kan dermed beregne $(R_{20})^{18}$ til:
\[
(R_{20})^{18} = \prod_{i=1}^{18} R_{20} = R_{20 \cdot 18} = R_{360}
\]
Med vores nye vinkel beregnet, kan vi nu indsætte $360$ på $\theta $-s plads i $R_\theta$:
\[
\left[\begin{array}{cccc}
    1 & 0 & 0 & 0 \\
    0 & 1 & 0 & 0 \\
    1 - cos(360) & -sin(360) & cos(360) &  sin(360)\\
    sin(360) & 1 - cos(360) & -sin(360) & cos(360)
\end{array}\right]
=
\left[\begin{array}{cccc}
    1 & 0 & 0 & 0 \\
    0 & 1 & 0 & 0 \\
    1 - 1 & 0 & 1 &  0\\
    0 & 1 - 1 & 0 & 1
\end{array}\right]
=
\left[\begin{array}{cccc}
    1 & 0 & 0 & 0 \\
    0 & 1 & 0 & 0 \\
    0 & 0 & 1 & 0\\
    0 & 0 & 0 & 1
\end{array}\right]
\]
Således ender vi med enhedsmatricen $I_4$.

\section[Opgave]{Opgave}
Se vedhæftede python-fil.



\end{document}