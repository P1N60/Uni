\documentclass[a4paper,12pt]{article}
\usepackage{standalone}
\usepackage{amsmath}
\input{../../sty/setup.sty}

% Changes sections from 1.1 to 1.a
\renewcommand{\thesubsection}{\thesection.\alph{subsection}}
\graphicspath{ {../../pictures/IDMA/IDMA3_a}} 
\title{Københavns Universitet\\
LinAlgDat - Project B}
\author{Victor Vangkilde Jørgensen - kft410\\ 
kft410@alumni.ku.dk\\
Hold 13 Mach}

\begin{document}
\author{Victor Vangkilde Jørgensen - kft410\\ 
kft410@alumni.ku.dk}
\makeatletter
\let\getauthor\@author
\let\gettitle\@title
\makeatother
\maketitle
\thispagestyle{empty}
\n\n
 

\pagebreak
\pagestyle{empty}
\tableofcontents
\pagebreak
\pagestyle{fancy}
\fancyhf{}
\setlength{\headheight}{15.2pt}
\renewcommand{\footrulewidth}{0.4pt}
\fancyhead[R]{\nouppercase \lastrightmark}
\fancyfoot[L]{\gettitle}
\fancyfoot[R]{\thepage}
 
\maketitle 

\section[Opgave]{Opgave}
\subsection{}

Vi forkaster $x_1, x_2, x_3$, og bruger deres konstanter til at aflæse $M_a$ til:\\
\[
\left[\begin{array}{ccc}
    a & -1 & -1 \\
    0 & a-1 & -1 \\
    0 & 2 & a+2
\end{array}\right]
\]
$x_1, x_2, x_3$ droppes, da disse kun indgår, når vi ganger $M_a$ med $\left[\begin{array}{ccc|c}
    x_1 \\
    x_2 \\
    x_3
\end{array}\right]$.



\subsection{}
Vi ser først på, om $T_a$ er injektiv. En transformation er injektiv, hvis kernen af transformationen kun er \{0\}. Det vil sige, at kun nulvektoren transformeret giver nulvektoren.\\

$
\left[\begin{array}{ccc|c}
    a & -1 & -1 & 0 \\
    0 & a-1 & -1 & 0 \\
    0 & 2 & a+2 & 0
\end{array}\right]
\begin{array}{ccc}
    \\
    \cdot \frac{1}{a-1} \\
    \\
\end{array}
\leadsto
\left[\begin{array}{ccc|c}
    a & -1 & -1 & 0 \\
    0 & 1 & -\frac{1}{a-1} & 0 \\
    0 & 2 & a+2 & 0
\end{array}\right]
\begin{array}{ccc}
    \\
    \\
    -2r_2\\
\end{array}
\leadsto
$\\
$
\left[\begin{array}{ccc|c}
    a & -1 & -1 & 0 \\
    0 & 1 & -\frac{1}{a-1} & 0 \\
    0 & 0 & \frac{a^2+a}{a-1} & 0
\end{array}\right]
\begin{array}{ccc}
    \\
    \\
    \cdot \frac{a-1}{a^2+a}\\
\end{array}
\leadsto
\left[\begin{array}{ccc|c}
    a & -1 & -1 & 0 \\
    0 & 1 & -\frac{1}{a-1} & 0 \\
    0 & 0 & 1 & 0
\end{array}\right]
\begin{array}{ccc}
    +r_3\\
    +r_3 \cdot \frac{1}{a-1}\\
    \\
\end{array}
\leadsto
$\\
$
\left[\begin{array}{ccc|c}
    a & -1 & 0 & 0 \\
    0 & 1 & 0 & 0 \\
    0 & 0 & 1 & 0
\end{array}\right]
\begin{array}{ccc}
    +r_2\\
    \\
    \\
\end{array}
\leadsto
\left[\begin{array}{ccc|c}
    a & 0 & 0 & 0 \\
    0 & 1 & 0 & 0 \\
    0 & 0 & 1 & 0
\end{array}\right]
\begin{array}{ccc}
    \cdot \frac{1}{a}\\
    \\
    \\
\end{array}
\leadsto
$\\
$
\left[\begin{array}{ccc|c}
    1 & 0 & 0 & 0 \\
    0 & 1 & 0 & 0 \\
    0 & 0 & 1 & 0
\end{array}\right]
$\\

Da ker $T_a = \{0\}$, er $T_a$ injektiv.\\

$T_a$ er surjektiv, hvis ran $T_a$ = $\R^3$, da vores vektorer har 3 koordinater.

\[
rank (T_a) = dim(ran (T_a)) = 3
\]\\
da vi har 3 pivotelementer.\\

Da dimissionen af ran $T_a$ er lig dimissionen af vektoerene der udgør $T_a$, er $T_a$ surjektiv. $T_a$ er dermed bijektiv, da den både er injektiv og surjektiv.\\

Vi bestemmer nu $T^{-1}_a$, ved at sætte enhedsmatricen på til højre, og reducere med Gauss-Jordan:\\

$
\left[\begin{array}{ccc|ccc}
    a & -1 & -1 & 1 & 0 & 0 \\
    0 & a-1 & -1 & 0 & 1 & 0 \\
    0 & 2 & a+2 & 0 & 0 & 1
\end{array}\right]
\begin{array}{ccc}
    \\
    \cdot \frac{1}{a-1} \\
    \\
\end{array}
\leadsto
\left[\begin{array}{ccc|ccc}
    a & -1 & -1 & 1 & 0 & 0 \\
    0 & 1 & -\frac{1}{a-1} & 0 & \frac{1}{a-1} & 0 \\
    0 & 2 & a+2 & 0 & 0 & 1
\end{array}\right]
\begin{array}{ccc}
    \\
    \\
    -2r_2\\
\end{array}
\leadsto
$\\
$
\left[\begin{array}{ccc|ccc}
    a & -1 & -1 & 1 & 0 & 0 \\
    0 & 1 & -\frac{1}{a-1} & 0 & \frac{1}{a-1} & 0 \\
    0 & 0 & \frac{a^2+a}{a-1} & 0 & -\frac{2}{a-1} & 1
\end{array}\right]
\begin{array}{ccc}
    \\
    \\
    \cdot \frac{a-1}{a^2+a}\\
\end{array}
\leadsto
\left[\begin{array}{ccc|ccc}
    a & -1 & -1 & 1 & 0 & 0 \\
    0 & 1 & -\frac{1}{a-1} & 0 & \frac{1}{a-1} & 0 \\
    0 & 0 & 1 & 0 & -\frac{2}{a^2+a} & \frac{a-1}{a^2+a}
\end{array}\right]
\begin{array}{ccc}
    +r_3\\
    +r_3 \cdot \frac{1}{a-1}\\
    \\
\end{array}
\leadsto
$\\
$
\left[\begin{array}{ccc|ccc}
    a & -1 & 0 & 1 & -\frac{2}{a^2+a} & \frac{a-1}{a^2+a} \\
    0 & 1 & 0 & 0 & \frac{a+2}{a^2+a} & \frac{1}{a^2+a} \\
    0 & 0 & 1 & 0 & -\frac{2}{a^2+a} & \frac{a-1}{a^2+a}
\end{array}\right]
\begin{array}{ccc}
    +r_2\\
    \\
    \\
\end{array}
\leadsto
\left[\begin{array}{ccc|ccc}
    a & 0 & 0 & 1 & \frac{1}{a+1} & \frac{1}{a+1} \\
    0 & 1 & 0 & 0 & \frac{a+2}{a^2+a} & \frac{1}{a^2+a} \\
    0 & 0 & 1 & 0 & -\frac{2}{a^2+a} & \frac{a-1}{a^2+a}
\end{array}\right]
\begin{array}{ccc}
    \cdot \frac{1}{a}\\
    \\
    \\
\end{array}
\leadsto
$\\
$
\left[\begin{array}{ccc|ccc}
    1 & 0 & 0 & \frac{1}{a} & \frac{1}{a^2+a} & \frac{1}{a^2+a} \\
    0 & 1 & 0 & 0 & \frac{a+2}{a^2+a} & \frac{1}{a^2+a} \\
    0 & 0 & 1 & 0 & -\frac{2}{a^2+a} & \frac{a-1}{a^2+a}
\end{array}\right]
$\\

\[
T^{-1}_a=
\left[\begin{array}{ccc}
    \frac{1}{a} & \frac{1}{a^2+a} & \frac{1}{a^2+a} \\
    0 & \frac{a+2}{a^2+a} & \frac{1}{a^2+a} \\
    0 & -\frac{2}{a^2+a} & \frac{a-1}{a^2+a}
\end{array}\right]
\]

\subsection{}
Vi opstill igen $T_a$, hvor $a = -1$:\\

$
\left[\begin{array}{ccc|c}
    -1 & -1 & -1 & 0 \\
    0 & -1-1 & -1 & 0 \\
    0 & 2 & -1+2 & 0
\end{array}\right]
\begin{array}{ccc}
    \\
    reducer\\
    \\
\end{array}
\leadsto
\left[\begin{array}{ccc|c}
    -1 & -1 & -1 & 0 \\
    0 & -2 & -1 & 0 \\
    0 & 2 & 1 & 0
\end{array}\right]
\begin{array}{ccc}
    \\
    \\
    +r_2\\
\end{array}
\leadsto
$\\
$
\left[\begin{array}{ccc|c}
    -1 & -1 & -1 & 0 \\
    0 & -2 & -1 & 0 \\
    0 & 0 & 0 & 0
\end{array}\right]
\begin{array}{ccc}
    \cdot (-1)\\
    \cdot (-1)\\
    \\
\end{array}
\leadsto
\left[\begin{array}{ccc|c}
    1 & 1 & 1 & 0 \\
    0 & 2 & 1 & 0 \\
    0 & 0 & 0 & 0
\end{array}\right]
\begin{array}{ccc}
    \\
    \cdot \frac{1}{2}\\
    \\
\end{array}
\leadsto
$\\
$
\left[\begin{array}{ccc|c}
    1 & 1 & 1 & 0 \\
    0 & 1 & \frac{1}{2} & 0 \\
    0 & 0 & 0 & 0
\end{array}\right]
\begin{array}{ccc}
    -r_2\\
    \\
    \\
\end{array}
\leadsto
\left[\begin{array}{ccc|c}
    1 & 0 & \frac{1}{2} & 0 \\
    0 & 1 & \frac{1}{2} & 0 \\
    0 & 0 & 0 & 0
\end{array}\right]
$\\

Vi aflæser løsningerne til:\\
\[
\left[\begin{array}{ccc}
    x_1\\
    x_2\\
    x_3\\
\end{array}\right]
=
t
\left[\begin{array}{ccc}
    -\frac{1}{2}\\
    -\frac{1}{2}\\
    1\\
\end{array}\right]
\]

Vi gør det samme for $a= 0$:\\

$
\left[\begin{array}{ccc|c}
    0 & -1 & -1 & 0\\
    0 & (0-1) & -1 & 0\\
    0 & 2 & (0+2) & 0
\end{array}\right]
\begin{array}{ccc}
    \\
    reducer\\
    \\
\end{array}
\leadsto
\left[\begin{array}{ccc|c}
    0 & -1 & -1 & 0 \\
    0 & -1 & -1 & 0 \\
    0 & 2 & 2 & 0
\end{array}\right]
\begin{array}{ccc}
    \\
    -r_1\\
    +2r_2\\
\end{array}
\leadsto
$\\
$
\left[\begin{array}{ccc|c}
    0 & -1 & -1 & 0 \\
    0 & 0 & 0 & 0 \\
    0 & 0 & 0 & 0
\end{array}\right]
\begin{array}{ccc}
    \cdot (-1)\\
    \\
    \\
\end{array}
\leadsto
\left[\begin{array}{ccc|c}
    0 & 1 & 1 & 0 \\
    0 & 0 & 0 & 0 \\
    0 & 0 & 0 & 0
\end{array}\right]
$\\

\subsection{}


\subsection{}

\section[Opgave]{Opgave}
\subsection{}
Vi opstiller et ligningssystem i form af en totalmatrix, hvor vi sætter $u_1, u_2, u_3$ lig hhv. $v_1, v_2, v_3$, og finder løsningerne til disse, ved brug af Gauss-Jordan elimination.\\

$u_1 + u_2 + u_3 = v_1 \Leftrightarrow$\\

$
\left[\begin{array}{ccc|c}
    2 & 0 & 1 & 7\\
    1 & 1 & -1 & -2\\
    -1 & -1 & 2 & 3\\
    1 & 1 & 2 & 1
\end{array}\right]
\begin{array}{ccc}
    \\
    \cdot 2\\
    \cdot (-2)\\
    \cdot 2\\
\end{array}
\leadsto
\left[\begin{array}{ccc|c}
    2 & 0 & 1 & 7\\
    2 & 2 & -2 & -4\\
    2 & 2 & -4 & -6\\
    2 & 2 & 4 & 2
\end{array}\right]
\begin{array}{ccc}
    \\
    -r_1\\
    -r_1\\
    -r_1\\
\end{array}
\leadsto
$\\
$
\left[\begin{array}{ccc|c}
    2 & 0 & 1 & 7\\
    0 & 2 & -3 & -11\\
    0 & 2 & -5 & -13\\
    0 & 2 & 3 & -5
\end{array}\right]
\begin{array}{ccc}
    \\
    \\
    -r_2\\
    -r_2\\
\end{array}
\leadsto
\left[\begin{array}{ccc|c}
    2 & 0 & 1 & 7\\
    0 & 2 & -3 & -11\\
    0 & 0 & -2 & -2\\
    0 & 0 & 6 & 6
\end{array}\right]
\begin{array}{ccc}
    \\
    \\
    \\
    +3r_3\\
\end{array}
\leadsto
$\\
$
\left[\begin{array}{ccc|c}
    2 & 0 & 1 & 7\\
    0 & 2 & -3 & -11\\
    0 & 0 & -2 & -2\\
    0 & 0 & 0 & 0
\end{array}\right]
\begin{array}{ccc}
    \\
    \\
    \cdot (-\frac{1}{2})\\
    \\
\end{array}
\leadsto
\left[\begin{array}{ccc|c}
    2 & 0 & 1 & 7\\
    0 & 2 & -3 & -11\\
    0 & 0 & 1 & 1\\
    0 & 0 & 0 & 0
\end{array}\right]
\begin{array}{ccc}
    -1r_3\\
    +3r_3\\
    \\
    \\
\end{array}
\leadsto
$\\
$
\left[\begin{array}{ccc|c}
    2 & 0 & 0 & 6\\
    0 & 2 & 0 & -8\\
    0 & 0 & 1 & 1\\
    0 & 0 & 0 & 0
\end{array}\right]
\begin{array}{ccc}
    \cdot \frac{1}{2}\\
    \cdot \frac{1}{2}\\
    \\
    \\
\end{array}
\leadsto
\left[\begin{array}{ccc|c}
    1 & 0 & 0 & 3\\
    0 & 1 & 0 & -4\\
    0 & 0 & 1 & 1\\
    0 & 0 & 0 & 0
\end{array}\right]
$\\

Vores første kolonne i $P_{B\leftarrow C}$ er dermed: 
$\left[\begin{array}{ccc}
    3\\
    -4\\
    1\\
    0\\
\end{array}\right]$\\

$u_1 + u_2 + u_3 = v_2 \Leftrightarrow$\\

$
\left[\begin{array}{ccc|c}
    2 & 0 & 1 & -1\\
    1 & 1 & -1 & 0\\
    -1 & -1 & 2 & -1\\
    1 & 1 & 2 & -3
\end{array}\right]
\begin{array}{ccc}
    \\
    \cdot 2\\
    \cdot (-2)\\
    \cdot 2\\
\end{array}
\leadsto
\left[\begin{array}{ccc|c}
    2 & 0 & 1 & -1\\
    2 & 2 & -2 & 0\\
    2 & 2 & -4 & 2\\
    2 & 2 & 4 & -6
\end{array}\right]
\begin{array}{ccc}
    \\
    -r_1\\
    -r_1\\
    -r_1\\
\end{array}
\leadsto
$\\
$
\left[\begin{array}{ccc|c}
    2 & 0 & 1 & -1\\
    0 & 2 & -3 & 1\\
    0 & 2 & -5 & 3\\
    0 & 2 & 3 & -5
\end{array}\right]
\begin{array}{ccc}
    \\
    \\
    -r_2\\
    -r_2\\
\end{array}
\leadsto
\left[\begin{array}{ccc|c}
    2 & 0 & 1 & -1\\
    0 & 2 & -3 & 1\\
    0 & 0 & -2 & 2\\
    0 & 0 & 6 & -6
\end{array}\right]
\begin{array}{ccc}
    \\
    \\
    \\
    +3r_3\\
\end{array}
\leadsto
$\\
$
\left[\begin{array}{ccc|c}
    2 & 0 & 1 & -1\\
    0 & 2 & -3 & 1\\
    0 & 0 & -2 & 2\\
    0 & 0 & 0 & 0
\end{array}\right]
\begin{array}{ccc}
    \\
    \\
    \cdot (-\frac{1}{2})\\
    \\
\end{array}
\leadsto
\left[\begin{array}{ccc|c}
    2 & 0 & 1 & -1\\
    0 & 2 & -3 & 1\\
    0 & 0 & 1 & -1\\
    0 & 0 & 0 & 0
\end{array}\right]
\begin{array}{ccc}
    -1r_3\\
    +3r_3\\
    \\
    \\
\end{array}
\leadsto
$\\
$
\left[\begin{array}{ccc|c}
    2 & 0 & 0 & 0\\
    0 & 2 & 0 & -2\\
    0 & 0 & 1 & -1\\
    0 & 0 & 0 & 0
\end{array}\right]
\begin{array}{ccc}
    \cdot \frac{1}{2}\\
    \cdot \frac{1}{2}\\
    \\
    \\
\end{array}
\leadsto
\left[\begin{array}{ccc|c}
    1 & 0 & 0 & 0\\
    0 & 1 & 0 & -1\\
    0 & 0 & 1 & -1\\
    0 & 0 & 0 & 0
\end{array}\right]
$\\

Vores anden kolonne i $P_{B\leftarrow C}$ er dermed: 
$\left[\begin{array}{ccc}
    0\\
    -1\\
    -1\\
    0\\
\end{array}\right]$\\

$u_1 + u_2 + u_3 = v_3 \Leftrightarrow$\\

$
\left[\begin{array}{ccc|c}
    2 & 0 & 1 & 3\\
    1 & 1 & -1 & -1\\
    -1 & -1 & 2 & 2\\
    1 & 1 & 2 & 2
\end{array}\right]
\begin{array}{ccc}
    \\
    \cdot 2\\
    \cdot (-2)\\
    \cdot 2\\
\end{array}
\leadsto
\left[\begin{array}{ccc|c}
    2 & 0 & 1 & 3\\
    2 & 2 & -2 & -2\\
    2 & 2 & -4 & -4\\
    2 & 2 & 4 & 4
\end{array}\right]
\begin{array}{ccc}
    \\
    -r_1\\
    -r_1\\
    -r_1\\
\end{array}
\leadsto
$\\
$
\left[\begin{array}{ccc|c}
    2 & 0 & 1 & 3\\
    0 & 2 & -3 & -5\\
    0 & 2 & -5 & -7\\
    0 & 2 & 3 & 1
\end{array}\right]
\begin{array}{ccc}
    \\
    \\
    -r_2\\
    -r_2\\
\end{array}
\leadsto
\left[\begin{array}{ccc|c}
    2 & 0 & 1 & 3\\
    0 & 2 & -3 & -5\\
    0 & 0 & -2 & -2\\
    0 & 0 & 6 & 6
\end{array}\right]
\begin{array}{ccc}
    \\
    \\
    \\
    +3r_3\\
\end{array}
\leadsto
$\\
$
\left[\begin{array}{ccc|c}
    2 & 0 & 1 & 3\\
    0 & 2 & -3 & -5\\
    0 & 0 & -2 & -2\\
    0 & 0 & 0 & 0
\end{array}\right]
\begin{array}{ccc}
    \\
    \\
    \cdot (-\frac{1}{2})\\
    \\
\end{array}
\leadsto
\left[\begin{array}{ccc|c}
    2 & 0 & 1 & 3\\
    0 & 2 & -3 & -5\\
    0 & 0 & 1 & 1\\
    0 & 0 & 0 & 0
\end{array}\right]
\begin{array}{ccc}
    -1r_3\\
    +3r_3\\
    \\
    \\
\end{array}
\leadsto
$\\
$
\left[\begin{array}{ccc|c}
    2 & 0 & 0 & 2\\
    0 & 2 & 0 & -2\\
    0 & 0 & 1 & 1\\
    0 & 0 & 0 & 0
\end{array}\right]
\begin{array}{ccc}
    \cdot \frac{1}{2}\\
    \cdot \frac{1}{2}\\
    \\
    \\
\end{array}
\leadsto
\left[\begin{array}{ccc|c}
    1 & 0 & 0 & 1\\
    0 & 1 & 0 & -1\\
    0 & 0 & 1 & 1\\
    0 & 0 & 0 & 0
\end{array}\right]
$\\

Vores sidste kolonne i $P_{B\leftarrow C}$ er dermed: 
$\left[\begin{array}{ccc}
    1\\
    -1\\
    1\\
    0\\
\end{array}\right]$\\

Sammensætter vi nu vores tre kolonner til en matrix, får vi:\\
\[
P_{B\leftarrow C} = 
\left[\begin{array}{ccc}
    3 & 0 & 1\\
    -4 & -1 & -1\\
    1 & -1 & 1\\
\end{array}\right]
\]

\subsection{}
\[
x = 
\left[\begin{array}{ccc}
    7\\
    -2\\
    3\\
    1
\end{array}\right]
+
\left[\begin{array}{ccc}
    -1\\
    0\\
    -1\\
    -1
\end{array}\right]
+
\left[\begin{array}{ccc}
    3\\
    -1\\
    2\\
    2
\end{array}\right]
=
\left[\begin{array}{ccc}
    9\\
    -3\\
    4\\
    0
\end{array}\right]
\]\\

Da konstanterne foran $v$ i hvert led er 1, og $v_1, v_2, v_3 \in \mathcal{C}$, er koordinaterne for $x$ med henhold til $\mathcal{C}$:
\[
[x]_\mathcal{C} = 
\left[\begin{array}{ccc}
    1\\
    1\\
    1
\end{array}\right]
\]

Vi benytter vores basisskriftmatrice til at transformere vores koordinater til basen $\mathcal{B}$ fra $\mathcal{C}$:
\[
[x]_\mathcal{B} = 
\left[\begin{array}{ccc}
    3 & 0 & 1\\
    -4 & -1 & -1\\
    1 & -1 & 1\\
\end{array}\right]
\left[\begin{array}{ccc}
    1\\
    1\\
    1
\end{array}\right]
=
\left[\begin{array}{ccc}
    4\\
    -6\\
    1
\end{array}\right]
\]

\subsection{}

Vi ganger kolonne 2 i vores basisskriftmatrice på $u_1$ og $u_2$:
\[
-1 \cdot u_1 + (-1)\cdot u_2 =
-1 \cdot
\left[\begin{array}{ccc}
    0\\
    1\\
    -1\\
    1
\end{array}\right]
+
(-1) \cdot
\left[\begin{array}{ccc}
    1\\
    -1\\
    2\\
    2
\end{array}\right]
=
\left[\begin{array}{ccc}
    -1\\
    0\\
    -1\\
    -3
\end{array}\right]
\]\\

Vi får $v_2$, så $v_2$ må dermed række spannet af $u2, u3$.\\

Mangler at lave resten af opgaven

\subsection{}


\subsection{}


\section{}
\subsection{}

Vi får givet, at koordinatforskydningen svarer til:\\

\[
\left[\begin{array}{ccc}
    s_1-c_1\\
    s_2-c_2\\
    s_1-c_1\\
    s_2-c_2
\end{array}\right]
\]

Tilføjer vi forskydningen til vores nuværende koordinater, kan vi beskrive spillerens nye position som:\\

\[
\left[\begin{array}{ccc}
    c^F_1\\
    c^F_2\\
    s^F_1\\
    s^F_2
\end{array}\right]
=
\left[\begin{array}{ccc}
    c_1\\
    c_2\\
    s_1\\
    s_2
\end{array}\right]
+
\left[\begin{array}{ccc}
    s_1-c_1\\
    s_2-c_2\\
    s_1-c_1\\
    s_2-c_2
\end{array}\right]
=
\left[\begin{array}{ccc}
    s_1\\
    s_2\\
    2s_1-c_1\\
    2s_2-c_2
\end{array}\right]
\Rightarrow
\left[\begin{array}{cccc}
    0 & 0 & 1 & 0\\
    0 & 0 & 0 & 1\\
    -1 & 0 & 2 & 0\\
    0 & -1 & 0 & 2
\end{array}\right]
\]\\


\subsection{}
Rotation mod venstre er bestemt som:\\

\[
\left[\begin{array}{ccc}
    s^L_1\\
    s^L_2
\end{array}\right]
=
\]
\[
\left[\begin{array}{ccc}
    c_1\\
    c_2
\end{array}\right]
+
\left[\begin{array}{cc}
    cos(\theta) & -sin(\theta)\\
    sin(\theta) & cos(\theta)
\end{array}\right]
\left[\begin{array}{cc}
    s_1 -c_1\\
    s_2 -c_2
\end{array}\right]
=
\]
\[
\left[\begin{array}{ccc}
    c_1\\
    c_2
\end{array}\right]
+
\left[\begin{array}{c}
    (s_1-c_1)cos(\theta) - (s_2-c_2) sin(\theta) \\
    (s_1-c_1) sin(\theta) + (s_2-c_2) cos(\theta)
\end{array}\right]
=
\]
\[
\left[\begin{array}{c}
    c_1 +(s_1-c_1)cos(\theta) - (s_2-c_2) sin(\theta) \\
    c_2 + (s_1-c_1) sin(\theta) + (s_2-c_2) cos(\theta)
\end{array}\right]
=
\]
\[
\left[\begin{array}{cccc}
    c_1 - c_1\cdot cos(\theta) + c_2 \cdot sin(\theta) + s_1 \cdot cos(\theta) - s_2\cdot sin(\theta)\\
    -c_1\cdot sin(\theta) + c_2 - c_2 \cdot cos(\theta) + s_1 \cdot sin(\theta) + s_2\cdot cos(\theta)\\
\end{array}\right]
\Rightarrow
\]
\[
\left[\begin{array}{cccc}
    1 - cos(\theta) & sin(\theta) & cos(\theta) &  -sin(\theta)\\
    -sin(\theta) & 1 - cos(\theta) & sin(\theta) & cos(\theta)\\
\end{array}\right]
\]\\

Og som der fremkommer i opgaven, er:\\
\[
\left[\begin{array}{ccc}
    c^L_1\\
    c^L_2
\end{array}\right]
=
\left[\begin{array}{ccc}
    c_1\\
    c_2
\end{array}\right]
\Rightarrow
\left[\begin{array}{cccc}
    1 & 0 & 0 & 0\\
    0 & 1 & 0 & 0
\end{array}\right]
\]\\

Den endelige matrix for rotation mod venstre er dermed bestemt ved følgende variable:\\
\[
L_\theta
=
\left[\begin{array}{cccc}
    1 & 0 & 0 & 0\\
    0 & 1 & 0 & 0\\
    1 - cos(\theta) & sin(\theta) & cos(\theta) &  -sin(\theta)\\
    -sin(\theta) & 1 - cos(\theta) & sin(\theta) & cos(\theta)\\
\end{array}\right]
\]

Vi mindes, at $cos(\theta) = cos(-\theta)$ og $sin(-\theta) = -sin(\theta)$.\\
Rotation mod højre er dermed bestemt som:\\

\[
R_\theta = L_{-\theta}=
\left[\begin{array}{cccc}
    1 & 0 & 0 & 0\\
    0 & 1 & 0 & 0\\
    1 - cos(-\theta) & sin(-\theta) & cos(-\theta) &  -sin(-\theta)\\
    -sin(-\theta) & 1 - cos(-\theta) & sin(-\theta) & cos(-\theta)\\
\end{array}\right]
=
\]
\[
\left[\begin{array}{cccc}
    1 & 0 & 0 & 0 \\
    0 & 1 & 0 & 0 \\
    1 - cos(\theta) & -sin(\theta) & cos(\theta) &  sin(\theta)\\
    sin(\theta) & 1 - cos(\theta) & -sin(\theta) & cos(\theta)
\end{array}\right]
\]

\subsection{}
Ved brug af matrixoperationerne fra python i $project \ A$, får vi følgende matricer efter vi ganger hhv. 'fremad', 'rotation til venstre' og 'rotation til højre' matricerne på til venstre:\\

$
\left[\begin{array}{cccc}
    0 & 0 & 1 & 0\\
    0 & 0 & 0 & 1\\
    -1 & 0 & 2 & 0\\
    0 & -1 & 0 & 2
\end{array}\right]
\left[\begin{array}{cccc}
    0.00000\\
    0.00000\\
    0.00000\\
    1.00000
\end{array}\right]
\leadsto
\left[\begin{array}{cccc}
    0.00000\\
    1.00000\\
    0.00000\\
    2.00000
\end{array}\right]
$\\
$
\left[\begin{array}{cccc}
    1 & 0 & 0 & 0 \\
    0 & 1 & 0 & 0 \\
    1 - cos(\theta) & -sin(\theta) & cos(\theta) &  sin(\theta)\\
    sin(\theta) & 1 - cos(\theta) & -sin(\theta) & cos(\theta)
\end{array}\right]
\left[\begin{array}{cccc}
    0.00000\\
    1.00000\\
    0.00000\\
    2.00000
\end{array}\right]
\leadsto
\left[\begin{array}{cccc}
    0.00000\\
    1.00000\\
    0.91295\\
    1.40808
\end{array}\right]
$\\
$
\left[\begin{array}{cccc}
    1 & 0 & 0 & 0 \\
    0 & 1 & 0 & 0 \\
    1 - cos(\theta) & -sin(\theta) & cos(\theta) &  sin(\theta)\\
    sin(\theta) & 1 - cos(\theta) & -sin(\theta) & cos(\theta)
\end{array}\right]
\left[\begin{array}{cccc}
    0.00000\\
    1.00000\\
    0.91295\\
    1.40808
\end{array}\right]
\leadsto
\left[\begin{array}{cccc}
    0.00000\\
    1.00000\\
    0.74511\\
    0.33306
\end{array}\right]
$\\
$
\left[\begin{array}{cccc}
    0 & 0 & 1 & 0\\
    0 & 0 & 0 & 1\\
    -1 & 0 & 2 & 0\\
    0 & -1 & 0 & 2
\end{array}\right]
\left[\begin{array}{cccc}
    0.00000\\
    1.00000\\
    0.74511\\
    0.33306
\end{array}\right]
\leadsto
\left[\begin{array}{cccc}
    0.74511\\
    0.33306\\
    1.49023\\
    -0.33388
\end{array}\right]
$\\
$
\left[\begin{array}{cccc}
    0 & 0 & 1 & 0\\
    0 & 0 & 0 & 1\\
    -1 & 0 & 2 & 0\\
    0 & -1 & 0 & 2
\end{array}\right]
\left[\begin{array}{cccc}
    0.74511\\
    0.33306\\
    1.49023\\
    -0.33388
\end{array}\right]
\leadsto
\left[\begin{array}{cccc}
    1.49023\\
    -0.33388\\
    2.23534\\
    -1.00081
\end{array}\right]
$\\
$
\left[\begin{array}{cccc}
    1 & 0 & 0 & 0\\
    0 & 1 & 0 & 0\\
    1 - cos(\theta) & sin(\theta) & cos(\theta) &  -sin(\theta)\\
    -sin(\theta) & 1 - cos(\theta) & sin(\theta) & cos(\theta)\\
\end{array}\right]
\left[\begin{array}{cccc}
    1.49023\\
    -0.33388\\
    2.23534\\
    -1.00081
\end{array}\right]
\leadsto
\left[\begin{array}{cccc}
    1.49023\\
    -0.33388\\
    2.40317\\
    0.07421
\end{array}\right]
$\\
$
\left[\begin{array}{cccc}
    1 & 0 & 0 & 0\\
    0 & 1 & 0 & 0\\
    1 - cos(\theta) & sin(\theta) & cos(\theta) &  -sin(\theta)\\
    -sin(\theta) & 1 - cos(\theta) & sin(\theta) & cos(\theta)\\
\end{array}\right]
\left[\begin{array}{cccc}
    1.49023\\
    -0.33388\\
    2.40317\\
    0.07421
\end{array}\right]
\leadsto
\left[\begin{array}{cccc}
    1.49023\\
    -0.33388\\
    1.49023\\
    0.66612
\end{array}\right]
$\\
$
\left[\begin{array}{cccc}
    1 & 0 & 0 & 0\\
    0 & 1 & 0 & 0\\
    1 - cos(\theta) & sin(\theta) & cos(\theta) &  -sin(\theta)\\
    -sin(\theta) & 1 - cos(\theta) & sin(\theta) & cos(\theta)\\
\end{array}\right]
\left[\begin{array}{cccc}
    1.49023\\
    -0.33388\\
    1.49023\\
    0.66612
\end{array}\right]
\leadsto
\left[\begin{array}{cccc}
    1.49023\\
    -0.33388\\
    0.57728\\
    0.07421
\end{array}\right]
$\\
$
\left[\begin{array}{cccc}
    0 & 0 & 1 & 0\\
    0 & 0 & 0 & 1\\
    -1 & 0 & 2 & 0\\
    0 & -1 & 0 & 2
\end{array}\right]
\left[\begin{array}{cccc}
    1.49023\\
    -0.33388\\
    0.57728\\
    0.07421
\end{array}\right]
\leadsto
\left[\begin{array}{cccc}
    0.57728\\
    0.07421\\
    -0.33566\\
    0.48229
\end{array}\right]
$\\

Efter alle 9 multiplikationer fra venste ender vi med postionen af spilleren og sidsen svarende til matricen:\\
\[
\left[\begin{array}{cccc}
    0.57728\\
    0.07421\\
    -0.33566\\
    0.48229
\end{array}\right]
\]

\subsection{}
At gange vores 'rotation mod højre' matrice på sig selv svarer til at gange det antal gange med vinkeln $\theta$, da:
\[
R_{\theta 1} \cdot R_{\theta 2} = R_{\theta 1 + \theta 2}
\]
og
\[
(R_\theta)^{n} = \prod_{i=1}^{n} R_{\theta i} = R_{\theta 1 + \theta 2 + \ \dots \ + \theta n }
\]
Vi kan dermed beregne $(R_{20})^{18}$ til:
\[
(R_{20})^{18} = \prod_{i=1}^{18} R_{20} = R_{20 \cdot 18} = R_{360}
\]
Med vores nye vinkel beregnet, kan vi nu indsætte $360$ på $\theta $-s plads i $R_\theta$:
\[
\left[\begin{array}{cccc}
    1 & 0 & 0 & 0 \\
    0 & 1 & 0 & 0 \\
    1 - cos(360) & -sin(360) & cos(360) &  sin(360)\\
    sin(360) & 1 - cos(360) & -sin(360) & cos(360)
\end{array}\right]
=
\left[\begin{array}{cccc}
    1 & 0 & 0 & 0 \\
    0 & 1 & 0 & 0 \\
    1 - 1 & 0 & 1 &  0\\
    0 & 1 - 1 & 0 & 1
\end{array}\right]
=
\left[\begin{array}{cccc}
    1 & 0 & 0 & 0 \\
    0 & 1 & 0 & 0 \\
    0 & 0 & 1 & 0\\
    0 & 0 & 0 & 1
\end{array}\right]
\]
Således ender vi med enhedsmatricen $I_4$.

\section[Opgave]{Opgave}
Se vedhæftede python-fil.



\end{document}