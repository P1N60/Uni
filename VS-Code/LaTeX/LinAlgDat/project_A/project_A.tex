\documentclass[a4paper,12pt]{article}
\usepackage{standalone}
\usepackage{amsmath}
\input{../../sty/setup.sty}

% Changes sections from 1.1 to 1.a
\renewcommand{\thesubsection}{\thesection.\alph{subsection}}
\graphicspath{ {../../pictures/IDMA/IDMA3_a}} 
\title{Københavns Universitet\\
LinAlgDat - Project A}
\author{Victor Vangkilde Jørgensen - kft410\\ 
kft410@alumni.ku.dk\\
Hold 13 Mach}

\begin{document}
\author{Victor Vangkilde Jørgensen - kft410\\ 
kft410@alumni.ku.dk}
\makeatletter
\let\getauthor\@author
\let\gettitle\@title
\makeatother
\maketitle
\thispagestyle{empty}
\n\n
 

\pagebreak
\pagestyle{empty}
\tableofcontents
\pagebreak
\pagestyle{fancy}
\fancyhf{}
\setlength{\headheight}{15.2pt}
\renewcommand{\footrulewidth}{0.4pt}
\fancyhead[R]{\nouppercase \lastrightmark}
\fancyfoot[L]{\gettitle}
\fancyfoot[R]{\thepage}

\maketitle 

\section[Opgave]{Opgave}
\subsection{}
Vi omskriver ligningssystemet til totalmatrix-form:
\[
\left[\begin{array}{ccc|c}
    1 & 2 & 8 & a \\
    a & a & 4a & a \\
    2 & 2 & 2a^2 & 0
\end{array}\right]
\]
Vi benytter Gauss-Jordan elimination til at omskrive totalmatrix'en til en reduceret rækkeeechelonform.\\

$
\left[\begin{array}{ccc|c}
    1 & 2 & 8 & a \\
    a & a & 4a & a \\
    2 & 2 & 2a^2 & 0
\end{array}\right]
\begin{array}{ccc}
    \\
    -ar_1\\
    \\
\end{array}
\leadsto
\left[\begin{array}{ccc|c}
    1 & 2 & 8 & a \\
    0 & -a & -4a & a-a^2 \\
    2 & 2 & 2a^2 & 0
\end{array}\right]
\begin{array}{ccc}
    \\
    \\
    -2r_1\\
\end{array}
\leadsto
$\\
$
\left[\begin{array}{ccc|c}
    1 & 2 & 8 & a \\
    0 & -a & -4a & a-a^2 \\
    0 & -2 & 2a^2-16 & -2a
\end{array}\right]
\begin{array}{ccc}
    \\
    \\
    r_2\cdot \frac{2}{-a}\\
\end{array}
\leadsto
\left[\begin{array}{ccc|c}
    1 & 2 & 8 & a \\
    0 & -a & -4a & a-a^2 \\
    0 & 0 & 2a^2-8 & -2a-\frac{2(a-a^2)}{a} = -2
\end{array}\right]
\begin{array}{ccc}
    r_2\cdot \dfrac{2}{a}\\
    \\
    \\
\end{array}
\leadsto
$\\
$
\left[\begin{array}{ccc|c}
    1 & 0 & 0 & 2-a \\
    0 & -a & -4a & a-a^2 \\
    0 & 0 & 2a^2-8 & -2
\end{array}\right]
\begin{array}{ccc}
    \\
    r_3\cdot\frac{2a}{a^2-4}\\
    \\
\end{array}
\leadsto
\left[\begin{array}{ccc|c}
    1 & 0 & 0 & 2-a \\
    0 & -a & 0 & a - a^2 -\frac{4a}{(a^2 - 4)} \\
    0 & 0 & 2a^2-8 & -2
\end{array}\right]
\begin{array}{ccc}
    \\
    \\
    \cdot \frac{1}{2a^2-8}\\
\end{array}
\leadsto
$\\
$
\left[\begin{array}{ccc|c}
    1 & 0 & 0 & 2-a \\
    0 & -a & 0 & a - a^2 -\frac{4a}{(a^2 - 4)} \\
    0 & 0 & 1 & -\frac{2}{2a^2-8} = -\frac{2}{2(a^2-4)} = \frac{1}{(4-a^2)}
\end{array}\right]
\begin{array}{ccc}
    \\
    \cdot \frac{1}{-a}\\
    \\
\end{array}
\leadsto
\left[\begin{array}{ccc|c}
    1 & 0 & 0 & 2-a \\
    0 & 1 & 0 & \frac{(a^3 - a^2 - 4a + 8)}{(a^2 - 4)} \\
    0 & 0 & 1 & \frac{1}{(4-a^2)}
\end{array}\right]
\square
$\\

Vi har nu fået den løsning vi ledte efter, så vi er dermed færdige.

\subsection{}
Vi opskriver igen vores ligningssystem som en totalmatrix, og erstatter denne gang $a$ med 0:\\
$
\left[\begin{array}{ccc|c}
    1 & 2 & 8 & 0 \\
    0 & 0 & 4\cdot0 & 0 \\
    2 & 2 & 2\cdot0^2 & 0
\end{array}\right]
reducer
\leadsto
\left[\begin{array}{ccc|c}
    1 & 2 & 8 & 0 \\
    0 & 0 & 0 & 0 \\
    2 & 2 & 0 & 0
\end{array}\right]
\begin{array}{cccc}
    \\
    \\
    > byttes\\
    \\
\end{array}
\leadsto
\left[\begin{array}{ccc|c}
    1 & 2 & 8 & 0 \\
    2 & 2 & 0 & 0 \\
    0 & 0 & 0 & 0
\end{array}\right]
\begin{array}{ccc}
    \\
    -2r_1\\
    \\
\end{array}
\leadsto
$\\
$
\left[\begin{array}{ccc|c}
    1 & 2 & 8 & 0 \\
    0 & -2 & -16 & 0 \\
    0 & 0 & 0 & 0
\end{array}\right]
\cdot (-\frac{1}{2})
\leadsto
\left[\begin{array}{ccc|c}
    1 & 2 & 8 & 0 \\
    0 & 1 & 8 & 0 \\
    0 & 0 & 0 & 0
\end{array}\right]
\begin{array}{ccc}
    -2r_1\\
    \\
    \\
\end{array}
\leadsto
\left[\begin{array}{ccc|c}
    1 & 0 & -8 & 0 \\
    0 & 1 & 8 & 0 \\
    0 & 0 & 0 & 0
\end{array}\right]
$\\

Vi kan nu aflæse løsningerne til:
\[
\left[\begin{array}{c}
    x_1 \\
    x_2 \\
    x_3
\end{array}\right]
=
t
\left[\begin{array}{c}
    8 \\
    -8 \\
    1
\end{array}\right]
\]
Vi ser nu, hvad vi får, når vi bruger den rækkereducerede totalmatrix fra tidligere, når vi erstatter $a$ med 0:\\

$
\left[\begin{array}{ccc|c}
    1 & 0 & 0 & 2-a \\
    0 & 1 & 0 & \frac{(a^3 - a^2 - 4a + 8)}{(a^2 - 4)} \\
    0 & 0 & 1 & \frac{1}{(4-a^2)}
\end{array}\right]
a=0
\leadsto
\left[\begin{array}{ccc|c}
    1 & 0 & 0 & 2-0 \\
    0 & 1 & 0 & \frac{(0^3 - 0^2 - 4\cdot0 + 8)}{(0^2 - 4)} \\
    0 & 0 & 1 & \frac{1}{(4-0^2)}
\end{array}\right]
reducer
\leadsto
\left[\begin{array}{ccc|c}
    1 & 0 & 0 & 2 \\
    0 & 1 & 0 & -2 \\
    0 & 0 & 1 & \frac{1}{4}
\end{array}\right]
$\\

Denne matrix antyder en unik løsning, hvilket ikke afspejler hvad vi fandt lige før, hvor vi fandt uendelig mange løsninger. Dette skete sandstnligvis fordi at den rækkereducerede totalmatrix er lavet ud fra antagelsen, at $a \neq 0$.

\subsection{}
Vi opskriver igen vores ligningssystem som en totalmatrix, og erstatter denne gang $a$ med 2:\\

$
\left[\begin{array}{ccc|c}
    1 & 2 & 8 & 2 \\
    0 & 0 & 4\cdot2 & 2 \\
    2 & 2 & 2\cdot2^2 & 0
\end{array}\right]
reducer
\leadsto
\left[\begin{array}{ccc|c}
    1 & 2 & 8 & 2 \\
    0 & 0 & 8 & 2 \\
    2 & 2 & 8 & 0
\end{array}\right]
\begin{array}{ccc}
    \\
    \\
    +r_2\\
\end{array}
\leadsto
\left[\begin{array}{ccc|c}
    1 & 2 & 8 & 2 \\
    2 & 2 & 8 & 0 \\
    0 & 0 & 8 & 2
\end{array}\right]
\begin{array}{ccc}
    \\
    -2r_1\\
    \\
\end{array}
\leadsto
$

$
\left[\begin{array}{ccc|c}
    1 & 2 & 8 & 2 \\
    0 & -2 & -8 & -4 \\
    0 & 0 & 8 & 2
\end{array}\right]
\begin{array}{ccc}
    -r_3\\
    \\
    \\
\end{array}
\leadsto
\left[\begin{array}{ccc|c}
    1 & 2 & 0 & 0 \\
    0 & -2 & -8 & -4 \\
    0 & 0 & 8 & 2
\end{array}\right]
\begin{array}{ccc}
    \\
    +r_3\\
    \\
\end{array}
\leadsto
\left[\begin{array}{ccc|c}
    1 & 2 & 0 & 0 \\
    0 & -2 & 0 & -2 \\
    0 & 0 & 8 & 2
\end{array}\right]
\begin{array}{ccc}
    +r_2\\
    \\
    \\
\end{array}
\leadsto
$

$
\left[\begin{array}{ccc|c}
    1 & 0 & 0 & -2 \\
    0 & -2 & 0 & -2 \\
    0 & 0 & 8 & 2
\end{array}\right]
\begin{array}{ccc}
    \\
    \cdot(-\frac{1}{2})\\
    \\
\end{array}
\leadsto
\left[\begin{array}{ccc|c}
    1 & 0 & 0 & -2 \\
    0 & 1 & 0 & 1 \\
    0 & 0 & 8 & 2
\end{array}\right]
\begin{array}{ccc}
    \\
    \cdot\frac{1}{8}\\
    \\
\end{array}
\leadsto
\left[\begin{array}{ccc|c}
    1 & 0 & 0 & -2 \\
    0 & 1 & 0 & 1 \\
    0 & 0 & 1 & \frac{1}{4}
\end{array}\right]
$\\

Vi kan nu aflæse løsningen til:
\[
\left[\begin{array}{c}
    x_1\\
    x_2\\
    x_3
\end{array}\right]
=
\left[\begin{array}{c}
    -2\\
    1\\
    \frac{1}{4}
\end{array}\right]
\]\\

Lad os nu se, hvad vi får, når vi bruger den rækkereducerede totalmatrix fra tidligere, når vi erstatter $a$ med 2:\\

$
\left[\begin{array}{ccc|c}
    1 & 0 & 0 & 2-2 \\
    0 & 1 & 0 & \frac{(2^3 - 2^2 - 4\cdot 2 + 8)}{(2^2 - 4)} \\
    0 & 0 & 1 & \frac{1}{(4-2^2)}
\end{array}\right]
\begin{array}{ccc}
    \\
    reducer\\
    \\
\end{array}
\leadsto
\left[\begin{array}{ccc|c}
    1 & 0 & 0 & 0 \\
    0 & 1 & 0 & \frac{4}{0} \\
    0 & 0 & 1 & \frac{1}{0}
\end{array}\right]
$\\

Vi ser nu, at vi får division med 0, hvilket må betyde, at der ikke er nogen løsninger, når $a=2$.\\

\subsection{}
Vi omskriver ligningssystemet til koefficientmatrice-form på venstre side, hvor $a=1$, og sættr identitetmatricen på højre side:\\

$
\left[\begin{array}{ccc|ccc}
    1 & 2 & 8 & 1 & 0 & 0 \\
    1 & 1 & 4\cdot 1 & 0 & 1 & 0 \\
    2 & 2 & 2\cdot 1^2 & 0 & 0 & 1
\end{array}\right]
reducer
\leadsto
\left[\begin{array}{ccc|ccc}
    1 & 2 & 8 & 1 & 0 & 0 \\
    1 & 1 & 4 & 0 & 1 & 0 \\
    2 & 2 & 2 & 0 & 0 & 1
\end{array}\right]
\begin{array}{ccc}
    \\
    -1r_1\\
    \\
\end{array}
\leadsto
$\\
$
\left[\begin{array}{ccc|ccc}
    1 & 2 & 8 & 1 & 0 & 0 \\
    0 & -1 & -4 & -1 & 1 & 0 \\
    2 & 2 & 2 & 0 & 0 & 1
\end{array}\right]
\begin{array}{ccc}
    \\
    \\
    -2r_1\\
\end{array}
\leadsto
\left[\begin{array}{ccc|ccc}
    1 & 2 & 8 & 1 & 0 & 0 \\
    0 & -1 & -4 & -1 & 1 & 0 \\
    0 & -2 & -14 & -2 & 0 & 1
\end{array}\right]
\begin{array}{ccc}
    \\
    \\
    -2r_2\\
\end{array}
\leadsto
$\\
$
\left[\begin{array}{ccc|ccc}
    1 & 2 & 8 & 1 & 0 & 0 \\
    0 & -1 & -4 & -1 & 1 & 0 \\
    0 & 0 & -6 & 0 & -2 & 1
\end{array}\right]
\begin{array}{ccc}
    \cdot 3\\
    \cdot (-3)\\
    \\
\end{array}
\leadsto
\left[\begin{array}{ccc|ccc}
    3 & 6 & 24 & 3 & 0 & 0 \\
    0 & 3 & 12 & 3 & -3 & 0 \\
    0 & 0 & -6 & 0 & -2 & 1
\end{array}\right]
\begin{array}{ccc}
    +4r_3\\
    \\
    \\
\end{array}
\leadsto
$\\
$
\left[\begin{array}{ccc|ccc}
    3 & 6 & 0 & 3 & -8 & 4 \\
    0 & 3 & 12 & 3 & -3 & 0 \\
    0 & 0 & -6 & 0 & -2 & 1
\end{array}\right]
\begin{array}{ccc}
    \\
    +2r_3\\
    \\
\end{array}
\leadsto
\left[\begin{array}{ccc|ccc}
    3 & 6 & 0 & 3 & -8 & 4 \\
    0 & 3 & 0 & 3 & -7 & 2 \\
    0 & 0 & -6 & 0 & -2 & 1
\end{array}\right]
\begin{array}{ccc}
    -2r_2\\
    \\
    \\
\end{array}
\leadsto
$\\
$
\left[\begin{array}{ccc|ccc}
    3 & 0 & 0 & -3 & 6 & 0 \\
    0 & 3 & 0 & 3 & -7 & 2 \\
    0 & 0 & -6 & 0 & -2 & 1
\end{array}\right]
\begin{array}{ccc}
    \cdot \frac{1}{3}\\
    \cdot \frac{1}{3}\\
    \cdot (-\frac{1}{6})\\
\end{array}
\leadsto
\left[\begin{array}{ccc|ccc}
    1 & 0 & 0 & -1 & 2 & 0 \\
    0 & 1 & 0 & 1 & -\frac{7}{3} & \frac{2}{3} \\
    0 & 0 & 1 & 0 & \frac{1}{3} & -\frac{1}{6}
\end{array}\right]
$


\section[Opgave]{Opgave}
\subsection{}


\subsection{}


\subsection{}


\section[Opgave]{Opgave}
\subsection{}
Vi sætter et 1-tal i hver $e_{uv}$-element i matricen for hver edge $E(u,v)$ i grafen $G_N$:\\

\[
N = 
\left[\begin{array}{ccccc}
    0 & 1 & 1 & 0 & 0 \\
    1 & 0 & 0 & 0 & 1 \\
    1 & 0 & 0 & 0 & 0 \\
    1 & 1 & 1 & 0 & 1 \\
    1 & 0 & 0 & 1 & 0 
\end{array}\right]
\]\\

For at finde antallet af veje fra knude 4 til knude 1 med netop længde 8, skal vi kigge på element $(4,1)$ i $N^8$. Vi kan finde $(N^8)_{4,1}$ ved:\\
\[
69 \cdot 0 + 45 \cdot 1 + 45 \cdot 1 + 18 \cdot 1 + 29 \cdot 1 = 137
\]
unikke veje fra knude 4 til knude 1 med længde 8.\\

\subsection{}
Linkmatricen $A$ findes ved at tage $N^T$ og dividere hvert element med antallet af udgående edges i dens column.\\
\[
N^T =
\left[\begin{array}{ccccc}
    0 & 1 & 1 & 1 & 1 \\
    1 & 0 & 0 & 1 & 0 \\
    1 & 0 & 0 & 1 & 0 \\
    0 & 0 & 0 & 0 & 1 \\
    0 & 1 & 0 & 1 & 0
\end{array}\right]
\]


\[
A =
\left[\begin{array}{ccccc}
    0 & \frac{1}{2} & 1 & \frac{1}{4} & \frac{1}{2} \\
    \frac{1}{2} & 0 & 0 & \frac{1}{4} & 0 \\
    \frac{1}{2} & 0 & 0 & \frac{1}{4} & 0 \\
    0 & 0 & 0 & 0 & \frac{1}{2} \\
    0 & \frac{1}{2} & 0 & \frac{1}{4} & 0
\end{array}\right]
\]

\subsection{}
Vi opskriver vores ligningssystem som en totalmatrix:\\
$
\left[\begin{array}{ccccc|c}
    -1 & \frac{1}{2} & 1 & 0 & 0 & 0 \\
    \frac{1}{2} & -1 & 0 & 0 & \frac{1}{2} & 0 \\
    \frac{1}{2} & 0 & -1 & 0 & 0 & 0 \\
    \frac{1}{2} & \frac{1}{2} & 1 & -1 & \frac{1}{2} & 0 \\
    \frac{1}{2} & 0 & 0 & \frac{1}{4} & -1 & 0
\end{array}\right]
\begin{array}{ccccc}
    r_1 \cdot 2\\
    r_2 \cdot 4\\
    r_3 \cdot 4\\
    r_4 \cdot 4\\
    r_5 \cdot 4\\
\end{array}
\leadsto
\left[\begin{array}{ccccc|c}
    -2 & 1 & 2 & 0 & 0 & 0 \\
    2 & -4 & 0 & 0 & 2 & 0 \\
    2 & 0 & -4 & 0 & 0 & 0 \\
    2 & 2 & 4 & -4 & 2 & 0 \\
    2 & 0 & 0 & 1 & -4 & 0
\end{array}\right]
\begin{array}{ccccc}
    \\
    +r_1\\
    +r_1\\
    +r_1\\
    +r_1\\
\end{array}
\leadsto
$\\
$
\left[\begin{array}{ccccc|c}
    -2 & 1 & 2 & 0 & 0 & 0 \\
    0 & -3 & 2 & 0 & 2 & 0 \\
    0 & 1 & -2 & 0 & 0 & 0 \\
    0 & 3 & 6 & -4 & 2 & 0 \\
    0 & 1 & 2 & 1 & -4 & 0
\end{array}\right]
\begin{array}{ccccc}
    \\
    \\
    r_3 \cdot 3\\
    \\
    r_5 \cdot 3\\
\end{array}
\leadsto
\left[\begin{array}{ccccc|c}
    -2 & 1 & 2 & 0 & 0 & 0 \\
    0 & -3 & 2 & 0 & 2 & 0 \\
    0 & 3 & -6 & 0 & 0 & 0 \\
    0 & 3 & 6 & -4 & 2 & 0 \\
    0 & 3 & 6 & 3 & -12 & 0
\end{array}\right]
\begin{array}{ccccc}
    \\
    \\
    +r_2\\
    +r_2\\
    +r_2
\end{array}
\leadsto
$\\
$
\left[\begin{array}{ccccc|c}
    -2 & 1 & 2 & 0 & 0 & 0 \\
    0 & -3 & 2 & 0 & 2 & 0 \\
    0 & 0 & -4 & 0 & 2 & 0 \\
    0 & 0 & 8 & -4 & 4 & 0 \\
    0 & 0 & 8 & 3 & -10 & 0
\end{array}\right]
\begin{array}{ccccc}
    \cdot 2
\end{array}
\leadsto
\left[\begin{array}{ccccc|c}
    -2 & 1 & 2 & 0 & 0 & 0 \\
    0 & -3 & 2 & 0 & 2 & 0 \\
    0 & 0 & -8 & 0 & 4 & 0 \\
    0 & 0 & 8 & -4 & 4 & 0 \\
    0 & 0 & 8 & 3 & -10 & 0
\end{array}\right]
\begin{array}{ccccc}
    \\
    \\
    \\
    +r_3\\
    +r_3
\end{array}
\leadsto
$\\
$
\left[\begin{array}{ccccc|c}
    -2 & 1 & 2 & 0 & 0 & 0 \\
    0 & -3 & 2 & 0 & 2 & 0 \\
    0 & 0 & -8 & 0 & 4 & 0 \\
    0 & 0 & 0 & -4 & 8 & 0 \\
    0 & 0 & 0 & 3 & -6 & 0
\end{array}\right]
\begin{array}{ccccc}
    \\
    \\
    \\
    \cdot 3\\
    \cdot 4
\end{array}
\leadsto
\left[\begin{array}{ccccc|c}
    -2 & 1 & 2 & 0 & 0 & 0 \\
    0 & -3 & 2 & 0 & 2 & 0 \\
    0 & 0 & -8 & 0 & 4 & 0 \\
    0 & 0 & 0 & -12 & 24 & 0 \\
    0 & 0 & 0 & 12 & -24 & 0
\end{array}\right]
\begin{array}{ccccc}
    \\
    \\
    \\
    \\
    +r_4
\end{array}
\leadsto
$\\
$
\left[\begin{array}{ccccc|c}
    -2 & 1 & 2 & 0 & 0 & 0 \\
    0 & -3 & 2 & 0 & 2 & 0 \\
    0 & 0 & -8 & 0 & 4 & 0 \\
    0 & 0 & 0 & -12 & 24 & 0 \\
    0 & 0 & 0 & 0 & 0 & 0
\end{array}\right]
\begin{array}{ccccc}
    \cdot 4\\
    \cdot 4\\
    \\
    \\
    \\
\end{array}
\leadsto
\left[\begin{array}{ccccc|c}
    -8 & 4 & 8 & 0 & 0 & 0 \\
    0 & -12 & 8 & 0 & 8 & 0 \\
    0 & 0 & -8 & 0 & 4 & 0 \\
    0 & 0 & 0 & -12 & 24 & 0 \\
    0 & 0 & 0 & 0 & 0 & 0
\end{array}\right]
\begin{array}{ccccc}
    +r_3\\
    +r_3\\
    \\
    \\
    \\
\end{array}
\leadsto
$\\
$
\left[\begin{array}{ccccc|c}
    -8 & 4 & 0 & 0 & 4 & 0 \\
    0 & -12 & 0 & 0 & 12 & 0 \\
    0 & 0 & -8 & 0 & 4 & 0 \\
    0 & 0 & 0 & -12 & 24 & 0 \\
    0 & 0 & 0 & 0 & 0 & 0
\end{array}\right]
\begin{array}{ccccc}
    \cdot 3\\
    \\
    \\
    \\
    \\
\end{array}
\leadsto
\left[\begin{array}{ccccc|c}
    -24 & 12 & 0 & 0 & 12 & 0 \\
    0 & -12 & 0 & 0 & 12 & 0 \\
    0 & 0 & -8 & 0 & 4 & 0 \\
    0 & 0 & 0 & -12 & 24 & 0 \\
    0 & 0 & 0 & 0 & 0 & 0
\end{array}\right]
\begin{array}{ccccc}
    +r_2\\
    \\
    \\
    \\
    \\
\end{array}
\leadsto
$\\
$
\left[\begin{array}{ccccc|c}
    -24 & 0 & 0 & 0 & 24 & 0 \\
    0 & -12 & 0 & 0 & 12 & 0 \\
    0 & 0 & -8 & 0 & 4 & 0 \\
    0 & 0 & 0 & -12 & 24 & 0 \\
    0 & 0 & 0 & 0 & 0 & 0
\end{array}\right]
\begin{array}{ccccc}
    \cdot (-\frac{1}{24})\\
    \cdot (-\frac{1}{12})\\
    \cdot (-\frac{1}{8})\\
    \cdot (-\frac{1}{12})\\
    \\
\end{array}
\leadsto
\left[\begin{array}{ccccc|c}
    1 & 0 & 0 & 0 & -1 & 0 \\
    0 & 1 & 0 & 0 & -1 & 0 \\
    0 & 0 & 1 & 0 & -\frac{1}{2} & 0 \\
    0 & 0 & 0 & 1 & -2 & 0 \\
    0 & 0 & 0 & 0 & 0 & 0
\end{array}\right]
$\\

Vi kan nu aflæse løsningerne til:
\[
\left[\begin{array}{c}
    x_1\\
    x_2\\
    x_3\\
    x_4\\
    x_5\\
\end{array}\right]
=
t
\left[\begin{array}{c}
    1\\
    1\\
    \frac{1}{2}\\
    2\\
    0\\
\end{array}\right]
\]

\section[Opgave]{Opgave}
Se vedhæftede python-fil.

\end{document}