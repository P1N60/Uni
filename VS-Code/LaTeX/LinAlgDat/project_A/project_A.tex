\documentclass[a4paper,12pt]{article}
\usepackage{standalone}
\usepackage{amsmath}



% Basic
\usepackage[T1]{fontenc}
\usepackage[utf8]{inputenc}
\usepackage{titlesec}
\titleformat{\section}
  {\normalfont\fontsize{14}{15}\bfseries}{\thesection}{1em}{}
\titleformat{\subsection}
  {\normalfont\fontsize{12}{15}\bfseries}{\thesubsection}{1em}{}

% Changes sections from 1.1 to 1.a
\renewcommand{\thesubsection}{\thesection.\alph{subsection}}

% ------------------------------------------------------------ %

% Packages
\RequirePackage{tcolorbox}
\usepackage{amsmath, amsthm, amssymb}
\usepackage{blindtext}
\usepackage{enumitem}
\usepackage{extramarks}
\usepackage{fancyhdr}
\usepackage[margin=1in]{geometry}
\usepackage{graphicx}
\usepackage{hyperref}
\usepackage{indentfirst}
\usepackage{listings}
\usepackage{mathrsfs}
\usepackage{mdframed}
\usepackage{multicol, multirow}
\usepackage{needspace, setspace}
\usepackage{paracol}
\usepackage{pgf, pgfplots}
\usepackage{tikz}
\usetikzlibrary{patterns}
\usepackage{silence}
\usepackage{xcolor}
\usepackage{bookmark}
\setlength{\parindent}{0pt}
\usetikzlibrary{patterns}
% \usepackage{subcaption}
\usetikzlibrary{decorations.pathreplacing}
\usepackage{caption}
\usepackage{subcaption}
\usepackage{xcolor}
\definecolor{maroon}{RGB}{128, 0, 0}
% ------------------------------------------------------------ %

% Spacings
\newcommand{\n}{\vspace{3mm}} % Context spacing
\newcommand{\s}{\vspace{1mm}} % Equation spacing
\newcommand{\m}{\vspace{-3mm}} % Reverse context spacing
\newcommand{\propdisp}{\pagebreak} % Proper display page break
\newcommand{\br}{\n\\\n}

% Wordings
\newcommand{\ans}[1][zb]{{\color{#1}\textit{Answer. }\hspace{3mm}}} % Answer
\newcommand{\arl}[1][zr]{{\color{#1}$\brr{\Leftarrow}$\hspace{3mm}}} % Left arrow
\newcommand{\arr}[1][zr]{{\color{#1}$\brr{\Rightarrow}$\hspace{3mm}}} % Right arrow
\newcommand{\cse}[2][zr]{{\color{#1}\textit{Case #2: }\hspace{1mm}}} % Case
\newcommand{\clm}[2][zr]{{\color{#1}$\vdash_{#2}$\hspace{1mm}}} % Claim
\newcommand{\prf}[1][zr]{{\color{#1}\textit{Proof. }\hspace{3mm}}} % Proof
\newcommand{\prt}[2][a]{\hspace{-2mm}{\color{#2}\textit{Part (#1) }}\hspace{1mm}} % Part
\newcommand{\prtc}[2][a]{\hspace{2mm}\prt[#1]{#2}} % Continued part

\newcommand{\rdft}[1][\sct]{{\color{zg}\textit{Definition #1}}} % Refer definition
\newcommand{\rexm}[1][\sct]{{\color{zb}\textit{Example #1}}} % Refer example
\newcommand{\rfig}[1][\sct]{{\color{zy}\textit{Figure #1}}} % Refer figure
\newcommand{\rpst}[1][\sct]{{\color{zr}\textit{Proposition #1}}} % Refer proposition
\newcommand{\rthm}[1][\sct]{{\color{zr}\textit{Theorem #1}}} % Refer theorem

\newcommand{\sct}{\thesection.\thescount} % Counter
\newcommand{\sctr}[2][0]{\the\numexpr\value{section}-#1\relax.\the\numexpr\value{scount}-#2\relax} % Relative counter

% Equations
\newcommand{\C}{\mathbb{C}} % Complex
\newcommand{\F}{\mathbb{F}} % Field
\newcommand{\I}{\mathbb{I}} % Irrational
\newcommand{\N}{\mathbb{N}} % Natural
\newcommand{\Q}{\mathbb{Q}} % Rational
\newcommand{\R}{\mathbb{R}} % Real
\newcommand{\Z}{\mathbb{Z}} % Integer

\newcommand{\GL}{\mathrm{GL}} % General linear group
\newcommand{\SL}{\mathrm{SL}} % Special linear group

\newcommand{\abs}[1]{\left| #1\right|} % Absolute
\newcommand{\bra}[1]{\left\langle #1\right\rangle} % Angled brackets
\newcommand{\brc}[1]{\left\{ #1\right\}} % Curly brackets
\newcommand{\brr}[1]{\left( #1\right)} % Round brackets
\newcommand{\brs}[1]{\left[ #1\right]} % Square brackets
\newcommand{\cond}[1]{\left. #1\right|} % Condition with right bar
\newcommand{\diff}{\,\mathrm{d}} % Differential
\newcommand{\erm}[1]{\;\;\;\;\text{#1}} % Equation remarks
\newcommand{\nrm}[1]{\left\| #1\right\|} % Norm
\newcommand{\srm}{\,\mid\,} % Set remarks

% Math operators
\let\Im\relax
\let\Re\relax

\DeclareMathOperator{\Im}{Im} % Imaginary function
\DeclareMathOperator{\Re}{Re} % Real function

% ------------------------------------------------------------ %

% Colours
\definecolor{gr}{RGB}{120, 120, 120} % Grey
\definecolor{zb}{RGB}{0, 38, 77} % Blue
\definecolor{zg}{RGB}{0, 77, 51} % Green
\definecolor{zp}{RGB}{51, 0, 77} % Purple
\definecolor{zr}{RGB}{77, 0, 38} % Red
\definecolor{zy}{RGB}{77, 64, 0} % Yellow

% Graphing
\usetikzlibrary{arrows}
\usetikzlibrary{calc}
\usetikzlibrary{patterns}
\pgfplotsset{compat=1.15}

% ------------------------------------------------------------ %

% Remark
\newcommand{\remark}[1]{
  \noindent\textbf{Remarks}
  
  \begin{nlist}
    \item Context of this document is based on university course \textit{\gettitle} from \textit{Department of Mathematics, The Chinese University of Hong Kong}. The original source can be found at \url{https://www.math.cuhk.edu.hk/course}. The author does not own the source.
    \item This document is assumed unavailable for unauthorized parties that have not attended the university course. It is prohibited to share, including distributing or copying this document to unauthorized parties in any means for any non-academic purpose.
    \item Context of this doucment may not be completely accurate. The author assumes no responsibility or liability for any errors or omissions in the context of this document.
    \item This document is under license CC-BY-SA 4.0. It is allowed to make any editions on this document, as long as terms of the license is not violated.
    #1
  \end{nlist}
}

% Prerequisites
\newenvironment{prereq}{
  \noindent \textbf{Prerequisites}\n

  This course requires prerequisites of
}{
  \n
}

% Reference list
\newenvironment{reflist}{
  \begin{alist}
    \item Course material from various professors associated to \textit{\gettitle}
}{
  \end{alist}
}

% ------------------------------------------------------------ %

% Environments
\newcounter{scount}[section] % Counter

\newenvironment{crl}{ % Corollary
  \parindent 0pt
  \begin{siderule}[linecolor=zr]{\color{zr}\textit{Corollary. }}
}{
  \end{siderule}
}

\newenvironment{cmt}{ % Comment
  \parindent 0pt
  \begin{siderule}[linecolor=zp]{\color{zp}\textit{Comment. }}
}{
  \end{siderule}
}

\newenvironment{dft}{ % Definition
  \parindent 0pt
  \refstepcounter{scount}
  \begin{siderule}[linecolor=zg]{\color{zg}\textit{Definition \sct. }}
}{
  \end{siderule}
}
\newenvironment{lem}{ % Lemma
  \parindent 0pt
  \refstepcounter{scount}
  \begin{siderule}[linecolor=zg]{\color{zg}\textit{Lemma \sct. }}
}{
  \end{siderule}
}

\newenvironment{exm}{ % Example
  \parindent 0pt
  \refstepcounter{scount}
  \begin{siderule}[linecolor=zb]{\color{zb}\textit{Example \sct. }}
}{
  \end{siderule}
}

\newenvironment{fig}{ % Figure
  \parindent 0pt
  \refstepcounter{scount}
  \begin{siderule}[linecolor=zy]{\color{zy}\textit{Figure \sct. }}\n
  
}{
  \end{siderule}
}

\newenvironment{prv}{ % Proof
  \parindent 0pt
  \begin{siderule}[linecolor=zr]\prf
}{
  \end{siderule}
}

\newenvironment{pst}{ % Proposition
  \parindent 0pt
  \refstepcounter{scount}
  \begin{siderule}[linecolor=zr]{\color{zr}\textit{Proposition \sct. }}
}{
  \end{siderule}
}

\newenvironment{tcn}{ % Technique
  \parindent 0pt
  \begin{siderule}[linecolor=zp]{\color{zp}\textit{Technique. }}
}{
  \end{siderule}
}

\newenvironment{thm}{ % Theorem
  \parindent 0pt
  \refstepcounter{scount}
  \begin{siderule}[linecolor=zr]{\color{zr}\textit{Sætning \sct. }}
}{
  \end{siderule}
}

\newenvironment{rmatrix}{ % Matrix in round brackets
  \left(\begin{matrix}
}{
  \end{matrix}\right)
}

\newenvironment{alist}{ % Alphabetical list
  \begin{enumerate}[label=(\alph*)]
}{
  \end{enumerate}
}

\newenvironment{Alist}{ % Capitalized alphabetical list
  \begin{enumerate}[label=(\Alph*)]
}{
  \end{enumerate}
}

\newenvironment{nlist}{ % Number list
  \begin{enumerate}[label=(\arabic*)]
}{
  \end{enumerate}
}

\newenvironment{plist}{ % Point list
  \begin{itemize}
}{
  \end{itemize}
}

\newenvironment{rlist}{ % Roman list
  \begin{enumerate}[label=(\roman*)]
}{
  \end{enumerate}
}

\newmdenv[ % Siderule line
  topline=false,
  bottomline=false,
  rightline=false,
  rightmargin=0
]{siderule}


% ------------------------------------------------------------ %
% Warning filters
\WarningFilter{mdframed}{You got a bad break}
\hfuzz=8pt


% Load required packages
\usepackage{tcolorbox}
% ------------------------------------------------------------ %

% Code

\usepackage{amsmath}
\usepackage{graphicx}
\definecolor{bluekeywords}{rgb}{0.13,0.13,1}
\definecolor{greencomments}{rgb}{0,0.5,0}
\definecolor{redstrings}{rgb}{0.9,0,0}
\definecolor{bgcolor}{rgb}{0.95,0.94,0.94}
\usepackage{listings}
\usepackage{upquote}
\usepackage{xcolor}

\lstdefinelanguage{Python}
{
  keywords={typeof, null, catch, switch, in, int, str, float, self},
  keywordstyle=\color{ForestGreen}\bfseries,
  ndkeywords={boolean, throw, import},
  ndkeywords={return, class, if ,elif, endif, while, do, else, True, False , catch, def},
  ndkeywordstyle=\color{BrickRed}\bfseries,
  identifierstyle=\color{black},
  sensitive=false,
  comment=[l]{\#},
  morecomment=[s]{/*}{*/},
  commentstyle=\color{purple}\ttfamily,
  stringstyle=\color{red}\ttfamily,
}

\lstset
{ %Formatting for code in appendix
    language=Python,
    numbers=left,
    stepnumber=1,
    showstringspaces=false,
    tabsize=1,
    breaklines=true,
    breakatwhitespace=false,
    backgroundcolor=\color{bgcolor},  % Background color
    basicstyle=\ttfamily\footnotesize, % Code font size and style
    frame=single,                    % Adds a frame around the code
    rulecolor=\color{bgcolor},       % Frame color
    breaklines=true,                 % Breaks long lines
}


% Changes sections from 1.1 to 1.a
\renewcommand{\thesubsection}{\thesection.\alph{subsection}}
\graphicspath{ {../../pictures/IDMA/IDMA3_a}} 
\title{Københavns Universitet\\
LinAlgDat - Project A}
\author{Victor Vangkilde Jørgensen - kft410\\ 
kft410@alumni.ku.dk\\
Hold 13 Mach}

\begin{document}
\author{Victor Vangkilde Jørgensen}
\makeatletter
\let\getauthor\@author
\let\gettitle\@title
\makeatother
\maketitle
\thispagestyle{empty}
\n\n
 

\pagebreak
\pagestyle{empty}
\tableofcontents
\pagebreak
\pagestyle{fancy}
\fancyhf{}
\setlength{\headheight}{15.2pt}
\renewcommand{\footrulewidth}{0.4pt}
% \fancyhead[R]{\nouppercase \lastrightmark}
\fancyfoot[L]{\gettitle}
\fancyfoot[R]{\thepage}

\maketitle 

\section[Opgave]{Opgave}
\subsection{}
Vi omskriver ligningssystemet til totalmatrix-form:
\[
\left[\begin{array}{ccc|c}
    1 & 2 & 8 & a \\
    a & a & 4a & a \\
    2 & 2 & 2a^2 & 0
\end{array}\right]
\]
Vi benytter Gauss-Jordan elimination til at omskrive totalmatrix'en til en reduceret rækkeeechelonform.\\
Først vælger vi, at tilføje $-ar_1$ til $r_2$: 
\[
\left[\begin{array}{ccc|c}
    1 & 2 & 8 & a \\
    0 & -a & -4a & a-a^2 \\
    2 & 2 & 2a^2 & 0
\end{array}\right]
\]
Herefter tilføjer vi $-2r_1$ til $r_3$:
\[
\left[\begin{array}{ccc|c}
    1 & 2 & 8 & a \\
    0 & -a & -4a & a-a^2 \\
    0 & -2 & 2a^2-16 & -2a
\end{array}\right]
\]
Vi tilføjer $\dfrac{2r_2}{-a}$ til $r_3$:
\[
\left[\begin{array}{ccc|c}
    1 & 2 & 8 & a \\
    0 & -a & -4a & a-a^2 \\
    0 & 0 & 2a^2-8 & -2a-\frac{2(a-a^2)}{a} = -2
\end{array}\right]
\]
Vi tilføjer $\dfrac{2r_2}{a}$ til $r_1$:
\[
\left[\begin{array}{ccc|c}
    1 & 0 & 0 & 2-a \\
    0 & -a & -4a & a-a^2 \\
    0 & 0 & 2a^2-8 & -2
\end{array}\right]
\]
Vi tilføjer $\dfrac{2ar_3}{a^2-4}$ til $r_2$:
\[
\left[\begin{array}{ccc|c}
    1 & 0 & 0 & 2-a \\
    0 & -a & 0 & a - a^2 -\frac{4a}{(a^2 - 4)} \\
    0 & 0 & 2a^2-8 & -2
\end{array}\right]
\]
Vi dividerer $r_3$ med $2a^2-8$:
\[
\left[\begin{array}{ccc|c}
    1 & 0 & 0 & 2-a \\
    0 & -a & 0 & a - a^2 -\frac{4a}{(a^2 - 4)} \\
    0 & 0 & 1 & -\frac{2}{2a^2-8} = -\frac{2}{2(a^2-4)} = \frac{1}{(4-a^2)}
\end{array}\right]
\]
Til sidst dividerer vi $r_2$ med $-a$:
\[
\left[\begin{array}{ccc|c}
    1 & 0 & 0 & 2-a \\
    0 & 1 & 0 & \frac{(a^3 - a^2 - 4a + 8)}{(a^2 - 4)} \\
    0 & 0 & 1 & \frac{1}{(4-a^2)}
\end{array}\right]
\square
\]
Vi har nu fået den løsning vi ledte efter, så vi er dermed færdige.

\subsection{}
Vi opskriver igen vores ligningssystem som en totalmatrix, og erstatter denne gang $a$ med 0:
\[
\left[\begin{array}{ccc|c}
    1 & 2 & 8 & 0 \\
    0 & 0 & 4\cdot0 & 0 \\
    2 & 2 & 2\cdot0^2 & 0
\end{array}\right]
\rightsquigarrow
\]
\[
\left[\begin{array}{ccc|c}
    1 & 2 & 8 & 0 \\
    0 & 0 & 0 & 0 \\
    2 & 2 & 0 & 0
\end{array}\right]
r_2 \ bytttes \ med \ r_3
\rightsquigarrow
\]
\[
\left[\begin{array}{ccc|c}
    1 & 2 & 8 & 0 \\
    2 & 2 & 0 & 0 \\
    0 & 0 & 0 & 0
\end{array}\right]
-2r_1 \ til \ r_2
\rightsquigarrow
\]
\[
\left[\begin{array}{ccc|c}
    1 & 2 & 8 & 0 \\
    0 & -2 & -16 & 0 \\
    0 & 0 & 0 & 0
\end{array}\right]
r_2 \cdot \brr{-\dfrac{1}{2}}
\rightsquigarrow
\]
\[
\left[\begin{array}{ccc|c}
    1 & 2 & 8 & 0 \\
    0 & 1 & 8 & 0 \\
    0 & 0 & 0 & 0
\end{array}\right]
-2r_2 \ til \ r_1
\rightsquigarrow
\]
\[
\left[\begin{array}{ccc|c}
    1 & 0 & -8 & 0 \\
    0 & 1 & 8 & 0 \\
    0 & 0 & 0 & 0
\end{array}\right]
\]
Vi kan nu aflæse løsningerne til:
\[
\left[\begin{array}{c}
    x_1 \\
    x_2 \\
    x_3
\end{array}\right]
=
t
\left[\begin{array}{c}
    8 \\
    -8 \\
    1
\end{array}\right]
\]
Vi ser nu, hvad vi får, når vi bruger den rækkereducerede totalmatrix fra tidligere, når vi erstatter $a$ med 0:
\[
\left[\begin{array}{ccc|c}
    1 & 0 & 0 & 2-a \\
    0 & 1 & 0 & \frac{(a^3 - a^2 - 4a + 8)}{(a^2 - 4)} \\
    0 & 0 & 1 & \frac{1}{(4-a^2)}
\end{array}\right]
\rightsquigarrow
\]
\[
\left[\begin{array}{ccc|c}
    1 & 0 & 0 & 2-0 \\
    0 & 1 & 0 & \frac{(0^3 - 0^2 - 4\cdot0 + 8)}{(0^2 - 4)} \\
    0 & 0 & 1 & \frac{1}{(4-0^2)}
\end{array}\right]
\rightsquigarrow
\]
\[
\left[\begin{array}{ccc|c}
    1 & 0 & 0 & 2 \\
    0 & 1 & 0 & -2 \\
    0 & 0 & 1 & \frac{1}{4}
\end{array}\right]
\]
Denne matrix antyder en unik løsning, hvilket ikke afspejler hvad vi fandt lige før, hvor vi fandt uendelig mange løsninger. Dette skete sandstnligvis fordi at den rækkereducerede totalmatrix er lavet ud fra antagelsen, at $a \neq 0$.

\subsection{}
Vi opskriver igen vores ligningssystem som en totalmatrix, og erstatter denne gang $a$ med 2:\\

$\left[\begin{array}{ccc|c}
    1 & 2 & 8 & 2 \\
    0 & 0 & 4\cdot2 & 2 \\
    2 & 2 & 2\cdot2^2 & 0
\end{array}\right]
\rightsquigarrow$\\

$\left[\begin{array}{ccc|c}
    1 & 2 & 8 & 2 \\
    0 & 0 & 8 & 2 \\
    2 & 2 & 8 & 0
\end{array}\right]
r_2 \ bytttes \ med \ r_3
\rightsquigarrow$\\

$\left[\begin{array}{ccc|c}
    1 & 2 & 8 & 2 \\
    2 & 2 & 8 & 0 \\
    0 & 0 & 8 & 2
\end{array}\right]
-2r_1 \ til \ r_2
\rightsquigarrow$\\

$\left[\begin{array}{ccc|c}
    1 & 2 & 8 & 2 \\
    0 & -2 & -8 & -4 \\
    0 & 0 & 8 & 2
\end{array}\right]
-r_3 \ til \ r_1
\rightsquigarrow$\\

$\left[\begin{array}{ccc|c}
    1 & 2 & 0 & 0 \\
    0 & -2 & -8 & -4 \\
    0 & 0 & 8 & 2
\end{array}\right]
+r_3 \ til \ r_2
\rightsquigarrow$\\

$\left[\begin{array}{ccc|c}
    1 & 2 & 0 & 0 \\
    0 & -2 & 0 & -2 \\
    0 & 0 & 8 & 2
\end{array}\right]
+r_2 \ til \ r_1
\rightsquigarrow$

$\left[\begin{array}{ccc|c}
    1 & 0 & 0 & -2 \\
    0 & -2 & 0 & -2 \\
    0 & 0 & 8 & 2
\end{array}\right]
r_2 \cdot \brr{-\dfrac{1}{2}}
\rightsquigarrow$

$\left[\begin{array}{ccc|c}
    1 & 0 & 0 & -2 \\
    0 & 1 & 0 & 1 \\
    0 & 0 & 8 & 2
\end{array}\right]
r_3 \cdot \brr{\dfrac{1}{8}}
\rightsquigarrow$

$\left[\begin{array}{ccc|c}
    1 & 0 & 0 & -2 \\
    0 & 1 & 0 & 1 \\
    0 & 0 & 1 & \frac{1}{4}
\end{array}\right]$

Vi kan nu aflæse løsningen til:
\[
\left[\begin{array}{c}
    x_1\\
    x_2\\
    x_3
\end{array}\right]
=
\left[\begin{array}{c}
    -2\\
    1\\
    \frac{1}{4}
\end{array}\right]
\]

Lad os nu se, hvad vi får, når vi bruger den rækkereducerede totalmatrix fra tidligere, når vi erstatter $a$ med 2:\\

$\left[\begin{array}{ccc|c}
    1 & 0 & 0 & 2-2 \\
    0 & 1 & 0 & \frac{(2^3 - 2^2 - 4\cdot 2 + 8)}{(2^2 - 4)} \\
    0 & 0 & 1 & \frac{1}{(4-2^2)}
\end{array}\right]
\rightsquigarrow$\\

$\left[\begin{array}{ccc|c}
    1 & 0 & 0 & 0 \\
    0 & 1 & 0 & \frac{4}{0} \\
    0 & 0 & 1 & \frac{1}{0}
\end{array}\right]
\rightsquigarrow$\\

Vi ser nu, at vi får division med 0, hvilket må betyde, at vi ikke kan bruge den rækkereducerede totalmatrix til at finde løsningerne når $a=2$.\\

\subsection{}


\section[Opgave]{Opgave}
\subsection{}


\subsection{}


\subsection{}


\subsection{}


\section[Opgave]{Opgave}
\subsection{}


\subsection{}


\subsection{}


\section[Opgave]{Opgave}

\end{document}