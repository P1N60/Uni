\documentclass[a4paper,12pt]{article}
\usepackage{standalone}
\usepackage{amsmath}
\input{../../sty/setup.sty}

% Changes sections from 1.1 to 1.a
\renewcommand{\thesubsection}{\thesection.\alph{subsection}}
\graphicspath{ {../../pictures/IDMA/IDMA3_a}} 
\title{Københavns Universitet\\
LinAlgDat - Project A}
\author{Victor Vangkilde Jørgensen - kft410\\ 
kft410@alumni.ku.dk\\
Hold 13 Mach}

\begin{document}
\author{Victor Vangkilde Jørgensen - kft410\\ 
kft410@alumni.ku.dk}
\makeatletter
\let\getauthor\@author
\let\gettitle\@title
\makeatother
\maketitle
\thispagestyle{empty}
\n\n
 

\pagebreak
\pagestyle{empty}
\tableofcontents
\pagebreak
\pagestyle{fancy}
\fancyhf{}
\setlength{\headheight}{15.2pt}
\renewcommand{\footrulewidth}{0.4pt}
\fancyhead[R]{\nouppercase \lastrightmark}
\fancyfoot[L]{\gettitle}
\fancyfoot[R]{\thepage}

\maketitle 

\section[Opgave]{Opgave}
\subsection{}
Vi omskriver ligningssystemet til totalmatrix-form:
\[
\left[\begin{array}{ccc|c}
    1 & 2 & 8 & a \\
    a & a & 4a & a \\
    2 & 2 & 2a^2 & 0
\end{array}\right]
\]
Vi benytter Gauss-Jordan elimination til at omskrive totalmatrix'en til en reduceret rækkeeechelonform.\\
Først vælger vi, at tilføje $-ar_1$ til $r_2$: 
\[
\left[\begin{array}{ccc|c}
    1 & 2 & 8 & a \\
    0 & -a & -4a & a-a^2 \\
    2 & 2 & 2a^2 & 0
\end{array}\right]
\]
Herefter tilføjer vi $-2r_1$ til $r_3$:
\[
\left[\begin{array}{ccc|c}
    1 & 2 & 8 & a \\
    0 & -a & -4a & a-a^2 \\
    0 & -2 & 2a^2-16 & -2a
\end{array}\right]
\]
Vi tilføjer $\dfrac{2r_2}{-a}$ til $r_3$:
\[
\left[\begin{array}{ccc|c}
    1 & 2 & 8 & a \\
    0 & -a & -4a & a-a^2 \\
    0 & 0 & 2a^2-8 & -2a-\frac{2(a-a^2)}{a} = -2
\end{array}\right]
\]
Vi tilføjer $\dfrac{2r_2}{a}$ til $r_1$:
\[
\left[\begin{array}{ccc|c}
    1 & 0 & 0 & 2-a \\
    0 & -a & -4a & a-a^2 \\
    0 & 0 & 2a^2-8 & -2
\end{array}\right]
\]
Vi tilføjer $\dfrac{2ar_3}{a^2-4}$ til $r_2$:
\[
\left[\begin{array}{ccc|c}
    1 & 0 & 0 & 2-a \\
    0 & -a & 0 & a - a^2 -\frac{4a}{(a^2 - 4)} \\
    0 & 0 & 2a^2-8 & -2
\end{array}\right]
\]
Vi dividerer $r_3$ med $2a^2-8$:
\[
\left[\begin{array}{ccc|c}
    1 & 0 & 0 & 2-a \\
    0 & -a & 0 & a - a^2 -\frac{4a}{(a^2 - 4)} \\
    0 & 0 & 1 & -\frac{2}{2a^2-8} = -\frac{2}{2(a^2-4)} = \frac{1}{(4-a^2)}
\end{array}\right]
\]
Til sidst dividerer vi $r_2$ med $-a$:
\[
\left[\begin{array}{ccc|c}
    1 & 0 & 0 & 2-a \\
    0 & 1 & 0 & \frac{(a^3 - a^2 - 4a + 8)}{(a^2 - 4)} \\
    0 & 0 & 1 & \frac{1}{(4-a^2)}
\end{array}\right]
\square
\]
Vi har nu fået den løsning vi ledte efter, så vi er dermed færdige.

\subsection{}
Vi opskriver igen vores ligningssystem som en totalmatrix, og erstatter denne gang $a$ med 0:
\[
\left[\begin{array}{ccc|c}
    1 & 2 & 8 & 0 \\
    0 & 0 & 4\cdot0 & 0 \\
    2 & 2 & 2\cdot0^2 & 0
\end{array}\right]
\rightsquigarrow
\]
\[
\left[\begin{array}{ccc|c}
    1 & 2 & 8 & 0 \\
    0 & 0 & 0 & 0 \\
    2 & 2 & 0 & 0
\end{array}\right]
r_2 \ bytttes \ med \ r_3
\rightsquigarrow
\]
\[
\left[\begin{array}{ccc|c}
    1 & 2 & 8 & 0 \\
    2 & 2 & 0 & 0 \\
    0 & 0 & 0 & 0
\end{array}\right]
-2r_1 \ til \ r_2
\rightsquigarrow
\]
\[
\left[\begin{array}{ccc|c}
    1 & 2 & 8 & 0 \\
    0 & -2 & -16 & 0 \\
    0 & 0 & 0 & 0
\end{array}\right]
r_2 \cdot \brr{-\dfrac{1}{2}}
\rightsquigarrow
\]
\[
\left[\begin{array}{ccc|c}
    1 & 2 & 8 & 0 \\
    0 & 1 & 8 & 0 \\
    0 & 0 & 0 & 0
\end{array}\right]
-2r_2 \ til \ r_1
\rightsquigarrow
\]
\[
\left[\begin{array}{ccc|c}
    1 & 0 & -8 & 0 \\
    0 & 1 & 8 & 0 \\
    0 & 0 & 0 & 0
\end{array}\right]
\]
Vi kan nu aflæse løsningerne til:
\[
\left[\begin{array}{c}
    x_1 \\
    x_2 \\
    x_3
\end{array}\right]
=
t
\left[\begin{array}{c}
    8 \\
    -8 \\
    1
\end{array}\right]
\]
Vi ser nu, hvad vi får, når vi bruger den rækkereducerede totalmatrix fra tidligere, når vi erstatter $a$ med 0:
\[
\left[\begin{array}{ccc|c}
    1 & 0 & 0 & 2-a \\
    0 & 1 & 0 & \frac{(a^3 - a^2 - 4a + 8)}{(a^2 - 4)} \\
    0 & 0 & 1 & \frac{1}{(4-a^2)}
\end{array}\right]
\]
Vi erstatter $a$ med 0:
\[
\left[\begin{array}{ccc|c}
    1 & 0 & 0 & 2-0 \\
    0 & 1 & 0 & \frac{(0^3 - 0^2 - 4\cdot0 + 8)}{(0^2 - 4)} \\
    0 & 0 & 1 & \frac{1}{(4-0^2)}
\end{array}\right]
\rightsquigarrow
\]
\[
\left[\begin{array}{ccc|c}
    1 & 0 & 0 & 2 \\
    0 & 1 & 0 & -2 \\
    0 & 0 & 1 & \frac{1}{4}
\end{array}\right]
\]
Denne matrix antyder en unik løsning, hvilket ikke afspejler hvad vi fandt lige før, hvor vi fandt uendelig mange løsninger. Dette skete sandstnligvis fordi at den rækkereducerede totalmatrix er lavet ud fra antagelsen, at $a \neq 0$.





\section[Opgave]{Opgave}

\section[Opgave]{Opgave}

\section[Opgave]{Opgave}

\end{document}