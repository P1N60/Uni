\documentclass[a4paper,12pt]{article}
\usepackage{standalone}
\usepackage{amsmath} % Package for advanced math typesetting



% Basic
\usepackage[T1]{fontenc}
\usepackage[utf8]{inputenc}
\usepackage{titlesec}
\titleformat{\section}
  {\normalfont\fontsize{14}{15}\bfseries}{\thesection}{1em}{}
\titleformat{\subsection}
  {\normalfont\fontsize{12}{15}\bfseries}{\thesubsection}{1em}{}

% Changes sections from 1.1 to 1.a
\renewcommand{\thesubsection}{\thesection.\alph{subsection}}

% ------------------------------------------------------------ %

% Packages
\RequirePackage{tcolorbox}
\usepackage{amsmath, amsthm, amssymb}
\usepackage{blindtext}
\usepackage{enumitem}
\usepackage{extramarks}
\usepackage{fancyhdr}
\usepackage[margin=1in]{geometry}
\usepackage{graphicx}
\usepackage{hyperref}
\usepackage{indentfirst}
\usepackage{listings}
\usepackage{mathrsfs}
\usepackage{mdframed}
\usepackage{multicol, multirow}
\usepackage{needspace, setspace}
\usepackage{paracol}
\usepackage{pgf, pgfplots}
\usepackage{tikz}
\usetikzlibrary{patterns}
\usepackage{silence}
\usepackage{xcolor}
\usepackage{bookmark}
\setlength{\parindent}{0pt}
\usetikzlibrary{patterns}
% \usepackage{subcaption}
\usetikzlibrary{decorations.pathreplacing}
\usepackage{caption}
\usepackage{subcaption}
\usepackage{xcolor}
\definecolor{maroon}{RGB}{128, 0, 0}
% ------------------------------------------------------------ %

% Spacings
\newcommand{\n}{\vspace{3mm}} % Context spacing
\newcommand{\s}{\vspace{1mm}} % Equation spacing
\newcommand{\m}{\vspace{-3mm}} % Reverse context spacing
\newcommand{\propdisp}{\pagebreak} % Proper display page break
\newcommand{\br}{\n\\\n}

% Wordings
\newcommand{\ans}[1][zb]{{\color{#1}\textit{Answer. }\hspace{3mm}}} % Answer
\newcommand{\arl}[1][zr]{{\color{#1}$\brr{\Leftarrow}$\hspace{3mm}}} % Left arrow
\newcommand{\arr}[1][zr]{{\color{#1}$\brr{\Rightarrow}$\hspace{3mm}}} % Right arrow
\newcommand{\cse}[2][zr]{{\color{#1}\textit{Case #2: }\hspace{1mm}}} % Case
\newcommand{\clm}[2][zr]{{\color{#1}$\vdash_{#2}$\hspace{1mm}}} % Claim
\newcommand{\prf}[1][zr]{{\color{#1}\textit{Proof. }\hspace{3mm}}} % Proof
\newcommand{\prt}[2][a]{\hspace{-2mm}{\color{#2}\textit{Part (#1) }}\hspace{1mm}} % Part
\newcommand{\prtc}[2][a]{\hspace{2mm}\prt[#1]{#2}} % Continued part

\newcommand{\rdft}[1][\sct]{{\color{zg}\textit{Definition #1}}} % Refer definition
\newcommand{\rexm}[1][\sct]{{\color{zb}\textit{Example #1}}} % Refer example
\newcommand{\rfig}[1][\sct]{{\color{zy}\textit{Figure #1}}} % Refer figure
\newcommand{\rpst}[1][\sct]{{\color{zr}\textit{Proposition #1}}} % Refer proposition
\newcommand{\rthm}[1][\sct]{{\color{zr}\textit{Theorem #1}}} % Refer theorem

\newcommand{\sct}{\thesection.\thescount} % Counter
\newcommand{\sctr}[2][0]{\the\numexpr\value{section}-#1\relax.\the\numexpr\value{scount}-#2\relax} % Relative counter

% Equations
\newcommand{\C}{\mathbb{C}} % Complex
\newcommand{\F}{\mathbb{F}} % Field
\newcommand{\I}{\mathbb{I}} % Irrational
\newcommand{\N}{\mathbb{N}} % Natural
\newcommand{\Q}{\mathbb{Q}} % Rational
\newcommand{\R}{\mathbb{R}} % Real
\newcommand{\Z}{\mathbb{Z}} % Integer

\newcommand{\GL}{\mathrm{GL}} % General linear group
\newcommand{\SL}{\mathrm{SL}} % Special linear group

\newcommand{\abs}[1]{\left| #1\right|} % Absolute
\newcommand{\bra}[1]{\left\langle #1\right\rangle} % Angled brackets
\newcommand{\brc}[1]{\left\{ #1\right\}} % Curly brackets
\newcommand{\brr}[1]{\left( #1\right)} % Round brackets
\newcommand{\brs}[1]{\left[ #1\right]} % Square brackets
\newcommand{\cond}[1]{\left. #1\right|} % Condition with right bar
\newcommand{\diff}{\,\mathrm{d}} % Differential
\newcommand{\erm}[1]{\;\;\;\;\text{#1}} % Equation remarks
\newcommand{\nrm}[1]{\left\| #1\right\|} % Norm
\newcommand{\srm}{\,\mid\,} % Set remarks

% Math operators
\let\Im\relax
\let\Re\relax

\DeclareMathOperator{\Im}{Im} % Imaginary function
\DeclareMathOperator{\Re}{Re} % Real function

% ------------------------------------------------------------ %

% Colours
\definecolor{gr}{RGB}{120, 120, 120} % Grey
\definecolor{zb}{RGB}{0, 38, 77} % Blue
\definecolor{zg}{RGB}{0, 77, 51} % Green
\definecolor{zp}{RGB}{51, 0, 77} % Purple
\definecolor{zr}{RGB}{77, 0, 38} % Red
\definecolor{zy}{RGB}{77, 64, 0} % Yellow

% Graphing
\usetikzlibrary{arrows}
\usetikzlibrary{calc}
\usetikzlibrary{patterns}
\pgfplotsset{compat=1.15}

% ------------------------------------------------------------ %

% Remark
\newcommand{\remark}[1]{
  \noindent\textbf{Remarks}
  
  \begin{nlist}
    \item Context of this document is based on university course \textit{\gettitle} from \textit{Department of Mathematics, The Chinese University of Hong Kong}. The original source can be found at \url{https://www.math.cuhk.edu.hk/course}. The author does not own the source.
    \item This document is assumed unavailable for unauthorized parties that have not attended the university course. It is prohibited to share, including distributing or copying this document to unauthorized parties in any means for any non-academic purpose.
    \item Context of this doucment may not be completely accurate. The author assumes no responsibility or liability for any errors or omissions in the context of this document.
    \item This document is under license CC-BY-SA 4.0. It is allowed to make any editions on this document, as long as terms of the license is not violated.
    #1
  \end{nlist}
}

% Prerequisites
\newenvironment{prereq}{
  \noindent \textbf{Prerequisites}\n

  This course requires prerequisites of
}{
  \n
}

% Reference list
\newenvironment{reflist}{
  \begin{alist}
    \item Course material from various professors associated to \textit{\gettitle}
}{
  \end{alist}
}

% ------------------------------------------------------------ %

% Environments
\newcounter{scount}[section] % Counter

\newenvironment{crl}{ % Corollary
  \parindent 0pt
  \begin{siderule}[linecolor=zr]{\color{zr}\textit{Corollary. }}
}{
  \end{siderule}
}

\newenvironment{cmt}{ % Comment
  \parindent 0pt
  \begin{siderule}[linecolor=zp]{\color{zp}\textit{Comment. }}
}{
  \end{siderule}
}

\newenvironment{dft}{ % Definition
  \parindent 0pt
  \refstepcounter{scount}
  \begin{siderule}[linecolor=zg]{\color{zg}\textit{Definition \sct. }}
}{
  \end{siderule}
}
\newenvironment{lem}{ % Lemma
  \parindent 0pt
  \refstepcounter{scount}
  \begin{siderule}[linecolor=zg]{\color{zg}\textit{Lemma \sct. }}
}{
  \end{siderule}
}

\newenvironment{exm}{ % Example
  \parindent 0pt
  \refstepcounter{scount}
  \begin{siderule}[linecolor=zb]{\color{zb}\textit{Example \sct. }}
}{
  \end{siderule}
}

\newenvironment{fig}{ % Figure
  \parindent 0pt
  \refstepcounter{scount}
  \begin{siderule}[linecolor=zy]{\color{zy}\textit{Figure \sct. }}\n
  
}{
  \end{siderule}
}

\newenvironment{prv}{ % Proof
  \parindent 0pt
  \begin{siderule}[linecolor=zr]\prf
}{
  \end{siderule}
}

\newenvironment{pst}{ % Proposition
  \parindent 0pt
  \refstepcounter{scount}
  \begin{siderule}[linecolor=zr]{\color{zr}\textit{Proposition \sct. }}
}{
  \end{siderule}
}

\newenvironment{tcn}{ % Technique
  \parindent 0pt
  \begin{siderule}[linecolor=zp]{\color{zp}\textit{Technique. }}
}{
  \end{siderule}
}

\newenvironment{thm}{ % Theorem
  \parindent 0pt
  \refstepcounter{scount}
  \begin{siderule}[linecolor=zr]{\color{zr}\textit{Sætning \sct. }}
}{
  \end{siderule}
}

\newenvironment{rmatrix}{ % Matrix in round brackets
  \left(\begin{matrix}
}{
  \end{matrix}\right)
}

\newenvironment{alist}{ % Alphabetical list
  \begin{enumerate}[label=(\alph*)]
}{
  \end{enumerate}
}

\newenvironment{Alist}{ % Capitalized alphabetical list
  \begin{enumerate}[label=(\Alph*)]
}{
  \end{enumerate}
}

\newenvironment{nlist}{ % Number list
  \begin{enumerate}[label=(\arabic*)]
}{
  \end{enumerate}
}

\newenvironment{plist}{ % Point list
  \begin{itemize}
}{
  \end{itemize}
}

\newenvironment{rlist}{ % Roman list
  \begin{enumerate}[label=(\roman*)]
}{
  \end{enumerate}
}

\newmdenv[ % Siderule line
  topline=false,
  bottomline=false,
  rightline=false,
  rightmargin=0
]{siderule}


% ------------------------------------------------------------ %
% Warning filters
\WarningFilter{mdframed}{You got a bad break}
\hfuzz=8pt


% Load required packages
\usepackage{tcolorbox}
% ------------------------------------------------------------ %

% Code

\usepackage{amsmath}
\usepackage{graphicx}
\definecolor{bluekeywords}{rgb}{0.13,0.13,1}
\definecolor{greencomments}{rgb}{0,0.5,0}
\definecolor{redstrings}{rgb}{0.9,0,0}
\definecolor{bgcolor}{rgb}{0.95,0.94,0.94}
\usepackage{listings}
\usepackage{upquote}
\usepackage{xcolor}

\lstdefinelanguage{Python}
{
  keywords={typeof, null, catch, switch, in, int, str, float, self},
  keywordstyle=\color{ForestGreen}\bfseries,
  ndkeywords={boolean, throw, import},
  ndkeywords={return, class, if ,elif, endif, while, do, else, True, False , catch, def},
  ndkeywordstyle=\color{BrickRed}\bfseries,
  identifierstyle=\color{black},
  sensitive=false,
  comment=[l]{\#},
  morecomment=[s]{/*}{*/},
  commentstyle=\color{purple}\ttfamily,
  stringstyle=\color{red}\ttfamily,
}

\lstset
{ %Formatting for code in appendix
    language=Python,
    numbers=left,
    stepnumber=1,
    showstringspaces=false,
    tabsize=1,
    breaklines=true,
    breakatwhitespace=false,
    backgroundcolor=\color{bgcolor},  % Background color
    basicstyle=\ttfamily\footnotesize, % Code font size and style
    frame=single,                    % Adds a frame around the code
    rulecolor=\color{bgcolor},       % Frame color
    breaklines=true,                 % Breaks long lines
}
 % Assuming these files exist and are correctly referenced
\graphicspath{ {../../pictures/PoP/assignment2}} % Assuming a pictures folder has been made and is correctly referenced

% \renewcommand{\thesection}{5.\arabic{section}} % Substitue 5. for any number

% Changes sections from 1.1 to 1.a
\renewcommand{\thesubsection}{\thesection.\alph{subsection}}

\begin{document}
% \includepdf[pages=-]{../../pictures/forside}

\title{Københavns Universitet\\
PoP Assignment 4}
\author{Victor Vangkilde Jørgensen}
\makeatletter
\let\getauthor\@author
\let\gettitle\@title
\makeatother
\maketitle
\thispagestyle{empty}
\n\n
 % Assuming this file contains the cover page setup

\pagebreak
\pagestyle{empty}
\tableofcontents
\pagebreak
\pagestyle{fancy}
\fancyhf{}
\setlength{\headheight}{15.2pt}
\renewcommand{\footrulewidth}{0.4pt}
% \fancyhead[R]{\nouppercase \lastrightmark}
\fancyfoot[L]{\gettitle}
\fancyfoot[R]{\thepage}
 % Assuming this file contains the header setup
\maketitle % This command will actually insert the title into the document

% Question 1 - (max 1/2 page)
\section{Question 1}

\subsection{Give a list of the external sources you used during the assignment and how you used them.
If you use a generative AI tool (e.g., copilot, chatGPT, AITutor, \ldots) you should keep a log of all the prompts you use.}

Alle mine promts og den generede kode kan findes i "promts.txt".\\
Jeg har benyttet ChatGPT4o til at opskrive type definitions'ne i "colouring.py," da vi ikke havde lært, at lave ADTs i python til forelæsningen.


% Question 2 - (max 4 pages)
\section{Question 2}

\subsection{What is a data type?\\
Explain what it is using the definitions given in PoP.
Illustrate the role data types play when representing data in Python, using examples from the thursday worksheets in week 13, 14 or 15.}

En datatype er en form for struktur, som bestemmer hvordan data skal repræsenteres.\\
Disse strukturer har begrænsninger for, hvordan deres data kan defineres.
\begin{lstlisting}
class Repeater:
    "Creates a list with 'num' elements of value: 'input'"

    def __init__(self, num):
        self.num = num
    
    def apply(self, input):
        l = []
        i = self.num
        if i <= 0:
            return l
        else:
            while i > 0:
                l.append(input)
                i += -1
            return l

print(Repeater.__doc__)
print(Repeater(0).apply(10))
print(Repeater(5).apply(20))
// OUTPUT:
Creates a list with 'num' elements of value: 'input'
[]
[20, 20, 20, 20, 20]
\end{lstlisting}
I det viste kode eksempel fra torsdags worksheet i uge 15, har jeg eksempelvis brugt den datatype, der kaldes for 'int', hvilket i python er int ræprsenteret som 32-bit heltal.\\
I mange tilfælde kan sådanne datatyper ikke interagere med hinanden, uden at blive konverteret til en af samme type først.


\subsection{Representing colours\\
Consider the colouring problem that Ken introduced in class.
In the context of this problem, describe three possible way to represent colour in Python.
Explain in detail how the different representations impact the canExtendColour function (both its signature and its body).}

\subsubsection*{1. Strings}
Farver kan repræsenteres som strings, hvor hver farve er en unik strings, eksempelvis: "rød", "grøn", og "blå".
Signaturen forbliver:
\begin{lstlisting}
canExtendColour(nr: NeighbourRelation, country1: Country, colour: Colour) -> bool
\end{lstlisting}
Funktionens body kræver ingen ændringer, da Colour allerede er en liste af strings.

\subsubsection*{2. RGB-list}
Endnu en måde er at repræsentere farver som RGB-værdier, eksempelvis: [255, 0, 0] for rød, [0, 255, 0] for grøn, og [0, 0, 255] for blå. Dette giver mulighed for at arbejde med mere præcise farver.
Signaturen ændres til:
\begin{lstlisting}
canExtendColour(nr: NeighbourRelation, country1: Country, colour: list[list[int, int, int]]) -> bool
\end{lstlisting}
hvor Colour er en liste med RGB-lister.
Funktionens body skal opdateres til at sammenligne RGB-lister i stedet for strings.

\subsubsection*{3. Heltal}
Man kunne også repræsentere farver som unikke heltal, hvor eksempelvis: 1 er rød, 2 er grøn, og 3 er blå.
Signaturen ændres til:
\begin{lstlisting}
canExtendColour(nr: NeighbourRelation, country1: Country, colour: list[int]) -> bool
\end{lstlisting}
hvor Colour er en liste med heltal.\\
Funktionens body skal opdateres til at sammenligne heltal i stedet for strings.

\subsection{URepresenting NeighbourRelation\\
Consider the colouring problem that Ken introduced in class.
In Ken's solution, the type NeighbourRelation is a list of pairs of Countries.
A list of pairs is a way to represent a graph (each pair represents an edge between the components of the pair that are vertices).
Define a NeighbourRelation type as a recursive data structure, in Python.}

\begin{lstlisting}
class NeighbourRelation:
    def __init__(self, country1: Country, country2: Country, rest=None):
        self.pair = (country1, country2)
        self.rest = rest

    def __repr__(self):
        return f"{self.pair}, {self.rest}"
    
nr = NeighbourRelation("de", "da", NeighbourRelation("da", "se", NeighbourRelation("se", "no")))
print(nr)
#OUTPUT:
('de', 'da'), ('da', 'se'), ('se', 'no'), None
\end{lstlisting}


\subsection{Colouring problem\\
Give your solution to the colouring problem in Python. For each function, describe its specification as a docstring.}

Omskrivningen af funktionerne samt deres specifications kan findes i "../src/colouring.py"

% Question 3 - (max 2 1/2 pages)
\section{Question 3\\
Consider a grid of the following form, where DIKU is written horizontally, vertically, diagonally, backwards, or even overlapping other words:}
\begin{itemize}
    \item[] \textbf{\\
    \lstinline{..D...}\\
    \lstinline{.UKID.}\\
    \lstinline{.K..K.}\\
    \lstinline{DIKU.U}\\
    \lstinline{.D....}}
\end{itemize}

\subsection{Write a program that finds all the instances of DIKU in this file Download this file.
Follow Ken's method and give the specification of each function that you define as a docstring.}
\textbf{
\begin{center}
    Kens method:
\end{center}
\begin{enumerate}
    \item Write a brief description of what the function should do
    \item Find a name for the function
    \item Write down test examples
    \item Find out the type of inputs and outputs
    \item Generate code for the function (and possibly helper functions)
    \item Write test cases
    \item Write short documentation for the fucntion
\end{enumerate}
}

Programmet bør kunne læse en given teks-fil for bogstaver, og beregne mængden af "DIKU" både horisontalt, vertikalt og baglæns. Herefter skal programmet printe mængden af "DIKU" til terminalen.\\
Eksempelvis skal programmet kunne læse bogstaverne:\\
\begin{center}
    \textbf{IIIUDDIKUI}
\end{center}
og beregne mængden af "DIKU" til 1.\\
Ideelt bør programmet læse disse rækker af bogstaver som string elementer i en liste, som kan gives til en funktion, der kan tælle mængden af "DIKU".\\

\subsection{What programming paradigm dominates in your program. Why?}

Funktionel programmering fylder en del, da det er det, jeg bruger til at læse bogstavrækkerne og sortere dem i forskellige lister.\\
Imperativ programmering bruges til at tælle mængden af "DIKU" i hvert af disse lister.

\subsection{Explain how you test your program.}

Neders i min "diku.py" fil, har jeg skrevet nogle test cases, der gør brug af input partitioning, for at finde inputs, der udløser en specification violation.\\
Jeg fandt bland andet en violation, når man giver en liste, der ikke har samme antal elementer, som elementernes string længde.

% Question 4 - (max 3 pages)
\section{Question 4\\
Consider the following problem involving a collection of cards.
Each card has two lists of numbers separated by a vertical bar (|): a list of winning numbers and then a list of numbers you have.
The first winning number you have is worth one point. Every other winning number doubles your number of points.
For example:}
\begin{itemize}
    \item[] \textbf{\\
    \lstinline{Card 1: 41 48 83 86 17 | 83 86  6 31 17  9 48 53}\\
    \lstinline{Card 2: 13 32 20 16 61 | 61 30 68 82 17 32 24 19}\\
    \lstinline{Card 3: 31 18 13 56 72 | 74 77 10 23 35 67 36 11}\\
    Card 1 is worth 8 points. Card 2 is worth 2 points. Card 3 is worth no points.}
\end{itemize}

\subsection{Write a program that counts the total number of points for all the cards in "cards.txt".
Follow Ken's method and give the specification of each function that you define as a docstring.}

Programmet bør kunne læse txt-filen "cards.txt" i mappen "data", beregne antallet af points for hvert card, og write resultatet til terminalen.\\
Eksempler på inputs og tilsvarende outputs kunne være dem som er givet i opgaven:
\begin{lstlisting}
Card 1: 41 48 83 86 17 | 83 86 6 31 17 9 48 53 # 8 Points
Card 2: 13 32 20 16 61 | 61 30 68 82 17 32 24 19 # 2 Points
Card 3: 31 18 13 56 72 | 74 77 10 23 35 67 36 11 # 0 Points
\end{lstlisting} 
Jeg har valgt, at definere en class "Card", der indtager en tekst-række som input (på den måde vist i eksempel inputsne).\\
Class'en har nogle members, der behandler data'en, og omskriver det på en måde, så vi kan tjekke, om værdierne af kortet matche vinderværdierne.\\
Antallet af points beregnes, og printes til terminalen for hvert kort.

\subsection{What programming paradigm dominates in your program. Why?}

Jeg vil sige, at funktionel programmering domminerer i mit program, da jeg sagtens kunne have undgået, at lave funktionerne i mit object Card.\\
Hvis jeg havde benyttet flere classes, ville object-orienteret programmering fylde mere. 

\subsection{Explain how you test your program.}

En af mine members hedder "cardStats". Denne member tager et kort, og viser resultatet af de fleste andre members. Jeg valgte at lave denne member, da jeg så kunne fodre den nogle inputs, og se om/hvor der sker errors.
\begin{lstlisting}
print(Card("Card   7: 51 64 21 49 30 93 60  4 10 44 |").cardStats()) # No winning numbers
print(Card("Card   8:  | 58 64 83 41 69 63 11 30 73 28 75 87 76 15  6  2 10 51 20 47 82 48 61 46 53").cardStats()) # No card numbers numbers
print(Card("Card   9:  | ").cardStats()) # Only vertical bar
print(Card("Card   10000: 49 98 13  3 61 66 16 46 22 44 | 58 26 35 75 96 64 18  9 92 45 94 67 49 89 84 20 48 39 43 38 55 90  6 69 30").cardStats()) # Large card index
print(Card("Card   9: ").cardStats()) # Empty card
\end{lstlisting}
Kun den sidste input forårsager en specification violation. 

\subsection{Explain the role of all variables in your program, using the role of variables framework introduced in PoP.}

\subsubsection*{Fixed Value}
cardRows er konstant, da det indeholder alle kortdata, som er læst fra en fil, og ændres ikke under programudførelse.\\
cardNumber bruges til at holde det aktuelle kortnummer. Selvom værdien ændres i løbet af programmet opdateres den med det samme (inkrementering med 1).

\subsubsection*{Stepper}
"i" Bruges i flere loops til at iterere gennem elementer i lister, eksempelvis i cardRemover, cardNumbers og winningNumbers.

\subsubsection*{Most Recent Holder}
"result" holder den nuværende værdi af kortdata under iterering i cardRemover.\\
head og tail holder starten og resten af listen, der bliver itereret i cardRemover.

\subsubsection*{Most Wanted Holder}
points holder den aktuelle score, der matches med kortets winningNumbers.

\subsubsection*{Container}
numbers manipulerer lister af tal fra kortdata i cardNumbers og winningNumbers.

\subsubsection*{Organizer}
numbers bruges også til at omskrive talene i methodsne.

\subsubsection*{Temporary}
head og tail bruges kun midlertidigt i cardRemover til at ændre kortdata.\\
result holder midlertidigt det ændrede kort i cardRemover.

\end{document}