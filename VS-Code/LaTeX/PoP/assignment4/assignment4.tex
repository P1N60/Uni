\documentclass[a4paper,12pt]{article}
\usepackage{standalone}
\usepackage{amsmath} % Package for advanced math typesetting
\input{../../sty/setup.sty} % Assuming these files exist and are correctly referenced
\graphicspath{ {../../pictures/PoP/assignment2}} % Assuming a pictures folder has been made and is correctly referenced

% \renewcommand{\thesection}{5.\arabic{section}} % Substitue 5. for any number

\begin{document}
% \includepdf[pages=-]{../../pictures/forside}

\title{Københavns Universitet\\
PoP Assignment 4}
\author{Victor Vangkilde Jørgensen - kft410\\ 
kft410@alumni.ku.dk}
\makeatletter
\let\getauthor\@author
\let\gettitle\@title
\makeatother
\maketitle
\thispagestyle{empty}
\n\n
 % Assuming this file contains the cover page setup

\pagebreak
\pagestyle{empty}
\tableofcontents
\pagebreak
\pagestyle{fancy}
\fancyhf{}
\setlength{\headheight}{15.2pt}
\renewcommand{\footrulewidth}{0.4pt}
\fancyhead[R]{\nouppercase \lastrightmark}
\fancyfoot[L]{\gettitle}
\fancyfoot[R]{\thepage}
 % Assuming this file contains the header setup
\maketitle % This command will actually insert the title into the document

% Question 1 - (max 1/2 page)
\section{Question 1}

\subsection{Give a list of the external sources you used during the assignment and how you used them.
If you use a generative AI tool (e.g., copilot, chatGPT, AITutor, \ldots) you should keep a log of all the prompts you use.}

% Question 2 - (max 4 pages)
\section{Question 2}

\subsection{What is a data type?\\
Explain what it is using the definitions given in PoP.
Illustrate the role data types play when representing data in Python, using examples from the thursday worksheets in week 13, 14 or 15.}

En datatype er en form for struktur, som bestemmer hvordan data skal repræsenteres.\\
Disse strukturer har begrænsninger for, hvordan deres data kan defineres.\\
\begin{lstlisting}
class Repeater:
    "Creates a list with 'num' elements of value: 'input'"

    def __init__(self, num):
        self.num = num
    
    def apply(self, input):
        l = []
        i = self.num
        if i <= 0:
            return l
        else:
            while i > 0:
                l.append(input)
                i += -1
            return l

print(Repeater.__doc__)
print(Repeater(0).apply(10))
print(Repeater(5).apply(20))
// OUTPUT:
Creates a list with 'num' elements of value: 'input'
[]
[20, 20, 20, 20, 20]
\end{lstlisting}
I det viste kode eksempel fra torsdags worksheet i uge 15, har jeg eksempelvis brugt den datatype, der kaldes for 'int' i python er int ræprsenteret som 32-bit integers.

\subsection{Representing colours\\
Consider the colouring problem that Ken introduced in class (you can refer to the announcement describing the problem and to his F\# solution Download F\# solution).
In the context of this problem, describe three possible way to represent colour in Python.
Explain in detail how the different representations impact the canExtendColour function (both its signature and its body).}



\subsection{URepresenting NeighbourRelation\\
Consider the colouring problem that Ken introduced in class (you can refer to the announcement describing the problem and to his F\# solution Download F\# solution).
In Ken's solution, the type NeighbourRelation is a list of pairs of Countries.
A list of pairs is a way to represent a graph (each pair represents an edge between the components of the pair that are vertices).
Define a NeighbourRelation type as a recursive data structure, in Python.}


\subsection{Colouring problem\\
Give your solution to the colouring problem in Python. For each function, describe its specification as a docstring.}

% Question 3 - (max 2 1/2 pages)
\section{Question 3\\
Consider a grid of the following form, where DIKU is written horizontally, vertically, diagonally, backwards, or even overlapping other words:}
\begin{itemize}
    \item[] \textbf{\\
    \lstinline{..D...}\\
    \lstinline{.UKID.}\\
    \lstinline{.K..K.}\\
    \lstinline{DIKU.U}\\
    \lstinline{.D....}}
\end{itemize}

\subsection{Write a program that finds all the instances of DIKU in this file Download this file.
Follow Ken's method and give the specification of each function that you define as a docstring.}

\subsection{What programming paradigm dominates in your program. Why?}

\subsection{Explain how you test your program.}

% Question 4 - (max 3 pages)
\section{Question 4\\
Consider the following problem involving a collection of cards.
Each card has two lists of numbers separated by a vertical bar (|): a list of winning numbers and then a list of numbers you have.
The first winning number you have is worth one point. Every other winning number doubles your number of points.
For example:}
\begin{itemize}
    \item[] \textbf{\\
    \lstinline{Card 1: 41 48 83 86 17 | 83 86  6 31 17  9 48 53}\\
    \lstinline{Card 2: 13 32 20 16 61 | 61 30 68 82 17 32 24 19}\\
    \lstinline{Card 3: 31 18 13 56 72 | 74 77 10 23 35 67 36 11}\\
    Card 1 is worth 8 points. Card 2 is worth 2 points. Card 3 is worth no points.}
\end{itemize}

\subsection{Write a program that counts the total number of points for all the cards in this input fileDownload this input file
Follow Ken's method and give the specification of each function that you define as a docstring.}

\subsection{What programming paradigm dominates in your program. Why?}

\subsection{Explain how you test your program.}

\subsection{Explain the role of all variables in your program, using the role of variables framework introduced in PoP.}

\end{document}

