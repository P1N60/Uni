% !TeX root = /Users/victor/Documents/GitHub/VS-Code/LaTeX/SS/SS2a/ss2a.tex


\documentclass[a4paper,12pt]{article}
\usepackage{standalone}
\usepackage{amsmath} % Package for advanced math typesetting
\input{../../sty/setup.sty} % Assuming these files exist and are correctly referenced
\graphicspath{ {../../pictures/PoP/assignment2}} % Assuming a pictures folder has been made and is correctly referenced

% \renewcommand{\thesection}{5.\arabic{section}} % Substitue 5. for any number

\begin{document}
% \includepdf[pages=-]{../../pictures/forside}

\title{Københavns Universitet\\
PoP Assignment 2}
\author{Victor Vangkilde Jørgensen - kft410\\ 
kft410@alumni.ku.dk}
\makeatletter
\let\getauthor\@author
\let\gettitle\@title
\makeatother
\maketitle
\thispagestyle{empty}
\n\n
 % Assuming this file contains the cover page setup

\pagebreak
\pagestyle{empty}
\tableofcontents
\pagebreak
\pagestyle{fancy}
\fancyhf{}
\setlength{\headheight}{15.2pt}
\renewcommand{\footrulewidth}{0.4pt}
\fancyhead[R]{\nouppercase \lastrightmark}
\fancyfoot[L]{\gettitle}
\fancyfoot[R]{\thepage}
 % Assuming this file contains the header setup
\maketitle % This command will actually insert the title into the document

\section{section 1}

\subsection[allah er stor]{question 1}

\subsubsection[yeet]{question 1.1}

\[
\brr{82x+2y^2\frac{1}{2}}
\]

\[
\brr{82x+2y^2\frac{1}{x}}
\]

\[
\brr{2x+2y^2\dfrac{\dfrac{4}{0}}{a_{1}}}
\]

\[
\lim_{n \to \infty}\brr{\dfrac{1}{n}}
\]

\[
\lim_{n \to \infty}\brc{\dfrac{1}{n}}
\]

\[
\text{math is cool} = true
\]

\[
e^{x\cdot0}\bullet\nabla f\circ g 
\]

\[
\begin{aligned}
1&+100000000000000000000000000000\\
20&+1
\end{aligned}
\]

$\dfrac{\partial u}{\partial x} = y\cos(xy+1) = -1\cos(1\cdot(-1)+1) = -1\cos(0) = -1$\\
$\dfrac{\partial v}{\partial x} = 2x = 2\cdot1 = 2$


\section[]{I den sidste afleveringsopgave betragtede vi funktionen $f: \R^2 \to \R$ givet ved
\[
f(x,y) = \begin{cases}
    \dfrac{x^2 y}{x^4 + y^2}, & (x,y) \neq \leq (0,0) \\
    0, & (x,y) = (0,0)
\end{cases}
\]
\\hassan}

\begin{figure} % Billeder fra Maple
    \centering
    \includegraphics[scale=0.4]{bruh skull.png}
    \caption{Maple plot af funktionen fra opgave 4: $f(x,y)=\begin{cases}\frac{x^2 y}{x^4+y^2},(x,y)\neq(0,0)\\ (0,0),(x,y)=(0,0)\end{cases}$}


    \includegraphics[scale=0.4]{bruh skull.png}
    \caption{Maple kode til at finde de partielt afledte af $f(x,y)$ ved $(x,y)=(5,5)$}
\end{figure}

\begin{lstlisting}
if (1==2) then kys
\end{lstlisting}



\end{document}

