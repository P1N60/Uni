\documentclass[a4paper,12pt]{article}
\usepackage{standalone}
\usepackage{amsmath} % Package for advanced math typesetting
\input{../../sty/setup.sty} % Assuming these files exist and are correctly referenced
\graphicspath{ {../../pictures/IDMA/IDMA2a}} % Assuming a pictures folder has been made and is correctly referenced

% \renewcommand{\thesection}{5.\arabic{section}} % Substitue 5. for any number

% Changes sections from 1.1 to 1.a
\renewcommand{\thesubsection}{\thesection.\alph{subsection}}

\begin{document}
% \includepdf[pages=-]{../../pictures/forside}

\title{Københavns Universitet\\
Introduktion til diskret matematik og algoritmer - Problem set 3}
\author{Victor Vangkilde Jørgensen - kft410\\ 
kft410@alumni.ku.dk}
\makeatletter
\let\getauthor\@author
\let\gettitle\@title
\makeatother
\maketitle
\thispagestyle{empty}
\n\n
 % Assuming this file contains the cover page setup

\pagebreak
\pagestyle{empty}
\tableofcontents
\pagebreak
\pagestyle{fancy}
\fancyhf{}
\setlength{\headheight}{15.2pt}
\renewcommand{\footrulewidth}{0.4pt}
\fancyhead[R]{\nouppercase \lastrightmark}
\fancyfoot[L]{\gettitle}
\fancyfoot[R]{\thepage}
 % Assuming this file contains the header setup
\maketitle % This command will actually insert the title into the document


\section[Question 1]{Recall that a standard deck of cards has $52$ cards partitioned into four suits (hearts,
spades, clubs, and diamonds) with $13$ ranks each (2-10 plus jack, queen, king, and ace). In this
problem, we assume that you are dealt $5$ cards from a perfectly shuffled deck of cards.}

\subsection[]{What is the probability that you get a flush, i.e., $5$ cards of the same suit but not all in
sequence with respect to rank? (Because five cards of the same suit in sequential rank
would be a straight flush.)}



\subsection[]{What is the probability that you get a straight, i.e., $5$ 
cards of sequential rank but not all of the same suit? (Because if the latter condition also held, 
we would again have a straight flush.)
}



\section[Question 2]{Prove mathematically that among all numbers on the form $11...100...0$, i.e., numbers
consisting of $m$ ones followed by $n$ zeros for some $m,n \in N^+$ (sometimes notation like $1^m0^n$ is
used to describe text strings constructed in such a way), there is some number that is divisible
by 2025. Hint: Look at all numbers $1^m = 11 ... 1$ and consider what their remainders can be modulo $2025$.}



\section[Question 3]{Let $a \in R^+$ be any positive real number. Show that for any integer $n \geq 2$ there is a
rational number $\frac{c}{d}, c,d \in \Z, d \leq n$, that approximates $a$ to within error $\frac{1}{dn}$, ie., $|a - \frac{c}{d}| \leq \frac{1}{dn}$.
Hint: Consider the numbers $a, 2a, \dots ,n \cdot a$ and show that one of these numbers is at distance at most $\frac{1}{n}$ from some integer.}



\section[Question 4]{In this problem we focus on relations. Suppose that $A = \{e_0,e_1,\dots,e_5\}$ is a set of
$6$ elements and consider the relation $R$ on $A$ represented by the matrix
\[M_R = \brr{
    \begin{bmatrix}
        0 & 0 & 1 & 0 & 0 & 0 \\
        0 & 0 & 0 & 1 & 0 & 0 \\
        0 & 0 & 0 & 0 & 1 & 0 \\
        0 & 0 & 0 & 0 & 0 & 1 \\
        1 & 0 & 0 & 0 & 0 & 0 \\
        0 & 1 & 0 & 0 & 0 & 0 \\
    \end{bmatrix}}
\]
(where element $e_i$ corresponds to the row and column $i+1$).}


\subsection[]{Let us write $S$ to denote the symmetric closure of the relation $R$. 
What is the matrix representation of $S$? Can you explain in words what the relation $S$ is by describing how it can be interpreted?
}



\subsection[]{Now let $T$ be the transitive closure of the relation $S$. What is the matrix representation of $T$? 
Can you explain in words what the relation $T$ is by describing how it can be interpreted?
}



\subsection[]{Suppose that we instead let $T^\prime$ be the transitive closure of the relation $R$, and then let
$S^\prime$ be the symmetric closure of $T^\prime$. Are $S^\prime$ and $T$ the same relation? If they are not the
same, show some way in which they differ. If they are the same, is it true that $S^\prime$ and $T$
constructed in this way from some relation $R$ on a set $A$ will always be the same? Please
make sure to motivate your answers clearly.
}



\section[Question 5]{}

\end{document}