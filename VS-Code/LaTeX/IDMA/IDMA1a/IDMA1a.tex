\documentclass[a4paper,12pt]{article}
\usepackage{standalone}
\usepackage{amsmath} % Package for advanced math typesetting



% Basic
\usepackage[T1]{fontenc}
\usepackage[utf8]{inputenc}
\usepackage{titlesec}
\titleformat{\section}
  {\normalfont\fontsize{14}{15}\bfseries}{\thesection}{1em}{}
\titleformat{\subsection}
  {\normalfont\fontsize{12}{15}\bfseries}{\thesubsection}{1em}{}

% Changes sections from 1.1 to 1.a
\renewcommand{\thesubsection}{\thesection.\alph{subsection}}

% ------------------------------------------------------------ %

% Packages
\RequirePackage{tcolorbox}
\usepackage{amsmath, amsthm, amssymb}
\usepackage{blindtext}
\usepackage{enumitem}
\usepackage{extramarks}
\usepackage{fancyhdr}
\usepackage[margin=1in]{geometry}
\usepackage{graphicx}
\usepackage{hyperref}
\usepackage{indentfirst}
\usepackage{listings}
\usepackage{mathrsfs}
\usepackage{mdframed}
\usepackage{multicol, multirow}
\usepackage{needspace, setspace}
\usepackage{paracol}
\usepackage{pgf, pgfplots}
\usepackage{tikz}
\usetikzlibrary{patterns}
\usepackage{silence}
\usepackage{xcolor}
\usepackage{bookmark}
\setlength{\parindent}{0pt}
\usetikzlibrary{patterns}
% \usepackage{subcaption}
\usetikzlibrary{decorations.pathreplacing}
\usepackage{caption}
\usepackage{subcaption}
\usepackage{xcolor}
\definecolor{maroon}{RGB}{128, 0, 0}
% ------------------------------------------------------------ %

% Spacings
\newcommand{\n}{\vspace{3mm}} % Context spacing
\newcommand{\s}{\vspace{1mm}} % Equation spacing
\newcommand{\m}{\vspace{-3mm}} % Reverse context spacing
\newcommand{\propdisp}{\pagebreak} % Proper display page break
\newcommand{\br}{\n\\\n}

% Wordings
\newcommand{\ans}[1][zb]{{\color{#1}\textit{Answer. }\hspace{3mm}}} % Answer
\newcommand{\arl}[1][zr]{{\color{#1}$\brr{\Leftarrow}$\hspace{3mm}}} % Left arrow
\newcommand{\arr}[1][zr]{{\color{#1}$\brr{\Rightarrow}$\hspace{3mm}}} % Right arrow
\newcommand{\cse}[2][zr]{{\color{#1}\textit{Case #2: }\hspace{1mm}}} % Case
\newcommand{\clm}[2][zr]{{\color{#1}$\vdash_{#2}$\hspace{1mm}}} % Claim
\newcommand{\prf}[1][zr]{{\color{#1}\textit{Proof. }\hspace{3mm}}} % Proof
\newcommand{\prt}[2][a]{\hspace{-2mm}{\color{#2}\textit{Part (#1) }}\hspace{1mm}} % Part
\newcommand{\prtc}[2][a]{\hspace{2mm}\prt[#1]{#2}} % Continued part

\newcommand{\rdft}[1][\sct]{{\color{zg}\textit{Definition #1}}} % Refer definition
\newcommand{\rexm}[1][\sct]{{\color{zb}\textit{Example #1}}} % Refer example
\newcommand{\rfig}[1][\sct]{{\color{zy}\textit{Figure #1}}} % Refer figure
\newcommand{\rpst}[1][\sct]{{\color{zr}\textit{Proposition #1}}} % Refer proposition
\newcommand{\rthm}[1][\sct]{{\color{zr}\textit{Theorem #1}}} % Refer theorem

\newcommand{\sct}{\thesection.\thescount} % Counter
\newcommand{\sctr}[2][0]{\the\numexpr\value{section}-#1\relax.\the\numexpr\value{scount}-#2\relax} % Relative counter

% Equations
\newcommand{\C}{\mathbb{C}} % Complex
\newcommand{\F}{\mathbb{F}} % Field
\newcommand{\I}{\mathbb{I}} % Irrational
\newcommand{\N}{\mathbb{N}} % Natural
\newcommand{\Q}{\mathbb{Q}} % Rational
\newcommand{\R}{\mathbb{R}} % Real
\newcommand{\Z}{\mathbb{Z}} % Integer

\newcommand{\GL}{\mathrm{GL}} % General linear group
\newcommand{\SL}{\mathrm{SL}} % Special linear group

\newcommand{\abs}[1]{\left| #1\right|} % Absolute
\newcommand{\bra}[1]{\left\langle #1\right\rangle} % Angled brackets
\newcommand{\brc}[1]{\left\{ #1\right\}} % Curly brackets
\newcommand{\brr}[1]{\left( #1\right)} % Round brackets
\newcommand{\brs}[1]{\left[ #1\right]} % Square brackets
\newcommand{\cond}[1]{\left. #1\right|} % Condition with right bar
\newcommand{\diff}{\,\mathrm{d}} % Differential
\newcommand{\erm}[1]{\;\;\;\;\text{#1}} % Equation remarks
\newcommand{\nrm}[1]{\left\| #1\right\|} % Norm
\newcommand{\srm}{\,\mid\,} % Set remarks

% Math operators
\let\Im\relax
\let\Re\relax

\DeclareMathOperator{\Im}{Im} % Imaginary function
\DeclareMathOperator{\Re}{Re} % Real function

% ------------------------------------------------------------ %

% Colours
\definecolor{gr}{RGB}{120, 120, 120} % Grey
\definecolor{zb}{RGB}{0, 38, 77} % Blue
\definecolor{zg}{RGB}{0, 77, 51} % Green
\definecolor{zp}{RGB}{51, 0, 77} % Purple
\definecolor{zr}{RGB}{77, 0, 38} % Red
\definecolor{zy}{RGB}{77, 64, 0} % Yellow

% Graphing
\usetikzlibrary{arrows}
\usetikzlibrary{calc}
\usetikzlibrary{patterns}
\pgfplotsset{compat=1.15}

% ------------------------------------------------------------ %

% Remark
\newcommand{\remark}[1]{
  \noindent\textbf{Remarks}
  
  \begin{nlist}
    \item Context of this document is based on university course \textit{\gettitle} from \textit{Department of Mathematics, The Chinese University of Hong Kong}. The original source can be found at \url{https://www.math.cuhk.edu.hk/course}. The author does not own the source.
    \item This document is assumed unavailable for unauthorized parties that have not attended the university course. It is prohibited to share, including distributing or copying this document to unauthorized parties in any means for any non-academic purpose.
    \item Context of this doucment may not be completely accurate. The author assumes no responsibility or liability for any errors or omissions in the context of this document.
    \item This document is under license CC-BY-SA 4.0. It is allowed to make any editions on this document, as long as terms of the license is not violated.
    #1
  \end{nlist}
}

% Prerequisites
\newenvironment{prereq}{
  \noindent \textbf{Prerequisites}\n

  This course requires prerequisites of
}{
  \n
}

% Reference list
\newenvironment{reflist}{
  \begin{alist}
    \item Course material from various professors associated to \textit{\gettitle}
}{
  \end{alist}
}

% ------------------------------------------------------------ %

% Environments
\newcounter{scount}[section] % Counter

\newenvironment{crl}{ % Corollary
  \parindent 0pt
  \begin{siderule}[linecolor=zr]{\color{zr}\textit{Corollary. }}
}{
  \end{siderule}
}

\newenvironment{cmt}{ % Comment
  \parindent 0pt
  \begin{siderule}[linecolor=zp]{\color{zp}\textit{Comment. }}
}{
  \end{siderule}
}

\newenvironment{dft}{ % Definition
  \parindent 0pt
  \refstepcounter{scount}
  \begin{siderule}[linecolor=zg]{\color{zg}\textit{Definition \sct. }}
}{
  \end{siderule}
}
\newenvironment{lem}{ % Lemma
  \parindent 0pt
  \refstepcounter{scount}
  \begin{siderule}[linecolor=zg]{\color{zg}\textit{Lemma \sct. }}
}{
  \end{siderule}
}

\newenvironment{exm}{ % Example
  \parindent 0pt
  \refstepcounter{scount}
  \begin{siderule}[linecolor=zb]{\color{zb}\textit{Example \sct. }}
}{
  \end{siderule}
}

\newenvironment{fig}{ % Figure
  \parindent 0pt
  \refstepcounter{scount}
  \begin{siderule}[linecolor=zy]{\color{zy}\textit{Figure \sct. }}\n
  
}{
  \end{siderule}
}

\newenvironment{prv}{ % Proof
  \parindent 0pt
  \begin{siderule}[linecolor=zr]\prf
}{
  \end{siderule}
}

\newenvironment{pst}{ % Proposition
  \parindent 0pt
  \refstepcounter{scount}
  \begin{siderule}[linecolor=zr]{\color{zr}\textit{Proposition \sct. }}
}{
  \end{siderule}
}

\newenvironment{tcn}{ % Technique
  \parindent 0pt
  \begin{siderule}[linecolor=zp]{\color{zp}\textit{Technique. }}
}{
  \end{siderule}
}

\newenvironment{thm}{ % Theorem
  \parindent 0pt
  \refstepcounter{scount}
  \begin{siderule}[linecolor=zr]{\color{zr}\textit{Sætning \sct. }}
}{
  \end{siderule}
}

\newenvironment{rmatrix}{ % Matrix in round brackets
  \left(\begin{matrix}
}{
  \end{matrix}\right)
}

\newenvironment{alist}{ % Alphabetical list
  \begin{enumerate}[label=(\alph*)]
}{
  \end{enumerate}
}

\newenvironment{Alist}{ % Capitalized alphabetical list
  \begin{enumerate}[label=(\Alph*)]
}{
  \end{enumerate}
}

\newenvironment{nlist}{ % Number list
  \begin{enumerate}[label=(\arabic*)]
}{
  \end{enumerate}
}

\newenvironment{plist}{ % Point list
  \begin{itemize}
}{
  \end{itemize}
}

\newenvironment{rlist}{ % Roman list
  \begin{enumerate}[label=(\roman*)]
}{
  \end{enumerate}
}

\newmdenv[ % Siderule line
  topline=false,
  bottomline=false,
  rightline=false,
  rightmargin=0
]{siderule}


% ------------------------------------------------------------ %
% Warning filters
\WarningFilter{mdframed}{You got a bad break}
\hfuzz=8pt


% Load required packages
\usepackage{tcolorbox}
% ------------------------------------------------------------ %

% Code

\usepackage{amsmath}
\usepackage{graphicx}
\definecolor{bluekeywords}{rgb}{0.13,0.13,1}
\definecolor{greencomments}{rgb}{0,0.5,0}
\definecolor{redstrings}{rgb}{0.9,0,0}
\definecolor{bgcolor}{rgb}{0.95,0.94,0.94}
\usepackage{listings}
\usepackage{upquote}
\usepackage{xcolor}

\lstdefinelanguage{Python}
{
  keywords={typeof, null, catch, switch, in, int, str, float, self},
  keywordstyle=\color{ForestGreen}\bfseries,
  ndkeywords={boolean, throw, import},
  ndkeywords={return, class, if ,elif, endif, while, do, else, True, False , catch, def},
  ndkeywordstyle=\color{BrickRed}\bfseries,
  identifierstyle=\color{black},
  sensitive=false,
  comment=[l]{\#},
  morecomment=[s]{/*}{*/},
  commentstyle=\color{purple}\ttfamily,
  stringstyle=\color{red}\ttfamily,
}

\lstset
{ %Formatting for code in appendix
    language=Python,
    numbers=left,
    stepnumber=1,
    showstringspaces=false,
    tabsize=1,
    breaklines=true,
    breakatwhitespace=false,
    backgroundcolor=\color{bgcolor},  % Background color
    basicstyle=\ttfamily\footnotesize, % Code font size and style
    frame=single,                    % Adds a frame around the code
    rulecolor=\color{bgcolor},       % Frame color
    breaklines=true,                 % Breaks long lines
}
 % Assuming these files exist and are correctly referenced
\graphicspath{ {../../pictures/PoP/assignment2}} % Assuming a pictures folder has been made and is correctly referenced

% \renewcommand{\thesection}{5.\arabic{section}} % Substitue 5. for any number

% Changes sections from 1.1 to 1.a
\renewcommand{\thesubsection}{\thesection.\alph{subsection}}

\begin{document}
% \includepdf[pages=-]{../../pictures/forside}

\title{Københavns Universitet\\
Introduktion til diskret matematik og algoritmer - Problem set 1}
\author{Victor Vangkilde Jørgensen}
\makeatletter
\let\getauthor\@author
\let\gettitle\@title
\makeatother
\maketitle
\thispagestyle{empty}
\n\n
 % Assuming this file contains the cover page setup

\pagebreak
\pagestyle{empty}
\tableofcontents
\pagebreak
\pagestyle{fancy}
\fancyhf{}
\setlength{\headheight}{15.2pt}
\renewcommand{\footrulewidth}{0.4pt}
% \fancyhead[R]{\nouppercase \lastrightmark}
\fancyfoot[L]{\gettitle}
\fancyfoot[R]{\thepage}
 % Assuming this file contains the header setup
\maketitle % This command will actually insert the title into the document

% Question 1
\section[Question 1]{Question 1 - In the following snippet of code A and B are arrays indexed from 1 to n that contain numbers. }

\begin{lstlisting}
for i := 1 upto n { 
    B[i] := 1 
    for j := 1 upto i {
        B[i] := B[i] * A[j] 
    }
} 
\end{lstlisting}

\subsection[]{Explain in plain language what the algorithm above does. In particular, what is the meaning of the entries B [i] after the algorithm has terminated? }

Algoritmet ændrer hvert element i array'en B til produktet af alle elementer i A op til samme nummer element (i).

\subsection[]{Provide an asymptotic analysis of the running time as a function of the array size n. (That is, state how the worst-case running time scales with n, focusing only on the highest-order term, and ignoring the constant factor in front of this term.)}

Lad os kigge på hvordan running time fordeles i vores algoritme:
\begin{lstlisting}
for i := 1 upto n {         //n running time units
    B[i] := 1               //1 running time unit
    for j := 1 upto i {     //n running time units
        B[i] := B[i] * A[j] //1 running time unit
    }
}
\end{lstlisting}
Vi ser, at vi har n gentagelser af for-loop'et, som indeholder n gentagelser i endnu et for-loop. Dette resulterer i en worst-case $O(n^2)$

\subsection[]{Can you improve the code to run faster while retaining the same functionality? How
much faster can you get the algorithm to run? Analyse the time complexity of your new
algorithm. Can you prove that it is asymptotically optimal? (That is, that no algorithm
solving this problem can run faster except possibly for a constant factor in the highest-order
term or improvements in lower-order terms. )}

For at gøre vores algoritme betydligt hurtigere, bør vi se, om vi kan få en worst-case asymptotic time complexcity på n eller bedre.\\
(Bemærk at jeg bruger en anden snytax for min egen pseudokode)
\begin{lstlisting}
B[1] := A[1]               //1 running time unit
for i := 2 upto n          //n running time units
    B[i] := B[i-1] * A[i]  //1 running time unit
\end{lstlisting} 
Dette nye algortime vedholder samme funktionalitet som det tidlige, og kører i $O(n)$ asymptotic running time, hvilket er hurtigere, da dette skalerer lineært.\\
Jeg vil selv mene, at dette er "asymptotically optimal", da jeg ikke ser en anden måde, hvorpå man kan iterere gennem hvert element i en array, som er hurtigere end lineær running time.

% Question 2
\section[Question 2]{Question 2 - In the following snippet of code A is an array indexed from 1 to n that contain elements
that can be compared}

\begin{lstlisting}
j := n
good := TRUE
while (j > 1 and good)
    i := j - 1
    while (i >= 1 and good)
        if (A[i] > A[j])
            good := FALSE
        i := i - 1
    j := j - 1
if (good)
    return "success"
else
    return "failure"
\end{lstlisting}
    
\subsection[]{Explain in plain language what the algorithm above does. In particular, what do we know
about the array A when "success" or "failure" is returned, respectively?}

Algoritmet retunerer "success" eller "failure", afhængigt af om alle elementers værdi i array'en A øges i tagt med elementnummeret, eller har samme værdi.
Når "success" er retuneret, ved vi, at array'en A kun indeholder elementer, der har samme eller højere værdi end elementet før for hvert element i A. Sagt med andre ord er elle elementer i A sorteret fra laveste til højeste værdi, hvis "success" er retuneret.\\
Når "failure" er retuneret, er der et eller flere elementer, hvis værdi ikke øges i tagt med elementnummeret eller ikke er den samme. Med andre ord er array'en ikke sorteret.
    
\subsection[]{Provide an asymptotic analysis of the running time as a function of the array size n. (That
is, state how the worst-case running time scales with n, focusing only on the highest-order
term, and ignoring the constant factor in front of this term.)}
    
\begin{lstlisting}
j := n                      //1 running time unit
good := TRUE                //1 running time unit
while (j > 1 and good)      //n running time units
    i := j - 1              //1 running time unit
    while (i >= 1 and good) //n running time units
        if (A[i] > A[j])    //1 running time unit
            good := FALSE   //1 running time unit
        i := i - 1          //1 running time unit
    j := j - 1              //1 running time unit
if (good)                   //1 running time unit
    return "success"        //1 running time unit
else                        //1 running time unit
    return "failure"        //1 running time unit
\end{lstlisting}

\subsection[]{Can you improve the code to run faster while retaining the same functionality? How
much faster can you get the algorithm to run? Analyse the time complexity of your new
algorithm. Can you prove that it is asymptotically optimal? (That is, that no algorithm
solving this problem can run faster except possibly for a constant factor in the highest-order
term or improvements in lower-order terms. )}



% Question 3
\section[Question 3]{Question 3 - In the following snippet of code A is an array indexed from 1 to n that contains integers,
and B is an auxiliary array, also indexed from 1 to n, that is meant to contain Boolean values.}

\begin{lstlisting}
for i := 1 upto n {
    if (A[i] < 1 or A[i] > n)
        return "failure"
}
i := 1
found := -1
while (i <= n and found < 0) {
    for j := 1 upto n {
        B[j] := false
    }
    j := 1
    while (B[j] == false) {
        B[j] := true
        j := A[j]
    }
    if (A[A[j]] == j)
        found := j
    i := i + 1
}
return found
\end{lstlisting}
    
\subsection[]{Explain in plain language what the algorithm above does. In particular, when does it
return a positive value, and, if it does, what is the meaning of this value?}
    
Algoritmet søgerer først for, at alle elementer i array'en A kun indeholder værdier der er imellem 1 og længden (antallet af elementer) af array'en.\\

Kort sagt retunerer algoritmet "failure" hvis A har elementer, der er mindre end 1, eller værdier, som er større end nummeret af elementer. Hvis algoritmet ikke retunerer "failure", retunerer den -1 hvis der ikke er nogle elementer, der har samme værdi som dens index, ellers retunerer den pladsen af elementer, der har samme værdi som dens index.
    
\subsection[]{Provide an asymptotic analysis of the running time as a function of the array size n. (That
is, state how the worst-case running time scales with n, focusing only on the highest-order
term, and ignoring the constant factor in front of this term.)}
    
    

\subsection[]{Can you improve the code to run faster while retaining the same functionality? How
much faster can you get the algorithm to run? Analyse the time complexity of your new
algorithm. Can you prove that it is asymptotically optimal?}



% Question 4
\section[Question 4]{Question 4 - In the following snippet of code A is an array indexed from 1 to n > 2 containing integers.}

\begin{lstlisting}
search (A, lo, hi)
    if (A[lo] >= A[hi])
        return "failure"
    else if (lo + 1 == hi)
        return lo
    else 
        mid = floor ((lo + hi) / 2)
        if (A[mid] > A[lo])
            search (A, lo, mid)
        else
            search (A, mid, hi)
\end{lstlisting}

\begin{itemize}
    \item[] \textbf{The first call to the algorithm is search (A, 1, n), where n is whatever size (at least 2) the
    array has.}
\end{itemize}

\subsection[]{Explain in plain language what the algorithm above does. If the algorithm returns some-
thing other than "failure", then what is the meaning of the value returned?}
    
    
    
\subsection[]{Provide an asymptotic analysis of the running time as a function of the array size n.}
    
    

\subsection[]{Could it be the case that recursive calls of the algorithm also return "failure", or would it
be sufficient to check just once before making the first recursive call? If we get the additional
guarantee that all elements in the array are distinct, could we remove the "failure" check
completely, since we would be guaranteed to never have this answer returned anyway?
What about if we get the additional guarantee that the array is sorted in increasing order?
What if both of these extra guarantees apply?}




\end{document}