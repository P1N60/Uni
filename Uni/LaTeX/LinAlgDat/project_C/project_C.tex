\documentclass[a4paper,12pt]{article}
\usepackage{standalone}
\usepackage{amsmath}
\input{../../sty/setup.sty}

% Changes sections from 1.1 to 1.a
\renewcommand{\thesubsection}{\thesection.\alph{subsection}}
\graphicspath{ {../../pictures/IDMA/IDMA3_a}} 
\title{Københavns Universitet\\
LinAlgDat - Project C}
\author{Victor Vangkilde Jørgensen - kft410\\ 
kft410@alumni.ku.dk\\
Hold 13 Mach}

\begin{document}
\author{Victor Vangkilde Jørgensen - kft410\\ 
kft410@alumni.ku.dk}
\makeatletter
\let\getauthor\@author
\let\gettitle\@title
\makeatother
\maketitle
\thispagestyle{empty}
\n\n
 

\pagebreak
\pagestyle{empty}
\tableofcontents
\pagebreak
\pagestyle{fancy}
\fancyhf{}
\setlength{\headheight}{15.2pt}
\renewcommand{\footrulewidth}{0.4pt}
\fancyhead[R]{\nouppercase \lastrightmark}
\fancyfoot[L]{\gettitle}
\fancyfoot[R]{\thepage}
 
\maketitle 

\section[Opgave]{Opgave}
\subsection{}


\subsection{}


\subsection{}


\subsection{}


\subsection{}


\section[Opgave]{Opgave}
\subsection{}

Vi bestemmer først $\lambda I - M$:
\[
\lambda
\left[\begin{array}{ccc}
    1 & 0 & 0\\
    0 & 1 & 0 \\
    0 & 0 & 1
\end{array}\right]
-
\left[\begin{array}{ccc}
    -2 & -6 & -3\\
    3 & 7 & 3 \\
    -6 & -12 & -5
\end{array}\right]
=
\left[\begin{array}{ccc}
    \lambda & 0 & 0\\
    0 & \lambda & 0 \\
    0 & 0 & \lambda
\end{array}\right]
-
\left[\begin{array}{ccc}
    -2 & -6 & -3\\
    3 & 7 & 3 \\
    -6 & -12 & -5
\end{array}\right]
=
\left[\begin{array}{ccc}
    \lambda+2 & 6 & 3\\
    -3 & \lambda-7 & -3 \\
    6 & 12 & \lambda+5
\end{array}\right]
\]\\

Vi udføerer de rækkeoperationer, som er givet i opgavens vink:\\

$
\left[\begin{array}{ccc}
    \lambda+2 & 6 & 3\\
    -3 & \lambda-7 & -3 \\
    6 & 12 & \lambda+5
\end{array}\right]
\begin{array}{ccc}
    +r_2\\
    \\
    \\
\end{array}
\leadsto
\left[\begin{array}{ccc}
    \lambda-1 & 6-\lambda & 0\\
    -3 & \lambda-7 & -3 \\
    6 & 12 & \lambda+5
\end{array}\right]
\begin{array}{ccc}
    \\
    \\
    +2r_2\\
\end{array}
\leadsto
\left[\begin{array}{ccc}
    \lambda-1 & 6-\lambda & 0\\
    -3 & \lambda-7 & -3 \\
    0 & 2\lambda -2 & \lambda-1
\end{array}\right]
$\\

\subsection{}


\subsection{}


\subsection{}


\subsection{}



\section[Opgave]{Opgave}
\subsection{}


\subsection{}


\subsection{}


\section[Opgave]{Opgave}
Se vedhæftede python-fil.

\end{document}