\documentclass[a4paper,12pt]{article}
\usepackage{standalone}
\usepackage{amsmath}
\input{../../sty/setup.sty}

% Changes sections from 1.1 to 1.a
\renewcommand{\thesubsection}{\thesection.\alph{subsection}}
\graphicspath{ {../../pictures/IDMA/IDMA3_a}} 
\title{Københavns Universitet\\
LinAlgDat - Project A}
\author{Victor Vangkilde Jørgensen - kft410\\ 
kft410@alumni.ku.dk\\
Hold 13 Mach}

\begin{document}
\author{Victor Vangkilde Jørgensen - kft410\\ 
kft410@alumni.ku.dk}
\makeatletter
\let\getauthor\@author
\let\gettitle\@title
\makeatother
\maketitle
\thispagestyle{empty}
\n\n
 

\pagebreak
\pagestyle{empty}
\tableofcontents
\pagebreak
\pagestyle{fancy}
\fancyhf{}
\setlength{\headheight}{15.2pt}
\renewcommand{\footrulewidth}{0.4pt}
\fancyhead[R]{\nouppercase \lastrightmark}
\fancyfoot[L]{\gettitle}
\fancyfoot[R]{\thepage}
 
\maketitle 

\section[Opgave]{Opgave}
\subsection{}
Vi omskriver ligningssystemet til totalmatrix-form:
\[
\left[\begin{array}{ccc|c}
    1 & 2 & 8 & a \\
    a & a & 4a & a \\
    2 & 2 & 2a^2 & 0
\end{array}\right]
\]
Vi benytter Gauss-Jordan elimination til at omskrive totalmatrix'en til en reduceret rækkeeechelonform.\\

$
\left[\begin{array}{ccc|c}
    1 & 2 & 8 & a \\
    a & a & 4a & a \\
    2 & 2 & 2a^2 & 0
\end{array}\right]
\begin{array}{ccc}
    \cdot 2a\\
    \cdot 2\\
    \cdot a\\
\end{array}
\leadsto
\left[\begin{array}{ccc|c}
    2a & 4a & 16a & 2a^2 \\
    2a & 2a & 8a & 2a \\
    2a & 2a & 2a^3 & 0
\end{array}\right]
\begin{array}{ccc}
    \\
    -r_1\\
    -r_1\\
\end{array}
\leadsto
$\\
$
\left[\begin{array}{ccc|c}
    2a & 4a & 16a & 2a^2 \\
    0 & -2a & -8a & 2a-2a^2 \\
    0 & -2a & 2a^3-16a & -2a^2
\end{array}\right]
\begin{array}{ccc}
    \\
    \\
    -r_2\\
\end{array}
\leadsto
\left[\begin{array}{ccc|c}
    2a & 4a & 16a & 2a^2 \\
    0 & -2a & -8a & 2a-2a^2 \\
    0 & 0 & 2a^3-8a & -2a
\end{array}\right]
\begin{array}{ccc}
    +2r_2\\
    \\
    \\
\end{array}
\leadsto
$\\
$
\left[\begin{array}{ccc|c}
    2a & 0 & 0 & 4a-2a^2 \\
    0 & -2a & -8a & 2a-2a^2 \\
    0 & 0 & 2a^3-8a & -2a
\end{array}\right]
\begin{array}{ccc}
    \\
    \cdot \frac{1}{-8a}\\
    \cdot \frac{1}{2a^3-8a}\\
\end{array}
\leadsto
\left[\begin{array}{ccc|c}
    2a & 0 & 0 & 4a-2a^2 \\
    0 & \frac{1}{4} & 1 & \frac{2a-2a^2}{-8a} \\
    0 & 0 & 1 & -\frac{2a}{2a^3-8a}
\end{array}\right]
\begin{array}{ccc}
    \\
    -r_3\\
    \\
\end{array}
\leadsto
$\\
$
\left[\begin{array}{ccc|c}
    2a & 0 & 0 & 4a-2a^2 \\
    0 & \frac{1}{4} & 0 & \frac{a^3 - a^2 - 4a + 8}{4a^2 - 16} \\
    0 & 0 & 1 & -\frac{2a}{2a^3-8a}
\end{array}\right]
\begin{array}{ccc}
    \cdot \frac{1}{2a}\\
    \cdot 4\\
    \\
\end{array}
\leadsto
\left[\begin{array}{ccc|c}
    1 & 0 & 0 & \frac{4a-2a^2}{2a} \\
    0 & 1 & 0 & \frac{4a^3 - 4a^2 - 16a + 32}{4a^2 - 16} \\
    0 & 0 & 1 & -\frac{2a}{2a^3-8a}
\end{array}\right]
reducer
\leadsto
$\\
$
\left[\begin{array}{ccc|c}
    1 & 0 & 0 & 2-a \\
    0 & 1 & 0 & \frac{a^3 - a^2 - 4a + 8}{a^2 - 4} \\
    0 & 0 & 1 & \frac{1}{4-a^2}
\end{array}\right]
$\\

Har nu reduceret til echelonform, så vi er dermed færdige.

\subsection{}
Vi opskriver igen vores ligningssystem som en totalmatrix, og erstatter denne gang $a$ med 0:\\
$
\left[\begin{array}{ccc|c}
    1 & 2 & 8 & 0 \\
    0 & 0 & 4\cdot0 & 0 \\
    2 & 2 & 2\cdot0^2 & 0
\end{array}\right]
reducer
\leadsto
\left[\begin{array}{ccc|c}
    1 & 2 & 8 & 0 \\
    0 & 0 & 0 & 0 \\
    2 & 2 & 0 & 0
\end{array}\right]
\begin{array}{cccc}
    \\
    r_2 \leftrightarrow r_3\\
    \\
\end{array}
\leadsto
\left[\begin{array}{ccc|c}
    1 & 2 & 8 & 0 \\
    2 & 2 & 0 & 0 \\
    0 & 0 & 0 & 0
\end{array}\right]
\begin{array}{ccc}
    \\
    -2r_1\\
    \\
\end{array}
\leadsto
$\\
$
\left[\begin{array}{ccc|c}
    1 & 2 & 8 & 0 \\
    0 & -2 & -16 & 0 \\
    0 & 0 & 0 & 0
\end{array}\right]
\cdot (-\frac{1}{2})
\leadsto
\left[\begin{array}{ccc|c}
    1 & 2 & 8 & 0 \\
    0 & 1 & 8 & 0 \\
    0 & 0 & 0 & 0
\end{array}\right]
\begin{array}{ccc}
    -2r_2\\
    \\
    \\
\end{array}
\leadsto
\left[\begin{array}{ccc|c}
    1 & 0 & -8 & 0 \\
    0 & 1 & 8 & 0 \\
    0 & 0 & 0 & 0
\end{array}\right]
$\\

Vi kan nu aflæse løsningerne til:
\[
\left[\begin{array}{c}
    x_1 \\
    x_2 \\
    x_3
\end{array}\right]
=
t
\left[\begin{array}{c}
    8 \\
    -8 \\
    1
\end{array}\right]
\]
Vi ser nu, hvad vi får, hvis vi bruger den rækkereducerede totalmatrix fra tidligere, når vi erstatter $a$ med 0:\\

$
\left[\begin{array}{ccc|c}
    1 & 0 & 0 & 2-a \\
    0 & 1 & 0 & \frac{(a^3 - a^2 - 4a + 8)}{(a^2 - 4)} \\
    0 & 0 & 1 & \frac{1}{(4-a^2)}
\end{array}\right]
a=0
\leadsto
\left[\begin{array}{ccc|c}
    1 & 0 & 0 & 2-0 \\
    0 & 1 & 0 & \frac{(0^3 - 0^2 - 4\cdot0 + 8)}{(0^2 - 4)} \\
    0 & 0 & 1 & \frac{1}{(4-0^2)}
\end{array}\right]
reducer
\leadsto
\left[\begin{array}{ccc|c}
    1 & 0 & 0 & 2 \\
    0 & 1 & 0 & -2 \\
    0 & 0 & 1 & \frac{1}{4}
\end{array}\right]
$\\

Denne matrix antyder en unik løsning, hvilket ikke afspejler hvad vi fandt lige før, hvor vi fandt uendelig mange løsninger. Dette skete sandstnligvis fordi at den rækkereducerede totalmatrix er lavet ud fra antagelsen, at $a \neq 0$.

\subsection{}
Vi opskriver igen vores ligningssystem som en totalmatrix, og erstatter denne gang $a$ med 2:\\

$
\left[\begin{array}{ccc|c}
    1 & 2 & 8 & 2 \\
    2 & 2 & 4\cdot2 & 2 \\
    2 & 2 & 2\cdot2^2 & 0
\end{array}\right]
reducer
\leadsto
\left[\begin{array}{ccc|c}
    1 & 2 & 8 & 2 \\
    2 & 2 & 8 & 2 \\
    2 & 2 & 8 & 0
\end{array}\right]
$\\

Vi ser med det samme, at række 2 og 3 har samme variablekonstanter, men er lig 2 forskellige værdier. Dette betyder, at der ikke er nogen løsninger til ligningssystemet, når $a=2$.\\

Lad os nu se, hvad vi får, hvis vi bruger den rækkereducerede totalmatrix fra tidligere, når vi erstatter $a$ med 2:\\

$
\left[\begin{array}{ccc|c}
    1 & 0 & 0 & 2-2 \\
    0 & 1 & 0 & \frac{(2^3 - 2^2 - 4\cdot 2 + 8)}{(2^2 - 4)} \\
    0 & 0 & 1 & \frac{1}{(4-2^2)}
\end{array}\right]
\begin{array}{ccc}
    \\
    reducer\\
    \\
\end{array}
\leadsto
\left[\begin{array}{ccc|c}
    1 & 0 & 0 & 0 \\
    0 & 1 & 0 & \frac{4}{0} \\
    0 & 0 & 1 & \frac{1}{0}
\end{array}\right]
$\\

Vi ser nu, at vi får division med 0, hvilket må betyde, at der ikke er nogen løsninger, når $a=2$. Dette må være grunden til, at totalmatricen var lavet under antagelsen, at $a\neq 2$.\\

\subsection{}
Vi omskriver ligningssystemet til koefficientmatrice-form på venstre side, hvor $a=1$, og sætter identitetmatricen på højre side. Herefter reducerer vi med Gauss-Jordan indtil, at vi har en reduceret rækkeeechelonform:\\

$
\left[\begin{array}{ccc|ccc}
    1 & 2 & 8 & 1 & 0 & 0 \\
    1 & 1 & 4\cdot 1 & 0 & 1 & 0 \\
    2 & 2 & 2\cdot 1^2 & 0 & 0 & 1
\end{array}\right]
reducer
\leadsto
\left[\begin{array}{ccc|ccc}
    1 & 2 & 8 & 1 & 0 & 0 \\
    1 & 1 & 4 & 0 & 1 & 0 \\
    2 & 2 & 2 & 0 & 0 & 1
\end{array}\right]
\begin{array}{ccc}
    \\
    -1r_1\\
    \\
\end{array}
\leadsto
$\\
$
\left[\begin{array}{ccc|ccc}
    1 & 2 & 8 & 1 & 0 & 0 \\
    0 & -1 & -4 & -1 & 1 & 0 \\
    2 & 2 & 2 & 0 & 0 & 1
\end{array}\right]
\begin{array}{ccc}
    \\
    \\
    -2r_1\\
\end{array}
\leadsto
\left[\begin{array}{ccc|ccc}
    1 & 2 & 8 & 1 & 0 & 0 \\
    0 & -1 & -4 & -1 & 1 & 0 \\
    0 & -2 & -14 & -2 & 0 & 1
\end{array}\right]
\begin{array}{ccc}
    \\
    \\
    -2r_2\\
\end{array}
\leadsto
$\\
$
\left[\begin{array}{ccc|ccc}
    1 & 2 & 8 & 1 & 0 & 0 \\
    0 & -1 & -4 & -1 & 1 & 0 \\
    0 & 0 & -6 & 0 & -2 & 1
\end{array}\right]
\begin{array}{ccc}
    \cdot 3\\
    \cdot (-3)\\
    \\
\end{array}
\leadsto
\left[\begin{array}{ccc|ccc}
    3 & 6 & 24 & 3 & 0 & 0 \\
    0 & 3 & 12 & 3 & -3 & 0 \\
    0 & 0 & -6 & 0 & -2 & 1
\end{array}\right]
\begin{array}{ccc}
    +4r_3\\
    \\
    \\
\end{array}
\leadsto
$\\
$
\left[\begin{array}{ccc|ccc}
    3 & 6 & 0 & 3 & -8 & 4 \\
    0 & 3 & 12 & 3 & -3 & 0 \\
    0 & 0 & -6 & 0 & -2 & 1
\end{array}\right]
\begin{array}{ccc}
    \\
    +2r_3\\
    \\
\end{array}
\leadsto
\left[\begin{array}{ccc|ccc}
    3 & 6 & 0 & 3 & -8 & 4 \\
    0 & 3 & 0 & 3 & -7 & 2 \\
    0 & 0 & -6 & 0 & -2 & 1
\end{array}\right]
\begin{array}{ccc}
    -2r_2\\
    \\
    \\
\end{array}
\leadsto
$\\
$
\left[\begin{array}{ccc|ccc}
    3 & 0 & 0 & -3 & 6 & 0 \\
    0 & 3 & 0 & 3 & -7 & 2 \\
    0 & 0 & -6 & 0 & -2 & 1
\end{array}\right]
\begin{array}{ccc}
    \cdot \frac{1}{3}\\
    \cdot \frac{1}{3}\\
    \cdot (-\frac{1}{6})\\
\end{array}
\leadsto
\left[\begin{array}{ccc|ccc}
    1 & 0 & 0 & -1 & 2 & 0 \\
    0 & 1 & 0 & 1 & -\frac{7}{3} & \frac{2}{3} \\
    0 & 0 & 1 & 0 & \frac{1}{3} & -\frac{1}{6}
\end{array}\right]
$\\

Vi har nu en rduceret rækkeeechelonform, så vi aflæser den inverse matrix af $A$ til:\\
\[
A^{-1}=
\left[\begin{array}{ccc}
    -1 & 2 & 0 \\
    1 & -\frac{7}{3} & \frac{2}{3} \\
    0 & \frac{1}{3} & -\frac{1}{6}
\end{array}\right]
\]
Følgende er på formen $Ax = b$:\\
\[
A
\left[\begin{array}{c}
    x\\
    y\\
    z
\end{array}\right]
=
\left[\begin{array}{ccc}
    1\\
    2\\
    3
\end{array}\right]
\]
og kan omskrives til $A^{-1}b = x$:\\
\[
\left[\begin{array}{ccc}
    -1 & 2 & 0 \\
    1 & -\frac{7}{3} & \frac{2}{3} \\
    0 & \frac{1}{3} & -\frac{1}{6}
\end{array}\right]
\left[\begin{array}{c}
    1\\
    2\\
    3
\end{array}\right]
=
\left[\begin{array}{ccc}
    3\\
    -\frac{5}{3}\\
    \frac{1}{6}
\end{array}\right]
\]
Vi ender altså med løsningen:\\
\[
\left[\begin{array}{c}
    x\\
    y\\
    z
\end{array}\right]
=
\left[\begin{array}{ccc}
    3\\
    -\frac{5}{3}\\
    \frac{1}{6}
\end{array}\right]
\]

\section[Opgave]{Opgave}
\subsection{}
Vi opskriver vores rækkeoperationer som elementærmatricer:\\
\[
E_1=
\left[\begin{array}{ccc}
    1 & 0 & 0 \\
    -2a & 1 & 0 \\
    0 & 0 & 1
\end{array}\right]
\]
\[
E_2=
\left[\begin{array}{ccc}
    1 & 0 & 0 \\
    0 & 1 & 0 \\
    0 & 0 & 5
\end{array}\right]
\]
\[
E_3=
\left[\begin{array}{ccc}
    0 & 0 & 1 \\
    0 & 1 & 0 \\
    1 & 0 & 0
\end{array}\right]
\]
\[
E_4=
\left[\begin{array}{ccc}
    1 & 0 & \frac{1}{a^2} \\
    0 & 1 & 0 \\
    0 & 0 & 1
\end{array}\right]
\]\\

Når man udfører række­operationerne i rækkefølgen $E_4 \cdot (E_3 \cdot (E_2 \cdot (E_1 \cdot A)))$ svarer det til at gange med elementærmatricerne fra venstre:\\

$
E_4 \cdot (E_3 \cdot (E_2 \cdot (E_1 \cdot A))) 
=
\left[\begin{array}{ccc}
    1 & 0 & \frac{1}{a^2} \\
    0 & 1 & 0 \\
    0 & 0 & 1
\end{array}\right]
\cdot \brr{
\left[\begin{array}{ccc}
    0 & 0 & 1 \\
    0 & 1 & 0 \\
    1 & 0 & 0
\end{array}\right]
\cdot \brr{
\left[\begin{array}{ccc}
    1 & 0 & 0 \\
    0 & 1 & 0 \\
    0 & 0 & 5
\end{array}\right]
\cdot \brr{
\left[\begin{array}{ccc}
    1 & 0 & 0 \\
    -2a & 1 & 0 \\
    0 & 0 & 1
\end{array}\right]
\cdot A}}}
$\\

Det må betyde, at en mulig opskrivning er:\\
\[
E_4  E_3  E_2  E_1 A = I
\]

da vi laver rækkeoperationer på $A$ i rækkefølgen $E_1, E_2, E_3, E_4$, og at dette er givet i opgaven, som lig enhedsmatricen.\\

Ifølge definition 2.7 på side 75 i Lineær Algebra i Datalogi, gælder der for en square matrix:\\
\[
XA = I \Rightarrow AX = I
\]
Det må dermed gælde, at:\\ 
\[
E_4  E_3  E_2  E_1 A = I \Rightarrow A E_4 E_3 E_2 E_1 = I
\]

\subsection{}
Vi bestemmer $E^{-1}_1, E^{-1}_2, E^{-1}_3, E^{-1}_4$:\\
$
\left[\begin{array}{ccc|ccc}
    1 & 0 & 0 & 1 & 0 & 0 \\
    -2a & 1 & 0 & 0 & 1 & 0 \\
    0 & 0 & 1 & 0 & 0 & 1
\end{array}\right]
\begin{array}{ccc}
    \\
    +r_1\cdot 2a\\
    \\
\end{array}
\leadsto
\left[\begin{array}{ccc|ccc}
    1 & 0 & 0 & 1 & 0 & 0 \\
    0 & 1 & 0 & 2a & 1 & 0 \\
    0 & 0 & 1 & 0 & 0 & 1
\end{array}\right]
\Rightarrow E^{-1}_1 = 
\left[\begin{array}{ccc}
    1 & 0 & 0 \\
    2a & 1 & 0 \\
    0 & 0 & 1
\end{array}\right]
$\\
$
\left[\begin{array}{ccc|ccc}
    1 & 0 & 0 & 1 & 0 & 0 \\
    0 & 1 & 0 & 0 & 1 & 0 \\
    0 & 0 & 5 & 0 & 0 & 1
\end{array}\right]
\begin{array}{ccc}
    \\
    \\
    \cdot \frac{1}{5}\\
\end{array}
\leadsto
\left[\begin{array}{ccc|ccc}
    1 & 0 & 0 & 1 & 0 & 0 \\
    0 & 1 & 0 & 0 & 1 & 0 \\
    0 & 0 & 1 & 0 & 0 & \frac{1}{5}
\end{array}\right]
\Rightarrow E^{-1}_2 = 
\left[\begin{array}{ccc}
    1 & 0 & 0 \\
    0 & 1 & 0 \\
    0 & 0 & \frac{1}{5}
\end{array}\right]
$\\
$
\left[\begin{array}{ccc|ccc}
    0 & 0 & 1 & 1 & 0 & 0 \\
    0 & 1 & 0 & 0 & 1 & 0 \\
    1 & 0 & 0 & 0 & 0 & 1
\end{array}\right]
\begin{array}{ccc}
    \\
    r_1 \leftrightarrow r_3\\
    \\
\end{array}
\leadsto
\left[\begin{array}{ccc|ccc}
    1 & 0 & 0 & 0 & 0 & 1 \\
    0 & 1 & 0 & 0 & 1 & 0 \\
    0 & 0 & 1 & 1 & 0 & 0
\end{array}\right]
\Rightarrow E^{-1}_3 = 
\left[\begin{array}{ccc}
    0 & 0 & 1 \\
    0 & 1 & 0 \\
    1 & 0 & 0
\end{array}\right]
$\\
$
\left[\begin{array}{ccc|ccc}
    1 & 0 & \frac{1}{a^2} & 1 & 0 & 0 \\
    0 & 1 & 0 & 0 & 1 & 0 \\
    0 & 0 & 1 & 0 & 0 & 1
\end{array}\right]
\begin{array}{ccc}
    -r_3 \cdot \frac{1}{a^2}\\
    \\
    \\
\end{array}
\leadsto
\left[\begin{array}{ccc|ccc}
    1 & 0 & 0 & 1 & 0 & -\frac{1}{a^2} \\
    0 & 1 & 0 & 0 & 1 & 0 \\
    0 & 0 & 1 & 0 & 0 & 1
\end{array}\right]
\Rightarrow E^{-1}_4 = 
\left[\begin{array}{ccc}
    1 & 0 & -\frac{1}{a^2} \\
    0 & 1 & 0 \\
    0 & 0 & 1
\end{array}\right]
$\\

$A$ er givet som produktet af de inverse elementærmatricer:\\
\[
A = E^{-1}_1E^{-1}_2E^{-1}_3E^{-1}_4 = 
\left[\begin{array}{ccc}
    1 & 0 & 0 \\
    2a & 1 & 0 \\
    0 & 0 & 1
\end{array}\right]
\left[\begin{array}{ccc}
    1 & 0 & 0 \\
    0 & 1 & 0 \\
    0 & 0 & \frac{1}{5}
\end{array}\right]
\left[\begin{array}{ccc}
    0 & 0 & 1 \\
    0 & 1 & 0 \\
    1 & 0 & 0
\end{array}\right]
\left[\begin{array}{ccc}
    1 & 0 & -\frac{1}{a^2} \\
    0 & 1 & 0 \\
    0 & 0 & 1
\end{array}\right]
=
\left[\begin{array}{ccc}
    0 & 0 & 1 \\
    0 & 1 & 2a \\
    \frac{1}{5} & 0 & -\frac{1}{5a^2}
\end{array}\right]
\]

\subsection{}
X er givet som række 1 i X = række 1 i $E_1 E_2 E_3 E_4$, og række 2 i X er = række 3 i $E_1 E_2 E_3 E_4$:\\
\[
E_1 E_2 E_3 E_4 = 
\left[\begin{array}{ccc}
    1 & 0 & 0 \\
    -2a & 1 & 0 \\
    0 & 0 & 1
\end{array}\right]
\left[\begin{array}{ccc}
    1 & 0 & 0 \\
    0 & 1 & 0 \\
    0 & 0 & 5
\end{array}\right]
\left[\begin{array}{ccc}
    0 & 0 & 1 \\
    0 & 1 & 0 \\
    1 & 0 & 0
\end{array}\right]
\left[\begin{array}{ccc}
    1 & 0 & \frac{1}{a^2} \\
    0 & 1 & 0 \\
    0 & 0 & 1
\end{array}\right]
=
\left[\begin{array}{ccc}
    \frac{1}{a^2} & 0 & 5 \\
    -2a & 1 & 0 \\
    1 & 0 & 0
\end{array}\right]
\]\\
\[
X =
\left[\begin{array}{ccc}
    \frac{1}{a^2} & 0 & 5 \\
    1 & 0 & 0
\end{array}\right]
\]\\

Vi skal nu bestemme, om $X$ er venstreinvers til $V$:

\[
XV = 
\left[\begin{array}{ccc}
    \frac{1}{a^2} & 0 & 5 \\
    1 & 0 & 0
\end{array}\right]
\left[\begin{array}{cc}
    0 & 1 \\
    0 & 2a \\
    \frac{1}{5} & -\frac{1}{5a^2}
\end{array}\right]
=
\left[\begin{array}{cc}
    1 & 0 \\
    0 & 1 \\
\end{array}\right]
\]

Da $XV$ giver enhedsmatricen, er $X$ venstreinvers til $V$.\\

Vi bestemmer nu alle venstreinverser til $V$, ved at lave et ligningssystem:\\
$x_3=1$\\
$x_1+2ax_2-\frac{1}{5a^2}x_3=0$\\
$\frac{1}{5}x_6=0$\\
$x_4+2ax_5-\frac{1}{5a^2}x_6=1$\\

Vi opskriver vores ligningssystem som en totalmatrix:\\

$
\left[\begin{array}{cccccc|c}
    0 & 0 & 1 & 0 & 0 & 0 & 1 \\
    1 & 2a & -\frac{1}{5a^2} & 0 & 0 & 0 & 0 \\
    0 & 0 & 0 & 0 & 0 & \frac{1}{5} & 0 \\
    0 & 0 & 0 & 1 & 2a & -\frac{1}{5a^2} & 1 \\
\end{array}\right]
\begin{array}{cccc}
    \\
    r_1 \leftrightarrow r_2\\
    \\
\end{array}
\leadsto
\left[\begin{array}{cccccc|c}
    1 & 2a & -\frac{1}{5a^2} & 0 & 0 & 0 & 0 \\
    0 & 0 & 1 & 0 & 0 & 0 & 1 \\
    0 & 0 & 0 & 0 & 0 & \frac{1}{5} & 0 \\
    0 & 0 & 0 & 1 & 2a & -\frac{1}{5a^2} & 1 \\
\end{array}\right]
\begin{array}{cccc}
    \\
    r_3 \leftrightarrow r_4\\
    \\
\end{array}
\leadsto
$\\
$
\left[\begin{array}{cccccc|c}
    1 & 2a & -\frac{1}{5a^2} & 0 & 0 & 0 & 0 \\
    0 & 0 & 1 & 0 & 0 & 0 & 1 \\
    0 & 0 & 0 & 1 & 2a & -\frac{1}{5a^2} & 1 \\
    0 & 0 & 0 & 0 & 0 & \frac{1}{5} & 0 \\
\end{array}\right]
\begin{array}{cccc}
    \\
    \\
    \\
    \cdot 5\\
\end{array}
\leadsto
\left[\begin{array}{cccccc|c}
    1 & 2a & -\frac{1}{5a^2} & 0 & 0 & 0 & 0 \\
    0 & 0 & 1 & 0 & 0 & 0 & 1 \\
    0 & 0 & 0 & 1 & 2a & -\frac{1}{5a^2} & 1 \\
    0 & 0 & 0 & 0 & 0 & 1 & 0 \\
\end{array}\right]
\begin{array}{cccc}
    +r_2 \cdot \frac{1}{5a^2}\\
    \\
    +r_4 \cdot \frac{1}{5a^2}\\
    \\
\end{array}
\leadsto
$\\
$
\left[\begin{array}{cccccc|c}
    1 & 2a & 0 & 0 & 0 & 0 & \frac{1}{5a^2} \\
    0 & 0 & 1 & 0 & 0 & 0 & 1 \\
    0 & 0 & 0 & 1 & 2a & 0 & 1 \\
    0 & 0 & 0 & 0 & 0 & 1 & 0 \\
\end{array}\right]
$\\

Vi kan nu aflæse løsningerne til:\\

\[
\left[\begin{array}{c}
    x_1 \\
    x_2 \\
    x_3 \\
    x_4 \\
    x_5 \\
    x_6
\end{array}\right]
=
\left[\begin{array}{c}
    \frac{1}{5a^2} \\
    0 \\
    1 \\
    1 \\
    0 \\
    0
\end{array}\right]
+
t
\left[\begin{array}{c}
    -2a \\
    1 \\
    0 \\
    0 \\
    0 \\
    0
\end{array}\right]
+
s
\left[\begin{array}{c}
    0 \\
    0 \\
    0 \\
    -2a \\
    1 \\
    0
\end{array}\right]
\]
For at se, om $V$ har en højre invers, skal $VX$ være lig med enhedsmatricen:\\
\[
VX =
\left[\begin{array}{cc}
    0 & 1 \\
    0 & 2a \\
    \frac{1}{5} & -\frac{1}{5a^2}
\end{array}\right]
\left[\begin{array}{ccc}
    \frac{1}{a^2} & 0 & 5 \\
    1 & 0 & 0
\end{array}\right]
=
\left[\begin{array}{ccc}
    1 & 0 & 0 \\
    2a & 0 & 0 \\
    0 & 0 & 1
\end{array}\right]
\]
Da vi ikke får enhedsmatricen, har $V$ ikke en højre invers.\\

\section[Opgave]{Opgave}
\subsection{}
Vi sætter et 1-tal i hver $e_{uv}$-element i matricen $N$ for hver edge $E(u,v)$ i grafen $G_N$:\\

\[
N = 
\left[\begin{array}{ccccc}
    0 & 1 & 1 & 0 & 0 \\
    1 & 0 & 0 & 0 & 1 \\
    1 & 0 & 0 & 0 & 0 \\
    1 & 1 & 1 & 0 & 1 \\
    1 & 0 & 0 & 1 & 0 
\end{array}\right]
\]\\

For at finde antallet af veje fra knude 4 til knude 1 med netop længde 8, skal vi kigge på element $(4,1)$ i $N^8$. Vi kan finde $N^8_{4,1}$ ved:\\
\[
N^7_{4,1} \cdot N^1_{1,1} + N^7_{4,2} \cdot N^1_{2,1} + N^7_{4,3} \cdot N^1_{3,1} + N^7_{4,4} \cdot N^1_{4,1} + N^7_{4,5} \cdot N^1_{5,1} =
\]
\[
69 \cdot 0 + 45 \cdot 1 + 45 \cdot 1 + 18 \cdot 1 + 29 \cdot 1 = 137
\]
unikke veje fra knude 4 til knude 1 med længde 8.\\

\subsection{}
Linkmatricen $A$ findes ved at tage $N^T$ og dividere hvert element med antallet af udgående edges i dens column.\\
\[
N^T =
\left[\begin{array}{ccccc}
    0 & 1 & 1 & 1 & 1 \\
    1 & 0 & 0 & 1 & 0 \\
    1 & 0 & 0 & 1 & 0 \\
    0 & 0 & 0 & 0 & 1 \\
    0 & 1 & 0 & 1 & 0
\end{array}\right]
\]


\[
A =
\left[\begin{array}{ccccc}
    0 & \frac{1}{2} & 1 & \frac{1}{4} & \frac{1}{2} \\
    \frac{1}{2} & 0 & 0 & \frac{1}{4} & 0 \\
    \frac{1}{2} & 0 & 0 & \frac{1}{4} & 0 \\
    0 & 0 & 0 & 0 & \frac{1}{2} \\
    0 & \frac{1}{2} & 0 & \frac{1}{4} & 0
\end{array}\right]
\]

\subsection{}
Vi kan lave et ligningssystem ud fra $A$:\\

$
x_1= \frac{1}{2}x_2 + x_3 + \frac{1}{4}x_4 + \frac{1}{2}x_5\\
x_2= \frac{1}{2}x_1 + \frac{1}{4}x_4 \\
x_3= \frac{1}{2}x_1 + \frac{1}{4}x_4 \\
x_4= \frac{1}{2}x_5 \\
x_5= \frac{1}{2}x_2 + \frac{1}{4}x_4
$\\

Vi omskriver så alle $x$'er ender på samme side af lighedstegnet:\\
$
-x_1 + \frac{1}{2}x_2 + x_3 + \frac{1}{4}x_4 + \frac{1}{2}x_5 = 0 \\
-x_2 + \frac{1}{2}x_1 + \frac{1}{4}x_4 = 0 \\
-x_3 + \frac{1}{2}x_1 + \frac{1}{4}x_4 = 0 \\
-x_4 + \frac{1}{2}x_5 = 0 \\
-x_5 + \frac{1}{2}x_2 + \frac{1}{4}x_4 = 0
$\\

Vi opskriver vores ligningssystem som en totalmatrix, og reducerer med Gauss-Jordan indtil vi har en reduceret rækkeeechelonform:\\
$
\left[\begin{array}{ccccc|c}
    -1 & \frac{1}{2} & 1 & \frac{1}{4} & \frac{1}{2} & 0 \\
    \frac{1}{2} & -1 & 0 & \frac{1}{4} & 0 & 0 \\
    \frac{1}{2} & 0 & -1 & \frac{1}{4} & 0 & 0 \\
    0 & 0 & 0 & -1 & \frac{1}{2} & 0 \\
    0 & \frac{1}{2} & 0 & \frac{1}{4} & -1 & 0
\end{array}\right]
\begin{array}{c}
    \cdot 4\\
    \cdot 8\\
    \cdot 8\\
    \\
    \\
\end{array}
\leadsto
\left[\begin{array}{ccccc|c}
    -4 & 2 & 4 & 1 & 2 & 0 \\
    4 & -8 & 0 & 2 & 0 & 0 \\
    4 & 0 & -8 & 2 & 0 & 0 \\
    0 & 0 & 0 & -1 & \frac{1}{2} & 0 \\
    0 & \frac{1}{2} & 0 & \frac{1}{4} & -1 & 0
\end{array}\right]
\begin{array}{ccccc}
    \\
    +r_1\\
    +r_1\\
    \\
    \\
\end{array}
\leadsto
$\\
$
\left[\begin{array}{ccccc|c}
    -4 & 2 & 4 & 1 & 2 & 0 \\
    0 & -6 & 4 & 3 & 2 & 0 \\
    0 & 2 & -4 & 3 & 2 & 0 \\
    0 & 0 & 0 & -1 & \frac{1}{2} & 0 \\
    0 & \frac{1}{2} & 0 & \frac{1}{4} & -1 & 0
\end{array}\right]
\begin{array}{ccccc}
    \\
    \\
    \cdot 3\\
    \\
    \cdot 12\\
\end{array}
\leadsto
\left[\begin{array}{ccccc|c}
    -4 & 2 & 4 & 1 & 2 & 0 \\
    0 & -6 & 4 & 3 & 2 & 0 \\
    0 & 6 & -12 & 9 & 6 & 0 \\
    0 & 0 & 0 & -1 & \frac{1}{2} & 0 \\
    0 & 6 & 0 & 3 & -12 & 0
\end{array}\right]
\begin{array}{ccccc}
    \\
    \\
    +r_2\\
    \\
    +r_2\\
\end{array}
\leadsto
$\\
$
\left[\begin{array}{ccccc|c}
    -4 & 2 & 4 & 1 & 2 & 0 \\
    0 & -6 & 4 & 3 & 2 & 0 \\
    0 & 0 & -8 & 12 & 8 & 0 \\
    0 & 0 & 0 & -1 & \frac{1}{2} & 0 \\
    0 & 0 & 4 & 6 & -10 & 0
\end{array}\right]
\begin{array}{ccccc}
    \\
    \\
    \\
    \\
    \cdot 2\\
\end{array}
\leadsto
\left[\begin{array}{ccccc|c}
    -4 & 2 & 4 & 1 & 2 & 0 \\
    0 & -6 & 4 & 3 & 2 & 0 \\
    0 & 0 & -8 & 12 & 8 & 0 \\
    0 & 0 & 0 & -1 & \frac{1}{2} & 0 \\
    0 & 0 & 8 & 12 & -20 & 0
\end{array}\right]
\begin{array}{ccccc}
    \\
    \\
    \\
    \\
    +r_3\\
\end{array}
\leadsto
$\\
$
\left[\begin{array}{ccccc|c}
    -4 & 2 & 4 & 1 & 2 & 0 \\
    0 & -6 & 4 & 3 & 2 & 0 \\
    0 & 0 & -8 & 12 & 8 & 0 \\
    0 & 0 & 0 & -1 & \frac{1}{2} & 0 \\
    0 & 0 & 0 & 24 & -12 & 0
\end{array}\right]
\begin{array}{ccccc}
    \\
    \\
    \\
    \cdot 24\\
    \\
\end{array}
\leadsto
\left[\begin{array}{ccccc|c}
    -4 & 2 & 4 & 1 & 2 & 0 \\
    0 & -6 & 4 & 3 & 2 & 0 \\
    0 & 0 & -8 & 12 & 8 & 0 \\
    0 & 0 & 0 & -24 & 12 & 0 \\
    0 & 0 & 0 & 24 & -12 & 0
\end{array}\right]
\begin{array}{ccccc}
    \\
    \\
    \\
    \\
    +r_4\\
\end{array}
\leadsto
$\\
$
\left[\begin{array}{ccccc|c}
    -4 & 2 & 4 & 1 & 2 & 0 \\
    0 & -6 & 4 & 3 & 2 & 0 \\
    0 & 0 & -8 & 12 & 8 & 0 \\
    0 & 0 & 0 & -24 & 12 & 0 \\
    0 & 0 & 0 & 0 & 0 & 0
\end{array}\right]
\begin{array}{ccccc}
    \cdot 6\\
    \cdot 2\\
    \cdot \frac{1}{2}\\
    \cdot \frac{1}{4}\\
    \\
\end{array}
\leadsto
\left[\begin{array}{ccccc|c}
    -24 & 12 & 24 & 6 & 12 & 0 \\
    0 & -12 & 8 & 6 & 4 & 0 \\
    0 & 0 & -4 & 6 & 4 & 0 \\
    0 & 0 & 0 & -6 & 3 & 0 \\
    0 & 0 & 0 & 0 & 0 & 0
\end{array}\right]
\begin{array}{ccccc}
    +r_4\\
    +r_4\\
    +r_4\\
    \\
    \\
\end{array}
\leadsto
$\\
$
\left[\begin{array}{ccccc|c}
    -24 & 12 & 24 & 0 & 15 & 0 \\
    0 & -12 & 8 & 0 & 7 & 0 \\
    0 & 0 & -4 & 0 & 7 & 0 \\
    0 & 0 & 0 & -6 & 3 & 0 \\
    0 & 0 & 0 & 0 & 0 & 0
\end{array}\right]
\begin{array}{ccccc}
    +6r_3\\
    +2r_3\\
    \\
    \\
    \\
\end{array}
\leadsto
\left[\begin{array}{ccccc|c}
    -24 & 12 & 0 & 0 & 57 & 0 \\
    0 & -12 & 0 & 0 & 21 & 0 \\
    0 & 0 & -4 & 0 & 7 & 0 \\
    0 & 0 & 0 & -6 & 3 & 0 \\
    0 & 0 & 0 & 0 & 0 & 0
\end{array}\right]
\begin{array}{ccccc}
    +r_2\\
    \\
    \\
    \\
    \\
\end{array}
\leadsto
$\\
$
\left[\begin{array}{ccccc|c}
    -24 & 0 & 0 & 0 & 78 & 0 \\
    0 & -12 & 0 & 0 & 21 & 0 \\
    0 & 0 & -4 & 0 & 7 & 0 \\
    0 & 0 & 0 & -6 & 3 & 0 \\
    0 & 0 & 0 & 0 & 0 & 0
\end{array}\right]
\begin{array}{ccccc}
    \cdot (-\frac{1}{24})\\
    \cdot (-\frac{1}{12})\\
    \cdot (-\frac{1}{4})\\
    \cdot (-\frac{1}{6})\\
    \\
\end{array}
\leadsto
\left[\begin{array}{ccccc|c}
    1 & 0 & 0 & 0 & -\frac{13}{4} & 0 \\
    0 & 1 & 0 & 0 & -\frac{7}{4} & 0 \\
    0 & 0 & 1 & 0 & -\frac{7}{4} & 0 \\
    0 & 0 & 0 & 1 & -\frac{1}{2} & 0 \\
    0 & 0 & 0 & 0 & 0 & 0
\end{array}\right]
$\\

Vi kan nu aflæse løsningerne til:
\[
\left[\begin{array}{c}
    x_1\\
    x_2\\
    x_3\\
    x_4\\
    x_5\\
\end{array}\right]
=
t
\left[\begin{array}{c}
    \frac{13}{4}\\
    \frac{7}{4}\\
    \frac{7}{4}\\
    \frac{1}{2}\\
    1\\
\end{array}\right]
\]

Vi ser nu, at side 1 er vigtigst, 2 og 3 er næst mest vigtige, 5 er næst mindst vigtig, og 4 er mindst vigtig.

\section[Opgave]{Opgave}
Se vedhæftede python-fil.

\end{document}