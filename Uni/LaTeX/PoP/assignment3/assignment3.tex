\documentclass[a4paper,12pt]{article}
\usepackage{standalone}
\usepackage{amsmath} % Package for advanced math typesetting
\input{../../sty/setup.sty} % Assuming these files exist and are correctly referenced
\graphicspath{ {../../pictures/PoP/assignment2}} % Assuming a pictures folder has been made and is correctly referenced

% \renewcommand{\thesection}{5.\arabic{section}} % Substitue 5. for any number

\begin{document}
% \includepdf[pages=-]{../../pictures/forside}

\title{Københavns Universitet\\
PoP Assignment 3}
\author{Victor Vangkilde Jørgensen - kft410\\ 
kft410@alumni.ku.dk}
\makeatletter
\let\getauthor\@author
\let\gettitle\@title
\makeatother
\maketitle
\thispagestyle{empty}
\n\n
 % Assuming this file contains the cover page setup

\pagebreak
\pagestyle{empty}
\tableofcontents
\pagebreak
\pagestyle{fancy}
\fancyhf{}
\setlength{\headheight}{15.2pt}
\renewcommand{\footrulewidth}{0.4pt}
\fancyhead[R]{\nouppercase \lastrightmark}
\fancyfoot[L]{\gettitle}
\fancyfoot[R]{\thepage}
 % Assuming this file contains the header setup
\maketitle % This command will actually insert the title into the document

% Question 1 - Reflections on group work and sources (20 points)
\section{Question 1 - Reflections on group work and sources}

\subsection{Who are the members of your group that participated in the assignment.}
Udover mig selv, har jeg udarbejdet opgaven i gruppe med:\\
- Daniel Friis-Hasché - rcb933\\
- Aksel Mannstaedt Rasmussen - qfl561

\subsection{Compare how your group worked on assignment 3 with the way you worked on assignment 2. Describe similarities and differences.}
Resten af min gruppe kom senere i gang med afleveringen, da de syntes, at afleveringen var meget svær, derudover havde vi andre afleveringer for i samme tidsramme som denne, vi havde derfor, alle sammen arbejdet på forskellige tidspunkter af afleveringsperioden.\\
Dette var ikke et problem som sådan, andet end at vi alle var på forskellige steder af afleveringen, da vi skulle tale om den. Vi talte om hvordan vi hver havde gjort brug af DIKU-Canvas, men udover det, havde vi bare talt om den basale opsetning af afleveringen. Afleveringens struktur gjorde det i højere grad svære at kommunikere om spicifikke opgaver, da selve afleveringen ikke var opdelt i opgaver.

\subsection{Give a list of the external sources you used during the assignment and how you used them.}
Jeg har benyttet \href{https://claude.ai}{Claude 3.5 Sonnet} til at genere koden til at beregne næste position:
\begin{lstlisting}
    member this.nextPos(lightOn: bool, lightPos: Vec option) = 
        let mothSpeed = 5.0

        let targetHeading =
            match lightOn, lightPos with
            | true, Some lp -> ang (sub lp this.pos)
            | _ -> this.heading + GetRandomRange -0.174 0.174 // random heading of -10 to 10 degrees in radians

        this.heading <- targetHeading

        let velocity = (mothSpeed * cos this.heading, mothSpeed * sin this.heading)

        let mutable (x, y) = this.pos
        this.pos <- add (cyclic 0.0 w x, cyclic 0.0 h y) velocity
\end{lstlisting}
\url{https://claude.ai} | Dato: 03/12/2024\\
\\
Jeg har også benyttet \href{https://claude.ai}{Claude 3.5 Sonnet} til at genere noget af koden i denne del:
\begin{lstlisting}
    let react (state: Moth list * bool) (ev: Event) : (Moth list * bool) option =
    let (mothList, lightOn) = state
    match ev with
    | Key ' ' -> Some (mothList, not lightOn)
    | TimerTick -> 
        reactAllMoths mothList lightOn
        Some (mothList, lightOn)
    | _ -> Some state 
\end{lstlisting}
\url{https://claude.ai} | Dato: 03/12/2024\\
Alt den generede kode er derefter blevet redigeret og verificeret af mig, og resten af afleveringen er skrevet af mig selv i samarbejde med min gruppe.

% Question 2 - Simulation of Moths (80 points)
\section{Question 2 - Simulation of Moths}
\begin{itemize}
    \item[] \textbf{You are to make a simulator of 5 moths that fly randomly on a cyclic domain until the light is turned on in the middle of the image, at which time the moths will move toward the light.\\
The solution must include a class with the following signature:\\
\lstinline{type Moth =}\\
\lstinline{new: pos: Vec * hdng: float -> Moth}\\
\lstinline{member heading: float}\\
\lstinline{member pos: Vec}\\
\lstinline{member draw: unit -> PrimitiveTree}\\
where Vec is a float * float pair denoting the moth's position, and hdng is its initial direction in radians. The draw function produces a DIKU-Canvas PrimitiveTree that represents the moth object at position pos. The type Vec is defined in the asteroid library presented at the lecture in Week 10, which contains helpful functionality for vector algebra and other things.\\
The moths never stop flying. The space, in which the moths fly, is to be cyclic meaning that when the moth flies out of the window on the right-hand side, then it reenters on the left and similarly for the other sides of the window. The light is to be turned on and off by pressing space. In each simulator step, a moth moves a small constant distance in the direction of its heading. When the light is on, a moth's heading is the direction of the light, and otherwise, the heading is updated in each step by adding a random number in the range of [-10,10] degrees to it.}
\end{itemize}

\subsection{Describe how your solution relies on the object-oriented programming paradigm.}
Ifølge min forståelse af OOP, har jeg benyttet classes, som Moth og Light, der er i stand til, at have bestemte functioner og members dannet til kun det objekt. Members gør det muligt, at kalde på bestemte funktioner, som ellers ikke bliver kørt.\\
Disse objekter bevares, og opdateres når vi animerer, ved brug af DIKU-Canvas. Eksempelvis kalder jeg på en member "nextPos," som beregner næste position hver frame.

\subsection{Describe the type Vec and explain how your solution makes use it.}
Jeg åbner Vectors fra asteroids namespac'et i asteroids-folderen, ved at bruge "open Asteroids.Vectors."\\
Herfra kan vi nu bruge type'en Vec og andre funktioner fra Vectors modulet. Jeg bruger selvfølgelig Vec som type af float * float, men derudover også "add" og "cyclic" til at lave skærm-grænserne.

\subsection{Using your code, detail what is happening in a simulator step. }
Der sker rigtig meget på bare et step, men her er en overordnet liste af ting der sker:
\begin{itemize}
    \item Hvis der trykkes på space, tændes lyset hvis det ikke er tændt, ellers slukkes lyset.
    \item Hvis lyset er tændt, beregner moth'en en ny retning $\brr {targetHeading}$, der peger mod lyset.
    \item Hvis lyset er slukket, tilføjes en tilfældig "drejning" til moth'ens retning for at simulere tilfældig bevægelse.
    \item Ud fra retningen beregnes en ny position baseret på moth'ens hastighed.
    \item Positionen opdateres, og moth'en bevæger sig på en måde, der sikrer, at den bliver inden for skærmens grænser (ved hjælp af cyclic).
    \item Hver moth tegnes ved at lave en cirkel omkring dens nuværende position
    \item Hvis lyset er tændt, tegnes en stor gul cirkel i midten af skærmen.
    \item Alle moth'nes tegninger kombineres, og lyset tilføjes, hvis det er tændt.
    \item Den samlede tegning (alle moths og lyset) opdagteres og vises på skærmen.
\end{itemize}
Alt dette sker $\frac{60}{1000}$ gange pr. sekundt, og da der efter standard går 1000 ticks for 1 sekundt, svarer dette til 60 gentagelser pr. sekundt.

\end{document}

