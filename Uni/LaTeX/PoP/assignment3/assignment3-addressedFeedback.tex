\documentclass[a4paper,12pt]{article}
\usepackage{standalone}
\usepackage{amsmath} % Package for advanced math typesetting
\input{../../sty/setup.sty} % Assuming these files exist and are correctly referenced
\graphicspath{ {../../pictures/PoP/assignment2}} % Assuming a pictures folder has been made and is correctly referenced

% \renewcommand{\thesection}{5.\arabic{section}} % Substitue 5. for any number

\begin{document}
% \includepdf[pages=-]{../../pictures/forside}

\title{Københavns Universitet\\
PoP Assignment 3 - Adressed Feedback}
\author{Victor Vangkilde Jørgensen - kft410\\ 
kft410@alumni.ku.dk}
\makeatletter
\let\getauthor\@author
\let\gettitle\@title
\makeatother
\maketitle
\thispagestyle{empty}
\n\n
 % Assuming this file contains the cover page setup

\pagebreak
\pagestyle{empty}
\tableofcontents
\pagebreak
\pagestyle{fancy}
\fancyhf{}
\setlength{\headheight}{15.2pt}
\renewcommand{\footrulewidth}{0.4pt}
\fancyhead[R]{\nouppercase \lastrightmark}
\fancyfoot[L]{\gettitle}
\fancyfoot[R]{\thepage}
 % Assuming this file contains the header setup
\maketitle % This command will actually insert the title into the document

% Feedback on Question 2
\section{Feedback on Question 2}

\subsection{Prøv at ændre Moth klassen så den har den samme signatur som i opgavebeskrivelsen, dvs. kun de
methods og properties der er beskrevet.}
Jeg har forsøgt, at flytte positions-beregningen ud af moth klassen, så den kun indeholder de nødtvendige members.\\
Til dette har jeg defineret en ny funktion:
\begin{lstlisting}
let moveMoth (moth: Moth) (lightOn: bool) (lightPos: Vec option) =
    let mothSpeed = 5.0

    let targetHeading =
        match lightOn, lightPos with
        | true, Some lp -> ang (sub lp moth.pos)
        | _ -> moth.heading + GetRandomRange -0.174 0.174 // random heading of -10 to 10 degrees in radians

    moth.heading <- targetHeading

    let velocity = (mothSpeed * cos moth.heading, mothSpeed * sin moth.heading)

    let mutable (x, y) = moth.pos
    moth.pos <- add (cyclic 0.0 w x, cyclic 0.0 h y) velocity
\end{lstlisting}

% Feedback on question 2B
\section{Feedback on question 2B}

\subsection{[-] The relation between the signature and implementation file is understood}
Såvidt jeg har forstået, bruger vi signature og implementation filerne i asteroids, hvor implementation filen indeholder funktionsdefinationerne, og signature filen indeholder valuations og andre types.

% Feedback on question 2C
\section{Feedback on question 2C}

\subsection{The description demonstrates the role of the update step in the diku-canvas interact function. Her er det en god ide at tilføje noget kode til din forklaring, da det gør det nemmere at forstå.}
Jeg har nu tilføjet en del flere kommentarer i min kode.

\end{document}

